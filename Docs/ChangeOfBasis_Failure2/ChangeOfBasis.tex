\documentclass[12pt]{article}

\usepackage{amsmath}
\usepackage{amssymb}
\usepackage{amsthm}

\title{Change Of Basis Transformations\\In\\Geometric Algebra}
\author{Spencer T. Parkin}

\newcommand{\G}{\mathbb{G}}
\newcommand{\V}{\mathbb{V}}
\newcommand{\R}{\mathbb{R}}
\newcommand{\A}{\mathbb{A}}
\newcommand{\B}{\mathbb{B}}

\newtheorem{theorem}{Theorem}[section]
\newtheorem{definition}{Definition}[section]
\newtheorem{corollary}{Corollary}[section]
\newtheorem{identity}{Identity}[section]
\newtheorem{lemma}{Lemma}[section]
\newtheorem{result}{Result}[section]

\begin{document}
\maketitle

\begin{abstract}
A method of representing and performing change-of-basis transformations in geometric algebra is given.
This type of transformation is equivilant to matrix multiplication.  Since shear and non-uniform scale
operations can be represented by change-of-basis transformations, it follows that the method provides
a way to perform these transformations using geometric algebra.  The change-of-basis transformation is
developed for 2-dimensional space.  A generalization to higher-dimensional spaces is considered.  A
generalization to tensor products is also considered.
\end{abstract}

In this paper we will let $\V^4$ denote a 4-dimensional Euclidean vector space spanned by
the set of orthonormal basis vectors $\{e_0,e_1,e_2,e_3\}$.  $\G(\V^4)$ will denote the
geometric algebra generated by this vector space, and we will let $I=e_0e_1e_2e_3$ be
the unit-psuedo scalar of $\G(\V^4)$.

So what is a change-of-basis transformation?  Consider the following matrix equation.
\begin{equation*}
\left[\begin{array}{cc} \alpha_x & \alpha_y \\ \beta_x & \beta_y \end{array}\right]
\left[\begin{array}{c} \gamma_x \\ \gamma_y \end{array}\right] =
\gamma_x\left[\begin{array}{c} \alpha_x \\ \beta_x \end{array}\right] +
\gamma_y\left[\begin{array}{c} \alpha_y \\ \beta_y \end{array}\right]
\end{equation*}
The reader will recognize this as ordinary matrix multiplication, but might not immediately recognize
this as a change-of-basis transformation.  Let $a,b,c\in\V^4$ be the vectors $a=\alpha_x e_0+\alpha_y e_1$,
$b=\beta_x e_0 + \beta_y e_1$ and $c=\gamma_x e_0 + \gamma_y e_1$.  If we can then be allowed a few small
abuses of matrix notation, the above equation becomes much clearer with these variables.
For example, in the context of matrices, let $c$ denote the row-vector
matrix $[ \begin{array}{cc} \gamma_x & \gamma_y \end{array} ]$.  Let us now rewrite
the above equation as follows.
\begin{equation*}
\left[\begin{array}{c} a \\ b\end{array}\right]((c\cdot e_0)e_0^T+(c\cdot e_1)e_1^T) =
(c\cdot e_0)a^T + (c\cdot e_1)b^T
\end{equation*}
We see now that the $2\times 2$ matrix above, acting on $c^T$, gives us a result that changes the
basis upon which the coordinates of $c$ are based.  That is, the basis $\{e_0,e_1\}$ is swapped
out in favor of $\{a,b\}$.  Using the language of geometric
algebra, we can achieve the same effect.

We begin with an idea set forth in $\cite{Hestenese1993}$.  In matrix algebra the change-of-basis
transformation is represented by a matrix whose rows or columns contain the desired basis.
With geometric algebra we may let a bivector $M$ represent the same transformation.
\begin{equation*}
M = ae_2 + be_3
\end{equation*}
Let us now define $\A$ as the 2-dimensional Euclidean vector space spanned by the
vectors $\{e_0,e_1\}$, and $\B$ as the 2-dimensional Euclidean vector space that
is the complement of $\A$ with respect to $\V^4$.  Clearly, $\B$ is spanned by $\{e_2,e_3\}$.

What we'll show now is that any change-of-basis transformation can be performed
in $\A$ if we provide a way to perform the change-of-basis transformation of a vector
taken from $\A$ to a vector taken from $\B$.  Such a transformation is simply given
by an isomorphism between these two spaces that maps $e_0$ to $e_2$ and $e_1$ to $e_3$.
To see why, let us define $f:\A\to\B$ to be such an isomorphism, and then $T:\A\to\A$
as follows.
\begin{equation*}
T(c) = M\cdot f(c)
\end{equation*}
As the reader can easily verify, $T(c) = (c\cdot e_0)a + (c\cdot e_1)b$, which is the
desired transformation of $c$ using $M$.

One way to define $f$ is using a rotor that rotates the 2-blade $e_0\wedge e_1$ into $e_2\wedge e_3$.
As such, $f$ is not only an isomorphism, but also an outermorphism.  This property will become essential
to finding inverse change-of-basis transformations.  For any $g\in\G(\V^4)$, we may define
$f:\G(\V^4)\to\G(V^4)$ as
$f(g)=Rg\tilde{R}$, where $R$ is the unit rotor
\begin{equation*}
R = \frac{1}{2}\left(1-e_0e_2-e_1e_3-I\right).
\end{equation*}
The reader can check that $f(e_0)=e_2$ and $f(e_1)=e_3$, that $f$ is a linear
transformation from $\A$ to $\B$, and that $f$ preserves the outer product.

We now make the observation that while $T(g)$ gives us the desired transformation,
it does not benefit from the invertability of the geometric product.  Let us therefore,
for any $g\in\G(\V^4)$, define $F:\G(\V^4)\to\G(\V^4)$ as
\begin{equation*}
F(g) = Mf(g).
\end{equation*}
For all vectors $c\in\V^4$, we see that $F(c)=T(c) + M\wedge f(c)$.  Interestingly, this
can be rewritten as
\begin{equation*}
F(c) = T(c) + \left(T(c)\cdot Me_2\wedge Me_3\right)i_{\B},
\end{equation*}
where we will let $i_{\A}$ denote the unit psuedo-scalar of $\G(\A)$ and $i_{\B}$ denote
the unit psuedo-scalar of $\G(\B)$.  Specifically, $i_{\A}=e_0e_1$ and $i_{\B}=e_2e_3$.
Here it is easy to see that the vector $T(c)\cdot Me_2\wedge Me_3$ is a $\pi/2$ radians
rotation of $T(c)$ in the plane determined by the basis in $M$.

As we know from linear algebra, a 2-dimensional change-of-basis transformation is invertible
if and only if the two axes are non-parallel.

\bibliographystyle{plain}
\bibliography{ChangeOfBasis}

\end{document}