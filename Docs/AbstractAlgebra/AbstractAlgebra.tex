\documentclass[12pt]{article}

\usepackage{amsmath}
\usepackage{amssymb}
\usepackage{amsthm}
\usepackage{graphicx}
\usepackage{float}

\title{Abstract Algebra Exercises}

\begin{document}
\maketitle

\section*{Chapter 1}

\subsection*{Exercise 5}

For $n\geq 3$, describe the elements of $D_n$.  How many elements does $D_n$ have?

The group $D_n$, when $n\geq 3$, will have $n$ rotation operations and $n$ reflections operations.
So the group will have order $2n$.  The group $D_2$ has a 2 rotation and 2 reflection operations that
are the same, so it must have order 2.  The group $D_1$ has order 1.

\subsection*{Exercise 6}

In $D_n$, explain geometrically why a reflection followed by a reflection must be a rotation.

Rotations preserve the winding order of the $n$-gon, but reflections do not.  An even
number of reflection will leave the winding order of the $n$-gon invariant.  Then since the
rotations are the set of all winding preserving operations, two successive reflections
must be a rotation.

\subsection*{Exercise 7}

In $D_n$, explain geometrically why a rotation followed by a rotation must be a rotation.

Because the set of all rotations in $D_n$ forms its own sub-group.

\subsection*{Exercise 8}

In $D_n$, explain geometrically why a rotation and a reflection taken together in either order must be a reflection.

An odd number of reflections combined with any number of rotations does not preserve winding order.
The only non-winding-order-preserving operations are the reflections.  So any rotation and
reflection combination must be a reflection.

\end{document}