\documentclass[12pt]{article}

\usepackage{amsmath}
\usepackage{amssymb}
\usepackage{amsthm}

\title{Change Of Basis Transformations\\In\\Geometric Algebra}
\author{Spencer T. Parkin}

\newcommand{\G}{\mathbb{G}}
\newcommand{\V}{\mathbb{V}}
\newcommand{\R}{\mathbb{R}}
\newcommand{\A}{\mathbb{A}}
\newcommand{\B}{\mathbb{B}}

\newtheorem{theorem}{Theorem}[section]
\newtheorem{definition}{Definition}[section]
\newtheorem{corollary}{Corollary}[section]
\newtheorem{identity}{Identity}[section]
\newtheorem{lemma}{Lemma}[section]
\newtheorem{result}{Result}[section]

\begin{document}
\maketitle

Let $\V^{2n}$ denote a $2n$-dimensional Euclidean vector space, and
$\A$ denote any $n$-dimensional Euclidean vector sub-space of $\V^{2n}$.
Let $\B$ denote the complement of $\A$ with respect to $\V^{2n}$.
Let $\{e_k\}_{k=1}^n$ be a set of $n$ Euclidean vectors
forming an orthonormal basis for $\A$.  Let $\{e_{k+n}\}_{k=1}^n$
be a set of $n$ Euclidean vectors forming an orthonormal basis for $\B$.
We will work in the geometric algebra $\G(\V^{2n})$.

Defining the function $r$ as the rotor $r(a,b)=\frac{\sqrt{2}}{2}(1-a\wedge b)$, where $a$ and
$b$ are vectors, we see that
\begin{equation*}
R = \prod_{k=1}^n r(e_k,e_{k+n})
\end{equation*}
is a unit-rotor rotating the blade $\bigwedge_{k=1}^n e_k$ into $\bigwedge_{k=1}^n e_{k+n}$, or vice-versa.
More to the point, $R$ also effectly performs a change of basis transformation of any
vector taken from $\A$ to its counter-part in $\B$, or vice-versa.  This is illustrated in the
following equation.
\begin{equation*}
R\left(\sum_{k=1}^n (a\cdot e_k)e_k\right)\tilde{R} = \sum_{k=1}^n (a\cdot e_k)e_{k+n}
\end{equation*}
Here, $a$ is any vector taken from $\A$ and $Ra\tilde{R}$ is in $\B$.

Unfortunately, we cannot employ the same technique we have used here in the
formulation of $R$ to formulate an element $E$ such that for any vector $a\in\A$,
the element $E$ takes $a$ to any given basis as $EaE^{-1}$.  Whether there
is such an element in a geometric algebra will be left as an open question for now,
but what will be said here is that clearly $E$ cannot be a rotor, because rotors
are angle preserving.  We want to be able to perform change of basis transformations
between basis sets that are not necessarily rotations of one another.

Defining the function $f:\A\to\B$ as $f(a)=Ra\tilde{R}$, we see that $f$ is an
outermorephism between $\A$ and $\B$.  Now let $\{m_k\}_{k=1}^n$ be any
set of $n$ linearly independent vectors taken from $\A$ and let $M$ be the
bivector
\begin{equation*}
M = \sum_{k=1}^n e_{k+n}\wedge m_k.
\end{equation*}
It then follows that the function $F:\A\to\A$ defined as
\begin{equation*}
F(a) = f(a)\cdot M
\end{equation*}
performs any change of basis transformation.  This includes those
that perform shear and non-uniform scale transformations.  To be more specific,
the vector $a\in\A$ written as
\begin{equation*}
a = \sum_{k=1}^n (a\cdot e_k)e_k
\end{equation*}
is transformed to
\begin{equation*}
F(a) = \sum_{k=1}^n (a\cdot e_k)m_k.
\end{equation*}
There are a few problems with $F$,
however.  The inner product is not an invertible product and $M$ does
not have an inverse with respect to the geometric product.  This means
that when we go to solve the problem of finding the bivector $M'$ such
that $F^{-1}(a) = M'\cdot f(a)$, geometric algebra offers no obvious
way to solve for $M'$.  Never-the-less, we will make do with what we have here.

Taking a lesson from linear algebra, we will define $\det(M)$ as
\begin{equation*}
\det(M) = \bigwedge_{k=1}^n F(e_k)\cdot I_{\A},
\end{equation*}
where $I_{\A}$ is the unit psuedo-scalar of $\G(\A)$.  Similarly, we will
define $I_{\B}$ as the unit psuedo-scalar of $\G(\B)$.

\end{document}