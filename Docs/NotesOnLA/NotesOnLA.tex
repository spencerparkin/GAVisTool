\documentclass[12pt]{article}

\usepackage{amsmath}
\usepackage{amssymb}
\usepackage{amsthm}
\usepackage{accents}

\title{Notes On Linear Algebra}
\author{Spencer T. Parkin}

\newcommand{\G}{\mathbb{G}}
\newcommand{\V}{\mathbb{V}}
\newcommand{\R}{\mathbb{R}}
\newcommand{\A}{\mathbb{A}}
\newcommand{\B}{\mathbb{B}}
\newcommand{\uf}{\underaccent{\bar}{f}}
\newcommand{\of}{\bar{f}}

\newtheorem{theorem}{Theorem}[section]
\newtheorem{definition}{Definition}[section]
\newtheorem{corollary}{Corollary}[section]
\newtheorem{identity}{Identity}[section]
\newtheorem{lemma}{Lemma}[section]
\newtheorem{result}{Result}[section]

\begin{document}
\maketitle

This paper is a formal compilation of all my notes on linear algebra.
It proceeds as a treatment of linear algebra guided by the material
given in the reference section, but here in some places I go into greater detail so as
to understand more fully what is being communicated in those sources.
The paper $\cite{hestenes91}$ is studied in particular.
\nocite{macdonald12}
\nocite{hestenes85}

\section{Enter Linear Functions}

Linear algebra is the study of linear functions defined on linear spaces.
Linear spaces are more often refered to as vector spaces, suggesting
to the mind a geometric interpretation of the elements of such spaces.
In fact, what we'll find is that geometric algebra facilitates the study
of linear algebra.

Holding the definitions of a vector space and linear independence as
already known, we begin with a formal definition of a linear function.
We arbitrarily define all vector spaces over the field of real numbers $\R$.
\begin{definition}
A function $f:\A\to\B$ is a mapping from a vector space $\A$ to
a vector space $\B$ with the property of preserving both scalar-vector
multiplication and vector addition.  That is, for any scalar $\lambda\in\R$,
and any two vectors $x,y\in\A$, we have $f(\lambda x)=\lambda f(x)$ and
$f(x+y)=f(x)+f(y)$.
\end{definition}
It is not enitrely clear to me how much, if any, loss in generality we incur
by restricting our study of such functions to those that map to and from
the same vector space.  Nevertheless, since this is the class of linear
functions for which I am most interested, we will proceed with this restriction.

That said, let $f:\V^n\to\V^n$ be the linear function we will study, where $\V^n$
denotes an $n$-dimensional vector space.  Then, for any $x\in\V^n$, right
away we learn two interesting things about linear functions.  Letting $\{e_i\}_{i=1}^n\subset\V^n$
be any set of $n$ linearly independent vectors taken from $\V^n$, we have
\begin{equation}\label{equ_linear_transform}
f(x) = f\left(\sum_{k=1}^n x_i e_i\right) = \sum_{k=1}^n x_i f(e_i),
\end{equation}
where $x_i = x\cdot e_i$.  The first thing this shows is that any linear function
is determined entirely by how it transforms the set of basis vectors $\{e_i\}_{i=1}^n$,
so that when we're faced with formulating a linear transformation, we need only
consider how it transforms a basis of $\V^n$.  The second thing this shows is
that every linear function is a change of basis transformation.  That is, the
set of coordinates $\{x_i\}_{i=1}^n$ for a vector $x$ are preserved while
the set of basis vectors $\{e_i\}_{i=1}^n$ are replaced with a new set
of vectors $\{f(e_i)\}_{i=1}^n$.

\section{Existance and Uniqueness of Linear Function Inverses}

Equation $\eqref{equ_linear_transform}$ also brings to bair immediate
implications on the invertibility of $f$.  That is, we have enough to
prove at this point that $f$ is invertible if and only if $f$ preserves
linear independence in the sense that if $\{e_i\}_{i=1}^n$ is a
linearly independent set, then so is $\{f(e_i)\}_{i=1}^n$.  Many graphics
transformations can be easily formulated this way.

Recall that one direction of the statement $x=y\iff f(x)=f(y)$ is the
requirement of a well defined function, while the other direction is optional,
and is the condition upon which $f^{-1}$ exists.  Specifically, if there
exist distinct vectors $x,y\in\V^n$ such that $f(x)=f(y)$, then
$f^{-1}$ does not exist.  If $z=f(x)=f(y)$, then do we let $f^{-1}(z)=x$
or $f^{-1}(z)=y$?

Suppose for the moment that the set $\{f(e_i)\}_{i=1}^n$ is linearly dependent.
Then, without loss of generality, we can write $f(e_n)$
as $\sum_{i=1}^{n-1}\lambda_i f(e_i)$, where each $\lambda_i\in\R$.
Now let $x_n=0$ and let $y\in\V^n$ be $\sum_{i=1}^n y_i e_i$, where $y_n\neq 0$,
and for all $i<n$, let $y_i=x_i-\lambda_i y_n$.  Clearly $x\neq y$, and we see that
\begin{align*}
f(x) &= \sum_{i=1}^{n-1} x_if(e_i)
 = \sum_{i=1}^{n-1}(y_i+\lambda_i y_n)f(e_i) \\
 &= \sum_{i=1}^{n-1}y_i f(e_i) + y_n\sum_{i=1}^{n-1}\lambda_i f(e_i)
 = \sum_{i=1}^{n-1}y_i f(e_i) + y_n f(e_n)
 = f(y),
\end{align*}
showing that $f$ is non-invertible.

Now suppose that $\{f(e_i)\}_{i=1}^n$ is linearly independent.  We must show
that for any $x,y\in\V^n$, if $f(x)=f(y)$, then $x=y$.  This follows immediately
from the equation
\begin{equation*}
0 = f(x)-f(y) = \sum_{i=1}^n(x_i-y_i)f(e_i),
\end{equation*}
because we must have for all integers $i\in[1,n]$, $x_i=y_i$ on the grounds
that $\{f(e_i)\}_{i=1}^n$ is a linearly independent set.

Having now established the conditions upon which $f^{-1}$ exists, let's quickly
prove the uniqueness of $f^{-1}$.  Suppose the functions $g$ and $h$ are
distinct inverses of $f$.  By distinct, this must mean that there exists $y\in f(\V^n)$
such that $g(y)\neq h(y)$.  Let $x\in\V^n$ be such that $f(x)=y$.  We then
have $x=g(f(x))=g(y)\neq h(y)=h(f(x))=x$, which is a contradiction.
Inverses of functions in general are therefore unique.

\section{Enter Outermorphic Functions}

To this point we have satisfied the basic questions of existance and uniqueness
for inverses of linear functions.  Given a linear function $f$, what we
would now hope to be able to do is find $f^{-1}$.
This is where geometric algebra comes into play.
\begin{definition}
A function $f$ is called an outermorphism if it preserves
the outer product.  That is, for any two vectors $x,y\in\V^n$,
we have $f(x\wedge y)=f(x)\wedge f(y)$.
\end{definition}
Every linear function $f$ can be extended to an outermorphism which
we denote by $\uf$.  If $f$ does not already possess the defining
characteristic of an outermorphism, (i.e., $f\neq\uf$), then it can be
extended to an outermorphism by simply defining
\begin{equation*}
\uf(B) = \bigwedge_{i=1}^s f(b_i),
\end{equation*}
where $B$ is the $s$-blade $\bigwedge_{i=1}^s b_i$, and by defining for all
scalars $\beta\in\R$, $\uf(\beta)=\beta$.  (If somehow $f$ preserved the outer
product for some grades, but not all, our extension to $\uf$ could be selective
by grade.  In $\cite{hestenes91}$ it is claimed that every linear function $f$ has a unique
extension to $\uf$.  This needs further justification in my mind.)
Clearly this extension of $f$ preserves the
original function and is therefore both linear and outermorphic.  Furthermore, if
we find $\uf^{-1}$, then we have also found $f^{-1}$.

Taking zero as being a blade of any grade, it is clear that outermorphic functions are
grade preserving.  From what we already know so far, it is clear that if a linear outermorphism
also preserves non-zero blades as non-zero blades, then it must be invertible.

\section{The Adjoint Outermorphism}

Given a linear function $f$, of particular
interest to us will be the function $\of$ implicity defined as
\begin{equation*}
x\cdot f(y) = \of(x)\cdot y,
\end{equation*}
where $x,y\in\V^n$.  I believe we call this the adjoint of $f$.
An explicit definition of $\of$ is therefore given by
\begin{equation*}
\of(x) = \sum_{i=1}^n (\of(x)\cdot e_i)e_i = \sum_{i=1}^n (x\cdot f(e_i))e_i,
\end{equation*}
which deserves careful comparison to equation $\eqref{equ_linear_transform}$.
Can we relate the concept of reciprical basis here?

As one can check, $\of$ is clearly a linear function.  The extension of this
function to an outermorphism, which we'll also denote by $\of$, is what
I believe we call the adjoint outermorphism of $f$.  An interesting reformulation
of the $\of$ is given by
\begin{align*}
\of(x) &= \sum_{i=1}^n e_i(x\cdot f(e_i)) \\
 &= \sum_{i=1}^n e_i(x\cdot\partial_i f(y)) \\
 &= \sum_{i=1}^n e_i\partial_i(x\cdot f(y)) \\
 &= \nabla_y(x\cdot f(y)),
\end{align*}
where here we must be careful to note that $\nabla_y$ denotes the gradient of its operand
with respect to $y$, not the directional derivative of its operand in the direction $y$.

We will now proceed to uncover an important property of the outermorphic extenstion
of $f$ and its adjoint outermorphism.  Letting $B$ be the $s$-blade $\bigwedge_{i=1}^s b_i$,
we see that
\begin{align*}
x\cdot \uf(B) &= -\sum_{i=1}^s(-1)^i(x\cdot\uf(b_i))\bigwedge_{j=1,j\neq i}^s\uf(b_j) \\
 &= -\sum_{i=1}^s(-1)^i(\of(x)\cdot b_i)\bigwedge_{j=1,j\neq i}^s\uf(b_j) \\
 &= \uf\left(-\sum_{i=1}^s(-1)^i(\of(x)\cdot b_i)\bigwedge_{j=1,j\neq i}^s b_j\right) \\
 &= \uf(\of(x)\cdot B).
\end{align*}
Now let $A$ be the $r$-blade $\bigwedge_{i=1}^r a_i$, where $r\leq s$, and recall
that
\begin{equation}\label{equ_dot_recursive}
A\cdot B = \left(\bigwedge_{i=1}^{r-1} a_i\right)\cdot(a_r\cdot B).
\end{equation}
By repeated application of this identity, we get
\begin{equation*}
A\cdot B = a_1\cdot\dots\cdot a_r\cdot B,
\end{equation*}
where here it is understood that the inner product is meant to be right-to-left associative.
Similarly, we may write
\begin{equation*}
A\cdot\uf(B) = \left(\bigwedge_{i=1}^{r-1}\of(a_i)\right)\cdot\uf(\of(a_r)\cdot B),
\end{equation*}
and again repeatedly apply the identity $\eqref{equ_dot_recursive}$ recursively
to obtain
\begin{equation*}
A\cdot\uf(B) = \uf(\of(a_1)\cdot\dots\cdot\of(a_r)\cdot B) = \uf(\of(A)\cdot B),
\end{equation*}
where again the inner product here is understood to be right-to-left associative.

\section{The Inverse Outermorphism}

We're now ready to find the inverse of an outermorphism.  We begin by making the
observation that
\begin{equation*}
\lambda AI = A\uf(I) = \uf(\of(A)I),
\end{equation*}
where here $\lambda\in\R$ and $I$ is the unit psuedo-scalar of $\G(\V^n)$.
In fact, we define $\det f=\lambda = \uf(I)I^{-1}$.  It follows that
\begin{equation*}
(\det f)\uf^{-1}(AI) = \of(A)I.
\end{equation*}
Solving $\uf^{-1}(A)$, we get
\begin{equation*}
\uf^{-1}(A) = \frac{\of(AI^{-1})I}{\det f} = \frac{\of(AI)I}{\det f}(-1)^{n(n-1)/2} = \frac{\of(AI)I^{-1}}{\det f}.
\end{equation*}
When $A\in\V^n$, this gives us the inverse function $f^{-1}$!  That is, for all $x\in\V^n$,
we have
\begin{equation}\label{equ_inverse_formula}
f^{-1}(x) = \frac{\of(xI)I^{-1}}{\det f}.
\end{equation}
It can be shown that $f^{-1}$ is outermorphic.
A proof of this is given in $\cite{hestenes91}$ which siffices me, so I will
not repeat it here.
The implication of this is that the inverse of the outermorphic
extension of $f$ is the outermorphic extension of the inverse of $f$.

\section{An Application Of Linear Function Inverses}

It is common to represent an invertible linear transformation as a set of basis vectors $\{a_i\}_{i=1}^n$ of $\V^n$.
Matrices, for example, represent linear transformations this way, the basis vectors taking up the rows or
columns of a matrix, and matrix multiplication is, by definition, a
change of basis transformation.  Using what we now know about linear algebra, we see that
the set of basis vectors $\{f^{-1}(e_i)\}_{i=1}^n$ must represent the inverse change of basis transformation
where $f(x)=\sum_{i=1}^n (x\cdot e_i)a_i$.  Using equation $\eqref{equ_inverse_formula}$, we can find
a formula for $f^{-1}(e_i)$.
Attempting to do so, we get
\begin{equation*}
f^{-1}(e_i) = \frac{-(-1)^i}{\det f}\left(\bigwedge_{j=1,j\neq i}^n\sum_{k=1}^n (e_j\cdot a_k)e_k\right)I^{-1}.
\end{equation*}
Admittedly, this isn't as reduced as I would like to see it, but it is interesting to me none-the-less.  Geometric algebra
has given us a formula for the inverse basis in terms of the original basis.

\section{Eigen Blades}

Geometric algebra naturally generalizes the idea of eigen vectors to eigen blades.  We have already
seen one example of this.  The unit psuedo-scalar $I$ is an eigen blade of $\uf$ with eigen value $\det f$,
provided $\uf$ is invertible.  This is because eigen values must be non-zero by definition.
\begin{definition}
A blade $B\in\G(\V^n)$ is an eigen blade of an outermorphism $\uf$ if there exists a non-zero scalar $\lambda\in\R$
such that $\uf(B)=\lambda B$.  We call $\lambda$ the eigen value of the eigen balde $B$.
\end{definition}
If we allowed zero to be an eigen value, then only non-zero eigen blades with zero eigen values
would arrise in non-invertible outermorphisms.  This is because we must have $\uf(0)=0$.

Interestingly, we can show that if $B$ is an eigen blade of $\uf$, and $\uf$ is invertible, then $B$ also represents
an invarient sub-space of $f$.  Indeed, for any vector $b\in B$, which is to say that $b\wedge B=0$, we have
\begin{equation*}
0 = \uf(b\wedge B) = f(b)\wedge\uf(B) = f(b)\wedge \lambda B,
\end{equation*}
showing that $f(b)\in B$ since $\lambda\neq 0$.  We can now conclude that if $\B$ denotes the sub-space represented
by $B$, then the image $f(\B)\subseteq \B$.  Now since $\uf$ is invertible, we must have $f(\B)=\B$.
To see this, let $B$ be the $m$-blade $\bigwedge_{i=1}^m b_i$.  Then for all $b_i\in B$, we
must have $f(b_i)\in B$, and we know that $\{f(b_i)\}_{i=1}^m$ is a linearly independent set.

Letting $\det\uf=\det f$, it makes sense to consider $\det\of$.  In fact, we find that
\begin{align*}
\det\of &= \uf(\det\of) = \uf(\of(I) I^{-1}) = \uf(\of(I)(-1)^{n(n-1)/2}I) \\
 &= \uf(\of((-1)^{n(n-1)/2}I) I) = \uf(\of(I^{-1}) I) = I^{-1}\uf(I) = \det\uf.
\end{align*}
On the subject of eigen blades, $\cite{hestenes85}$ refers to eigen blades of $\of$ as right eigen blades.
Similarly, we may call eigen blades of $\uf$ as left eigen blades.  A blade that is eigen in both senses
is called a proper eigen blade.  Clearly $\uf$ is invertible if and only if $I$ is a proper eigen blade.

There is much more to learn about eigen blades, such as the idea of factoring them into
eigen blades that have no eigen blade factors.  Perhaps I'll study this when I'm more capable
of following the disucssion in $\cite{hestenes85}$.

\bibliographystyle{plain}
\bibliography{NotesOnLA}

% Also ref David C. Lay and the GA for CS book.

\end{document}