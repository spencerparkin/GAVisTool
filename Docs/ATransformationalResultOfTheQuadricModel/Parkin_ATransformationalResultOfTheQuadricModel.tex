\documentclass{birkjour}

\usepackage{amsmath}
\usepackage{amssymb}
\usepackage{amsthm}
\usepackage{graphicx}
\usepackage{float}

\newtheorem{thm}{Theorem}[section]
 \newtheorem{cor}[thm]{Corollary}
 \newtheorem{lem}[thm]{Lemma}
 \newtheorem{prop}[thm]{Proposition}
 \theoremstyle{definition}
 \newtheorem{defn}[thm]{Definition}
 \theoremstyle{remark}
 \newtheorem{rem}[thm]{Remark}
 \newtheorem*{ex}{Example}
 \numberwithin{equation}{section}

\newcommand{\G}{\mathbb{G}}
\newcommand{\V}{\mathbb{V}}
\newcommand{\Vb}{\mathbb{\overline{V}}}
\newcommand{\W}{\mathbb{W}}
\newcommand{\R}{\mathbb{R}}
\newcommand{\Alpha}{A}

\begin{document}

\title{A Transformational Result Of The Quadric Model}

\author{Spencer T. Parkin}
\address{%
2113 S. Claremont Dr.\\
Bountiful, Utah  84010\\
USA}
\email{spencer.parkin@gmail.com}

\numberwithin{equation}{section}

\subjclass{Primary 14J70; Secondary 14J29}

\keywords{Quadric Surface, Geometric Algebra, Quadric Model}

%\dedicatory{To Melinda and Naomi}

\begin{abstract}
An important feature of the conformal model is found
to be possessed by the quadric model.  Specifically, it is
shown in this paper that the action of a versor on a point
reveals the action of this versor on any geometry of the model.
This leads to a possible direction in which we might look for a better model of quadric surfaces.
\end{abstract}

\maketitle

\section{Introduction}

It is well known that, with the exceptoin of flat points, all geometries of
the conformal model may be expressed directly as the outer product
of one or more vectors representative of points.  It then immediately
follows by the outermorphic property of versor conjugation, (see
equation \eqref{equ_versor_identity_wedge} below), that the
action of a versor on a conformal point reveals the action of this versor
on any point on the surface of a conformal geometry, and therefore
the conformal geometry itself.  This feature of the conformal model
facilitates the search for versors performing desired actions, and the
analysis of what action a given versor performs.  In this paper we
will find that the quadric model possess its own version of this very feature.
We'll also find that this feature may help point us in a direction of where
we might look for a better model of quadric geometry.

The definitions and results of \cite{Parkin12} will be assumed for
the remainder of this paper so that no background will need be
given before we can dive into the new material.

\section{The Versor Identities Of Geometric Algebra}

There are two well known identities in geometric algebra involving versors.
For any two vectors $p,b\in\W$ and any versor $V\in\G$, we have
\begin{equation}\label{equ_versor_identity_wedge}
V(p\wedge b)V^{-1} = VpV^{-1}\wedge VbV^{-1},
\end{equation}
as well as
\begin{equation}\label{equ_versor_identity_dot}
p\cdot b = VpV^{-1}\cdot VbV^{-1}.
\end{equation}
Proofs of these identities may be found in \cite{Parkin12_intro}.
A perhaps less known identity, however, is the following.
\begin{equation}\label{equ_versor_identity}
VpV^{-1}\cdot b = p\cdot V^{-1}bV
\end{equation}
Let us give a proof of it now.  We will proceed by induction.
Letting $v\in\W$ be a vector, it is easy to see that
\begin{equation}\label{equ_induction_start}
vpv^{-1}\cdot b = \frac{2(v\cdot p)(v\cdot b)}{v^2} - p\cdot b = p\cdot v^{-1}bv.
\end{equation}
Assuming now that the identity \eqref{equ_versor_identity} holds for a versor
composed as the geometric product of some fixed number of vectors, the proof of identity
\eqref{equ_versor_identity} follows by induction with
\begin{align}
& vVp(vV)^{-1}\cdot b \\
=\;& vVpV^{-1}v^{-1}\cdot b \\
=\;& VpV^{-1}\cdot v^{-1}bv & \mbox{by equation \eqref{equ_induction_start},} \\
=\;& p\cdot V^{-1}v^{-1}bvV & \mbox{by our inductive hypothesis,} \\
=\;& p\cdot (vV)^{-1}bvV.
\end{align}

In the next section we'll make use of identity \eqref{equ_versor_identity} as well as
\eqref{equ_versor_identity_wedge} to prove the main result.  The identity
\eqref{equ_versor_identity_dot} has many use cases while working in $\G$
of \cite{Parkin12}, but will not be needed to prove the main result.
Looking back, however, it is not hard to see that \eqref{equ_versor_identity_dot}
implies \eqref{equ_versor_identity} in an easier proof than what has just been given.

\section{Relating The Action Of Versors On Quadrics To That Of Points}

It was established in \cite{Parkin12} that quadrics $E\in\G$ are bivectors
of the form
\begin{equation}\label{equ_quadric_form}
E = \sum_{i=1}^k a_i\wedge\overline{b_i}
\end{equation}
where for each integer $i\in[1,k]$, each of $a_i$ and $b_i$ are
taken from $\V$.  Being a quadric, the set of all projective points
$p\in\V$ on $E$ is given by the set of all projective points $p\in\V$
such that
\begin{equation}\label{equ_quadric_representation}
0 = p\wedge\overline{p}\cdot E.
\end{equation}
Clearly now, if we can visualize the quadric $E$, and if we can understand the action
of a versor $V$ on a projective point $p$, then our imaginations are likely able
to visualize the geometry that is the set of all projective points $p\in\V$ such that
\begin{equation}\label{equ_quadric_transformed}
0 = VpV^{-1}\wedge\overline{VpV^{-1}}\cdot E.
\end{equation}
For example, if $V$ translates $p$ by a direction vector $t$, then $E$ must be
translated by the direction vector $-t$.  Similarly, if $V$ rotates $p$ on
an axis $a$ by an angle $\theta$, then $E$ must be rotated by an
angle $-\theta$ about the axis $a$.  Of course, no
claim is being made here that either of such versor exists.  (A translation
versor is not known to exist, but it has been shown in \cite{Parkin12} that the
rotation versor does exist.)  The idea, however,
that the action of $V$ on $p$ translates into the inverse
action of $V$ on $E$, should be well understood.

We will now proceed to show that the geometry represented in
equation \eqref{equ_quadric_transformed} is the very geometry
represented by the quadric $(V\overline{V})^{-1}EV\overline{V}$,
provided that $V$ has the property that for all vectors $v\in\V$,
we have
\begin{equation}\label{equ_versor_invariant_property}
\begin{array}{cl}
v=\overline{V}v\overline{V^{-1},} \\
\overline{v}=V\overline{v}V^{-1}, & \mbox{(which follows from $v=\overline{V}v\overline{V^{-1}}$),} \\
\end{array}
\end{equation}
as well as
\begin{equation}\label{equ_versor_closure_property}
\begin{array}{cl}
VvV^{-1}\in\V, \\
\overline{V}\overline{v}\overline{V^{-1}}\in\overline{\V}, & \mbox{(which follows from $VvV^{-1}\in\V$),}
\end{array}
\end{equation}
which is to say that $V$ leaves vectors in $\overline{\V}$ invariant under versor
conjugation as $\overline{V}$ leaves vectors in $\V$ invariant under versor conjugation,
as well as that conjugation of a vector in $\V$ by the versor $V$ is an operation closed in $\V$.
It will then immediately follow
that if we understand the action of $V^{-1}$ on a projective point $p$, then
we understand the action of $V$ on $E$ as
\begin{equation}
V\overline{V}E(V\overline{V})^{-1}.
\end{equation}

The proof is straight forward as it follows from the equality of \eqref{equ_start} with \eqref{equ_finish}.
{\allowdisplaybreaks
\begin{align}
 & VpV^{-1}\wedge\overline{VpV^{-1}}\cdot E\label{equ_start} \\
=\;& \sum_{i=1}^k VpV^{-1}\wedge\overline{VpV^{1}}\cdot a_i\wedge\overline{b_i}\label{equ_second} \\
=\;& -\sum_{i=1}^k (VpV^{-1}\cdot a_i)(\overline{VpV^{-1}}\cdot\overline{b_i}) & \mbox{by property \eqref{equ_versor_closure_property},} \\
=\;& -\sum_{i=1}^k (p\cdot V^{-1}a_iV)(\overline{p}\cdot\overline{V^{-1}b_iV}) & \mbox{by identity \eqref{equ_versor_identity},} \\
=\;& \sum_{i=1}^k p\wedge\overline{p}\cdot V^{-1}a_iV\wedge\overline{V^{-1}b_iV} &\mbox{by property \eqref{equ_versor_closure_property},} \\
=\;& \sum_{i=1}^k p\wedge\overline{p}\cdot (V\overline{V})^{-1}a_iV\overline{V}\wedge
(V\overline{V})^{-1}\overline{b_i}V\overline{V} &\mbox{by property \eqref{equ_versor_invariant_property},}\label{equ_third_to_last} \\
=\;& \sum_{i=1}^k p\wedge\overline{p}\cdot (V\overline{V})^{-1}(a_i\wedge\overline{b_i})V\overline{V} & \mbox{by identity \eqref{equ_versor_identity_wedge},} \\
=\;& p\wedge\overline{p}\cdot (V\overline{V})^{-1}EV\overline{V}.\label{equ_finish}
\end{align}}

Of course, this is just one of perhaps many algebraic routes one could take to prove the
identity that \eqref{equ_start} is \eqref{equ_finish}.  In fact, it is not hard to see that
a shorter route can be found from \eqref{equ_second} to \eqref{equ_third_to_last} using only
\eqref{equ_versor_identity_dot} and \eqref{equ_versor_invariant_property}.  Nevertheless,
the route shown above illustrates algebraic techniques that are useful as their need is frequently encountered.

The property \eqref{equ_versor_invariant_property} is not unreasonable at all
since a versor providing any action on a projective point $p\in\V$ must come from $\G(\V)$ anyway,
and by so doing, naturally leaves vectors in $\overline{\V}$ untouched, up to scale.
In fact, the condition of \eqref{equ_versor_invariant_property} may be relaxed to allow
for a sign change, as such a change leaves the geometry represented by a bivector invariant.

\section{Concluding Remarks}

The main result now given, we have a tool that we can use to discover or derive the
action of a versor upon an element of the form \eqref{equ_quadric_form} representative of some piece of geometry
by equation \eqref{equ_quadric_representation}.  As noted earlier, \cite{Parkin12} shows that such elements may
be representative of quadric surfaces; but that there appears to be a limited number of
versors in $\G$ for desired transformations suggests, among other evidences, that there must
be a better model based upon geometric algebra, or perhaps some other algebra, that is capable of both representing the
quadrics and offering a wider array of operations we  can perform on such geometries.

Where we might start looking for a better model is in a re-examination of how we're using
vectors to represent points.\footnote{Note that the way spoken of here by which such vectors are
representative of points is different than how elements of the form \eqref{equ_quadric_form}
represent quadrics by the definition \eqref{equ_quadric_representation}.  One must not
confuse the two interpretations.  Note that there are elements of the form \eqref{equ_quadric_form}
representative of points, but, unlike the conformal model, these are not used in definition
\eqref{equ_quadric_representation}.  That the points of the conformal model appear
in the definition of any conformal geometry, (which includes conformal points themselves), is one of many
features contributing to how interesting and pleasing the results of the conformal model are.}
Experiences to date suggest that there is not much of interest we can do with versors acting
upon vectors representing points in classical homogenous space.  That is, a projective point $p\in\V$
of the form
\begin{equation}\label{equ_proj_point}
p = \lambda(e_0 + \vec{p}),
\end{equation}
where $\lambda\in\R$ is a scalar, and $\vec{p}\in\V$ is a non-projective point or direction vector.  Reflections and therefore rotations
about the origin are one immediately observable type of transformation of interest
that we can perform on such points of the form \eqref{equ_proj_point} using a versor.
If there is a versor that reflects such points about
an arbitrary plane, which would be highly desirable as it would lead to versors
representing the rigid body motions, it is not obvious.

We'll therefore close with the following question.
Assuming that a better model for quadric surfaces still makes use of the form \eqref{equ_quadric_form} and
the definition \eqref{equ_quadric_representation} in some geometric algebra other than $\G$
in \cite{Parkin12}, how might we maintain the ability of this form and definition, collectively,
to allow for the representation of all quadrics while altering the form \eqref{equ_proj_point},
if possible, to lend itself to a wider variety of transformations by versor conjugation?  As we've
shown in the main result, if a better form than \eqref{equ_proj_point} can be found to
allow for more transformations, then
the quadrics of the form \eqref{equ_quadric_form} will become endowed with the ability
to be transformed by this larger set of transformations.

\bibliographystyle{amsplain}
\bibliography{Parkin_ATransformationalResultOfTheQuadricModel}

\end{document}