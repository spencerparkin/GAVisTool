\documentclass{article}

\usepackage{amsmath}
\usepackage{amssymb}
\usepackage{amsthm}

\addtolength{\oddsidemargin}{-.575in}
\addtolength{\evensidemargin}{-.575in}
\addtolength{\textwidth}{1.0in}
\addtolength{\topmargin}{-.575in}
\addtolength{\textheight}{1.25in}

\newcommand{\R}{\mathbb{R}}
\newcommand{\V}{\mathbb{V}}
\newcommand{\G}{\mathbb{G}}
\newcommand{\prl}{\parallel}
\newcommand{\prp}{\perp}
\newcommand{\nlo}{o}
\newcommand{\nli}{\infty}

%\swapnumbers
\newtheorem{theorem}{Theorem}[section]
\newtheorem{definition}{Definition}[section]
\newtheorem{corollary}{Corollary}[section]
\newtheorem{identity}{Identity}[section]

\title{A Compilation Of Fun GA Problems}
%\author{Spencer T. Parkin}

\begin{document}
\maketitle

\section{The First Set Of Problems}

For this first set of problems, assume that all elements
come from a \emph{Euclidean} geometric algebra.

\subsection{The Inverse Of A Vector}

Given a non-zero vector $v$, determine its inverse $v^{-1}$ with respect to the geometric product.
Show that this inverse is unique.  Does $v$ have an inverse with respect to the inner or outer products?
Do all vectors have an inverse with respect to the geometric product?

\subsection{Reflecting A Vector About A Vector}

Given a unit-length vector $n$ and a vector $v$, find the vector $v'$ which is the
orthogonal reflection of $v$ about $n$.

\subsection{Reflecting A Vector About A Blade}

Given a 2-blade $B$ and a vector $v$, find the vector $v'$ which is the orthogonal
reflection of $v$ about $B$.

\subsection{The Angle Between Blades}

Given two blades $A$ and $B$, each of the same grade $k\geq 1$, find a formula for the product $A\cdot B$
in terms of the angle $\theta$ made between the planes containing each of these blades.

\subsection{The Magnitude Of The Outer Product Of Two Blades}

Given two blades $A$ and $B$, each of the same grade $k\geq 1$, find a formula for the quantity $|A\wedge B|$
in terms of the angle $\theta$ made between the planes containing each of these blades.

\subsection{The Inner Product Of A Vector And A Blade}

Given a vector $v$ and a $k$-blade $B$, which may be written in terms
of the $k$ vectors in $\{b_i\}_{i=1}^k$ as
\begin{equation}
B = \bigwedge_{i=1}^k b_k,
\end{equation}
we may define the inner product of $v$ and $B$ as
\begin{equation}
v\cdot B = -\sum_{i=1}^k (v\cdot b_i)B_i,
\end{equation}
where we define $B_i$ as the $(i-1)$-blade given as
\begin{equation}
B_i = \bigwedge_{\substack{j=1\\j\neq i}} b_j.
\end{equation}
Given this definition of $v\cdot B$, show that
\begin{equation}
v\cdot B = (v\cdot a_k)B_1 - a_k\cdot(v\wedge B_1).
\end{equation}

\subsection{Versors}

A versor is any geometric product of a finite sequence $\{v_i\}_{i=1}^k$ of invertible vectors.
That is, if $V$ is a versor, then we may write
\begin{equation}
V = a_1a_2a_3\dots a_k = \prod_{i=1}^k a_i
\end{equation}
The reverse of $V$, written as $\tilde{V}$, is defined as
\begin{equation}
\tilde{V} = a_ka_{k-1}a_{k-2}\dots a_1 = \prod_{i=1}^k a_{k-i+1}.
\end{equation}
Given all of this information, determine the inverse $V^{-1}$ of $V$.

\subsection{Converting Between Versors And Blades}\label{prob_blade_versor_conversion}

Show that any versor can be rewritten as a blade.  Then show that any blade can be rewritten as a versor.

\subsection{The Inverse Of a Blade}

Given a blade $B$, determine when $B^{-1}$ exists, given a formula for $B^{-1}$ in terms of $B$,
and show that if $B^{-1}$ exists, then it is unique.

\subsection{The Conjugation Of Vectors By Versors}\label{prob_vector_versor_conjugation}

Given any vector $v$ and a versor $V$, show that $VvV^{-1}$ is also a vector.

\subsection{The Conjugation Of Blades By Versors}\label{prob_blade_versor_conjugation}

Given any $k$-blade $B$ and a versor $V$, show that
\begin{equation}
VBV^{-1} = \bigwedge_{i=1}^k Vb_kV^{-1},
\end{equation}
in the case that $B$ may be factored in terms of the vectors in $\{b_i\}_{i=1}^k$ as
\begin{equation}
B = \bigwedge_{i=1}^k b_k.
\end{equation}
Also show that $VBV^{-1}$ has the same grade as $B$.

\subsection{Versors And The Inner Product}\label{prob_versors_and_inner_prod}

Given any two vectors $a$ and $B$, and a versor $V$, show that
\begin{equation}
a\cdot b = (VaV^{-1})\cdot(VbV^{-1}),
\end{equation}
and that
\begin{equation}
(V^{-1}aV)\cdot b = a\cdot(VbV^{-1}).
\end{equation}
(Hint: One of these follows trivially from the other.)

\section{The Second Set Of Problems}

In this second set of problems, assume that all elements
come from a \emph{non-Euclidean} geometric algebra.

\subsection{Null Vectors}

In a non-Euclidean geometric algebra, we allow vectors $v$ having an inner product
square $v\cdot v$ of zero.  How does this impact the invertibility of non-zero vectors?

\subsection{Converting Between Versors And Blades}

Re-examine Problem~\ref{prob_blade_versor_conversion}.  Does your
proof work in a non-Euclidean geometric algebra?!  What can you conclude?

\subsection{Conjugation By Versors}

Re-examine Problem~\ref{prob_vector_versor_conjugation},
Problem~\ref{prob_blade_versor_conjugation} and Problem~\ref{prob_versors_and_inner_prod}.
Do your proofs for these problems still
go through in a non-Euclidean geometric algebra?  If not, can an alternative
proof by found that does work in a non-Euclidean geometric algebra, or if so,
could you have found an easier proof that takes advantage of properties
available in a Euclidean geometric algebra?

\end{document}