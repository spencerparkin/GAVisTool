\documentclass[12pt]{article}

\usepackage{amsmath}
\usepackage{amssymb}
\usepackage{amsthm}

\title{Geometric Calculus}
\author{Spencer T. Parkin}

\newcommand{\G}{\mathbb{G}}
\newcommand{\V}{\mathbb{V}}
\newcommand{\R}{\mathbb{R}}
\newcommand{\A}{\mathbb{A}}
\newcommand{\B}{\mathbb{B}}

\newtheorem{theorem}{Theorem}[section]
\newtheorem{definition}{Definition}[section]
\newtheorem{corollary}{Corollary}[section]
\newtheorem{identity}{Identity}[section]
\newtheorem{lemma}{Lemma}[section]
\newtheorem{result}{Result}[section]

\begin{document}
\maketitle

\section{The Limit}

Geometric calculus is the development of a calculus based on the idea of the limit
using geometric algebra.  A knowledge of geometric algebra is assumed here, so
let us start with the definition of the limit.
\begin{definition}\label{def_limit}
Given a real valued function $f$ of a real variable, we say that the limit
of $f$ as $x$ approaches $\lambda$ exists and is a real number $L$, which
may be expressed as
\begin{equation*}
\lim_{x\to\lambda}f(x) = L,
\end{equation*}
if for every real number $\epsilon>0$, there exists a real number $\delta>0$
such that if $|x-\lambda|<\delta$, then $|f(x)-L|<\epsilon$.
\end{definition}
This definition concerns itself with only the real number line, but as we know, geometric
algebra developes the theory of Euclidean spaces to arbitrary dimensions.  Therefore,
as we go on to develop geometric calculus, variations and expansions of definition $\eqref{def_limit}$
will be required.  For example, we may need to generalize the metric being used in
definition $\eqref{def_limit}$ to work in higher dimensional spaces, because we'll
want to study the limits of functions defined on such spaces.

Also central to the study of geometric calculus, therefore, is the study of functions.
Of particular interest to us will be a class of functions known as linear functions,
the study of which is known as linear algebra, and therefore a sub-topic of geometric calculus.
Linear functions are defined on linear spaces, most often referred to as vector spaces,
suggesting to the mind a geometric interpretation of the elements of such spaces.
Indeed, vector spaces play a fundamental role in the development of geometric algebra.

Before jumping into limit processes, it makes sense for us to first treat the subject
of linear algebra as we'll find many examples of linear functions in the broader
subject of geometric calculus.

\section{Linear Algebra}

We begin with a formal definition of a linear function.
\begin{definition}
Let $\A$ and $\B$ denote any two vector spaces, not necessariliy unique, arbitrarily defined
over the field of real numbers $\R$.  Then a
linear function $f$ is a mapping from $\A$ to $\B$ that preserves both scalar-vector multiplication
and vector addition.  That is, for any scalar $\lambda\in\R$ and any two vectors $x,y\in\A$,
we have $f(\lambda x)=\lambda f(x)$ and $f(x+y)=f(x)+f(y)$.
\end{definition}
Given such a definition, it is natural to begin by seeing how much we can learn about
such functions without looking at any one particular example.  Recall that if $\A$ is
an $m$-dimensional vector space, then there exists at most a set of $m$ linearly
independent vectors $\{a_i\}_{k=1}^m\subset\A$ that form a basis for $\A$.
Letting $x = \sum_{i=1}^m x_ia_i$, where each $x_i\in\R$, we see that
\begin{equation*}
f(x) = \sum_{i=1}^m x_i f(a_i),
\end{equation*}
showing that $f$ is entirely determined by how it transforms any set of basis
vectors for $\A$.  This has immediate implications for the existance of $f^{-1}$
when we recall the definition of a non-invertible function.
\begin{definition}
A function $f:\A\to\B$ is non-invertible if for any given element $b\in f(\A)$,
there exists at least two unique elements $x,y\in\A$ such that $f(x)=b$ and $f(y)=b$.
\end{definition}
This definition illustrates the non-invertibility of $f$ by the ambiguity we face in the construction
of $f^{-1}$ when we consider how to map $f^{-1}(b)$.  Does this map to $x$ or $y$?

Consider now the set $\{f(a_i)\}_{i=1}^m\subset\B$ and suppose for the moment that
it is linearly dependent.  Then, without loss of generality, we can write $f(a_m)$
as $f(a_m)=\sum_{i=1}^{m-1}\lambda_i f(a_i)$, where each $\lambda_i\in\R$.
Now let $x_m=0$ and let $y\in\A$ be $y=\sum_{i=1}^m y_i a_i$, where $y_m\neq 0$,
and for all $i<m$, we have $y_i=x_i-\lambda_i y_m$.  Clearly $x\neq y$, and we see that
\begin{align*}
f(x) &= \sum_{i=1}^{m-1} x_if(a_i)
 = \sum_{i=1}^{m-1}(y_i+\lambda_i y_m)f(a_i) \\
 &= \sum_{i=1}^{m-1}y_i f(a_i) + y_m\sum_{i=1}^{m-1}\lambda_i f(a_i)
 = \sum_{i=1}^{m-1}y_i f(a_i) + y_m f(a_m)
 = f(y),
\end{align*}
showing that $f$ is non-invertible.

Considering now the set $\{f(a_i)\}_{i=1}^m\subset\B$ to be linearly independent, we
see that...

 It follows that $f$ is invertible if and only if $f$ preserves linear independence
in the sense that if $\{a_i\}_{i=1}^m$ is a linearly independent set, then so is $\{f(a_i)\}_{i=1}^n$.

Now knowing when $f^{-1}$ exists, how do we find $f^{-1}$?  Interestingly, geometric
algebra gives us a means to answering this question.  We begin with the definition
of an outermorphism.
\begin{definition}
A function $f:\A\to\B$ is an outermorphism if it is linear and perserves the outer product.  That is,
$f$ is a linear function, and for any two vectors $x,y\in\A$, we have $f(x\wedge y)=f(x)\wedge f(y)$.
\end{definition}
Every linear function can be extended to an outermorphism if it does not already possess the
defining characteristics of such a function.  (This may need some proof.)  As an extention of a linear function, an
outermorphism preserves the original linear function so that if we find the inverse of
an outermorphism, then we have also found the inverse of the original linear function.

Taking the zero element of any geometric algebra as being without grade, we see already
that an outermorphism $f$ is invertible if and only if it preserves the grade of blades.  Indeed,
if $f$ is invertible, then it must map $I_{\A}$ to some non-zero scalar multiple of $I_{\B}$,
where each of these denote the unit psuedo-scalars of $\G(\A)$ and $\G(\B)$, respectively.
In fact, this scalar multiple has special significance, and we define it to be the determinant
of the outermorphism.
\begin{equation*}
f(I_{\A}) = (\det f)I_{\B}
\end{equation*}
Clearly $f$ is invertible if and only if $\det f\neq 0$, but we chose this scalar in particular
because of the role it will play in the formulation of the $f^{-1}$.
\begin{definition}
The adjoint of an outermorphism $f$ is an outermorphism $g$ with the property that
for any two vectors $x,y\in\A$, we have $x\cdot f(y)=g(x)\cdot y$.
\end{definition}
How do we know that this exists?

\end{document}