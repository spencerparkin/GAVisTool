\documentclass[12pt]{article}

\usepackage{amsmath}
\usepackage{amssymb}
\usepackage{amsthm}

\title{Geometric Calculus}
\author{Spencer T. Parkin}

\newcommand{\G}{\mathbb{G}}
\newcommand{\V}{\mathbb{V}}
\newcommand{\R}{\mathbb{R}}

\newtheorem{theorem}{Theorem}[section]
\newtheorem{definition}{Definition}[section]
\newtheorem{corollary}{Corollary}[section]
\newtheorem{identity}{Identity}[section]
\newtheorem{lemma}{Lemma}[section]
\newtheorem{result}{Result}[section]

\begin{document}
\maketitle

\section{What Is This Paper?}

This paper reads like an introductory treatment of geometric calculus, and was
written that way intentially, but it is really just a formal compilation of my notes
on the subject.
The idea is that an exposition of this subject helps to solidify one's understanding
of it, and helps one organize his thoughts on it.  Often a good test of one's understanding
of a subject is one's ability to expound it.  And I like to write.  And I'm a nerd.
Let's move on.

\section{The Limit}

Geometric calculus is the development of a calculus based on the idea of the limit
using geometric algebra.  A knowledge of geometric algebra is assumed here, so
let us start with the definition of the limit.
\begin{definition}\label{def_limit}
Given a real valued function $f$ of a real variable, we say that the limit
of $f$ as $x$ approaches $\lambda$ exists and is a real number $L$, which
may be expressed as
\begin{equation*}
\lim_{x\to\lambda}f(x) = L,
\end{equation*}
if for every real number $\epsilon>0$, there exists a real number $\delta>0$
such that if $|x-\lambda|<\delta$, then $|f(x)-L|<\epsilon$.
\end{definition}
This definition concerns itself with only the real number line, but as we know, geometric
algebra developes the theory of Euclidean spaces to arbitrary dimensions.  Therefore,
as we go on to develop geometric calculus, variations and expansions of definition $\eqref{def_limit}$
will be required.  For example, we may need to generalize the metric being used in
definition $\eqref{def_limit}$ to work in higher dimensional spaces, because we'll
want to study the limits of functions defined on such spaces.

Also central to the study of geometric calculus, therefore, is the study of functions.
Of particular interest to us will be a class of functions known as linear functions,
the study of which is known as linear algebra, and therefore a sub-topic of geometric calculus.
Linear functions are defined on linear spaces, most often referred to as vector spaces,
suggesting to the mind a geometric interpretation of the elements of such spaces.
Indeed, vector spaces play an obvious role in the development of geometric algebra.

Before jumping into limit processes, it makes sense for us to first treat the subject
of linear algebra as we'll find many examples of linear functions in the broader
subject of geometric calculus.

\section{Linear Algebra}

Though this needs some proof, I believe that there will be no loss in generality if
we restrict ourselves here to the study of linear functions mapping to and from the
same vector space.  Doing this will simplify much of the notation to follow.
\begin{definition}
Define linear a function here.
\end{definition}

\end{document}