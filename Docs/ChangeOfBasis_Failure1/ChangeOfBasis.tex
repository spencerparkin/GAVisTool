\documentclass[12pt]{article}

\usepackage{amsmath}
\usepackage{amssymb}
\usepackage{amsthm}

\title{Performing a Change of Basis\\in\\Geometric Algebra}
\author{Spencer T. Parkin}

\newcommand{\G}{\mathbb{G}}
\newcommand{\V}{\mathbb{V}}
\newcommand{\R}{\mathbb{R}}
\newcommand{\A}{\mathbb{A}}
\newcommand{\B}{\mathbb{B}}

\newtheorem{theorem}{Theorem}[section]
\newtheorem{definition}{Definition}[section]
\newtheorem{corollary}{Corollary}[section]
\newtheorem{identity}{Identity}[section]
\newtheorem{lemma}{Lemma}[section]
\newtheorem{result}{Result}[section]

\begin{document}
\maketitle

It is well known that rotations, translations, uniform scalings and other types of transformations
can be performed in geometric algebra using versors.  Here we explore the possibility
of performing shear and non-uniform transformations using general multivectors.  We do this
by developing a means of performing change-of-basis transformations using bivectors.

Let $\V^{2n}$ denote a Euclidean vector space of dimension $2n$.  We will use the
geometric algebrta $\G(\V^{2n})$ generated by this vector space.  Let $f:\A\to\B$ be a linear
function mapping any $n$-dimensional vector sub-space of $\V^{2n}$ to its disjoint
$n$-dimensional vector sub-space with respect to $\V^{2n}$.  (These are disjoint sub-spaces with the exception
of the zero vector.)  Let $\{e_k\}_{k=1}^n$ be any set of orthonormal basis vectors genearting $\A$
and $\{e_k\}_{k=n+1}^{2n}$ be any set of orthonormal basis vectors generating $\B$.  We define
$f$ as a linear transformation from $\A$ to $\B$ as well as an isomorphism between these two identical
vector spaces.  That is, for all integers $k\in[1,n]$, $f(e_k)=e_{k+n}$, and we say that
$f$ preserves the sum of vectors and the scalar-vector product.
(Using a higher dimensional geometric
algebra, it might be possible to define $f$ as a rotation of $\A$ onto $\B$.)

With this geometric algebra in place, we wish to find a multivector representative of the following
change of basis transformation.  Given a vector $v\in\A$, we wish to transform it in terms
of a new basis defined by any set of $n$ linearly independent vectors $\{a_k\}_{k=1}^n$ taken from $\A$.
We may think of $v$ as being currently expressed in terms of the set of vectors $\{e_k\}_{k=1}^n$ as
\begin{equation*}
v = \sum_{k=1}^n (v\cdot e_k)e_k.
\end{equation*}
We wish to find a transformation $T:\A\to\A$ taking $v$ in this form to the new vector
\begin{equation*}
T(v) = \sum_{k=1}^n (v\cdot e_k)a_k.
\end{equation*}
To this end, letting $M$ be the bivector
\begin{equation*}
M = \sum_{k=1}^n a_k\wedge e_{k+n},
\end{equation*}
we see that the desired transformation is given by
\begin{equation*}
T(v) = M\cdot f(v).
\end{equation*}
If $M_a$ and $M_b$ were two such change-of-basis transformations, then the concatination $M$
of these transformations would be given by
\begin{equation*}
M = \sum_{k=1}^n (M_b\cdot f(M_a\cdot e_{k+n}))\wedge e_{k+n}.
\end{equation*}
Admittedly, neither of the latter two formulas above seem appealing.  Representing transformations
with versors affords us all the properties of the geometric product, one of which is invertability.
How do we formulate the inverse transformation of the change-of-basis transformation represented
by the bivector $M$?  Is there anything to be gained by somehow extending $f$ to an outermorphism?

I would also note here that we shouldn't feel bad about refering to a basis.  One of the strengths of
geometric algebra is often the ability to derive coordinate independent formulas.  But the very nature
of the transformation we're trying to provide here is based on choosing a new basis with which
to express a set of coordinates.

\end{document}