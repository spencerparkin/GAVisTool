\documentclass{birkjour}

\usepackage{amsmath}
\usepackage{amssymb}
\usepackage{amsthm}
\usepackage{graphicx}
\usepackage{float}
\usepackage{url}

\newtheorem{thm}{Theorem}[section]
 \newtheorem{cor}[thm]{Corollary}
 \newtheorem{lem}[thm]{Lemma}
 \newtheorem{prop}[thm]{Proposition}
 \theoremstyle{definition}
 \newtheorem{defn}[thm]{Definition}
 \theoremstyle{remark}
 \newtheorem{rem}[thm]{Remark}
 \newtheorem*{ex}{Example}
 \numberwithin{equation}{section}

\newcommand{\G}{\mathbb{G}}
\newcommand{\V}{\mathbb{V}}
\newcommand{\Vb}{\mathbb{\overline{V}}}
\newcommand{\W}{\mathbb{W}}
\newcommand{\R}{\mathbb{R}}
\newcommand{\B}{\mathbb{B}}

\newcommand{\Alpha}{A}
%\Omega is already defined

\newcommand{\nvao}{o}
\newcommand{\nvai}{\infty}
\newcommand{\nvaob}{\overline{o}}
\newcommand{\nvaib}{\overline{\infty}}

\newcommand{\eminus}{e_{-}}
\newcommand{\eplus}{e_{+}}
\newcommand{\eminusb}{\overline{e}_{-}}
\newcommand{\eplusb}{\overline{e}_{+}}

\begin{document}

\title{A Variation Of The Quadric Model\\Of Geometric Algebra}

\author{Spencer T. Parkin}
\email{spencer.parkin@gmail.com}

\numberwithin{equation}{section}

\subjclass{Primary 14J70; Secondary 14J29}

\keywords{Quadric Surface, Quartic Surface, Geometric Algebra, Quadric Model, Conformal Model}

%\dedicatory{To Melinda and Naomi}

\begin{abstract}
A variation of the quadric model set forth in \cite{Parkin12} is
found in which the rigid body motions are represented by
versors applicable to any quadric surface.
Extending this variation of the original model to include
a specific form of quartic surface, we find that these surfaces are
closed under the application of all conformal transformations.
Results of a computer program implementing this new model are presented.
\end{abstract}

\maketitle

\section{Introduction}

In the original paper \cite{Parkin12}, a model for quadric surfaces was
presented based upon the ideas of projective geometry.  What was unfortunate
about this model, however, was its lack of support for the rigid body transformations.  It was
predicted in the conclusion of that paper that a better model for quadric
surfaces may exist that is more like the conformal model of geometric algebra.
The present paper details what may be such a model.  We'll find that the rigid
body transformations can be incorporated into the model by using an alternative
method of encoding the quadric form.  An extension to this quadric form
will then allow us to support the conformal transformations at the expense of
expanding our model to necessarily include a specific form of quartic surfaces.  The
new model and its extension will both use the same geometric algebra to be given as follows.

\section{The Geometric Algebra}

We begin here with a description of the structure of the geometric algebra upon
which our model will be imposed.  This
geometric algebra will contain the following vector spaces.
\begin{equation}\label{equ_vector_spaces}
\begin{array}{ll}
\mbox{Notation} & \mbox{Basis} \\
\hline
\V^e & \{e_i\}_{i=1}^n \\
\V^{\nvao} & \{\nvao\}\cup\{e_i\}_{i=1}^n \\
\V^{\nvai} & \{e_i\}_{i=1}^n\cup\{\nvai\} \\
\V & \{\nvao\}\cup\{e_i\}_{i=1}^n\cup\{\nvai\}
\end{array}
\end{equation}
The set of vectors $\{e_i\}_{i=1}^n$ forms an orthonormal set of basis
vectors for the $n$-dimensional Euclidean vector space $\V^e$, which we'll
use to represent $n$-dimensional Euclidean space.
The vectors $\nvao$ and $\nvai$ are the familiar null-vectors representing the
points at origin and infinity, respectively, taken from the conformal model of geometric algebra.
An inner-product table for these basis vectors is given as follows, where
$1\leq i<j\leq n$.
\begin{equation}
\begin{array}{c|cccc}
\cdot & \nvao & e_i & e_j & \nvai \\
\hline
\nvao & 0 & 0 & 0 & -1 \\
e_i & 0 & 1 & 0 & 0 \\
e_j & 0 & 0 & 1 & 0 \\
\nvai & -1 & 0 & 0 & 0
\end{array}
\end{equation}
We will now let $\G(\V)$ denote the Minkowski geometric algebra generated by $\V$.
For each vector space in table \eqref{equ_vector_spaces}, we will let an over-bar
above this vector space denote an identical copy of that vector space.  The vector
space $\W$ will denote the smallest vector space containing each of $\V$ and $\Vb$
as vector subspaces.  In symbols, one may write
\begin{equation}
\G(\W) = \G(\V\oplus\Vb)
\end{equation}
to illustrate the structure of $\G(\W)$ in terms of its two isomorphic Minkowski
geometric sub-algebras $\G(\V)$ and $\G(\Vb)$.

We will use over-bar notation to distinguish between vectors taken from $\V$
with vectors taken from $\Vb$.  Then, so that we may use the over-bar notation,
in a well defined manner, to distinguish between elements of $\G(\V)$ and $\G(\Vb)$,
we will define it as an outermorphic function that is also an isomorphism between
$\G(\V)$ and $\G(\Vb)$.  Doing so, we see that for any element $E\in\G(\V)$,
we may define $\overline{E}\in\G(\Vb)$ as
\begin{equation}
\overline{E} = SES^{-1},
\end{equation}
where $S$ is the versor given by
\begin{equation}\label{equ_isomorphism_versor}
S = (1+\eminus\eminusb)(1-\eplus\eplusb)\prod_{i=1}^n(1-e_i\overline{e}_i).
\end{equation}
This definition is non-circular if we let the over-bars in equation \eqref{equ_isomorphism_versor}
be purely notation.  The vectors $\eminus$ and $\eplus$, taken from \cite{LiRockwood},
are defined as
\begin{align}
\eminus &= \frac{1}{2}\nvai + \nvao, \\
\eplus &= \frac{1}{2}\nvai - \nvao.
\end{align}
The vectors $\eminusb$ and $\eplusb$ are defined similarly in terms of $\nvaob$ and $\nvaib$.
Defined this way, realize that, like the over-bar function defined in \cite{Parkin12},
here we have the property that for any vector $v\in\V$, we have $\overline{\overline{v}}=-v$.

\section{The Form Of Quadric Surfaces In $\G(\W)$}

We now give a formal definition under which elements $E\in\G(\W)$
are representative of $n$-dimensional quadric surfaces in our present variation
of the original model.
\begin{defn}\label{def_quadric}
Referring to an element $E\in\G(\W)$ as a quadric surface, it is representative of such an $n$-dimensional surface as the set of all points $p\in\V^o$ such that
\begin{equation}\label{equ_quadric_equation}
0 = p\wedge\overline{p}\cdot E.
\end{equation}
\end{defn}
From this definition it can be seen that the general form of a quadric $E\in\G(\W)$ is given by
\begin{equation}\label{equ_quadric_form}
E = \sum_{i=1}^k a_i\overline{b}_i,
\end{equation}
where each of $\{a_i\}_{i=1}^k$ and $\{b_i\}_{i=1}^k$ is a sequence of $k$ vectors
taken from $\V^\nvai$.  To see why, realize that the form \eqref{equ_quadric_form} can
always be reduced to the form
\begin{equation}\label{equ_quadric_reduced_form}
E \equiv \sum_{i=1}^n\sum_{j=i}^n\lambda_{ij}e_i\overline{e}_j+
\sum_{i=1}^n\lambda_i e_i\nvaib+
\lambda\nvai\nvaib,
\end{equation}
where each of $\lambda_{ij}$, $\lambda_i$, and $\lambda$ are scalars, in the
sense that this reduced form represents the same surface as that in equation \eqref{equ_quadric_form}
under Definition~\ref{def_quadric}.
We then see that this form \eqref{equ_quadric_reduced_form}, when it is
substituted into equation \eqref{equ_quadric_equation}, produces to a polynomial
equation of degree 2 in the vector components of $p-\nvao$.
Doing so with $p=\nvao+x$, where $x\in\V^e$, we get the equation
\begin{equation}\label{equ_quadric_polynomial_equation}
0 = -\sum_{i=1}^n\sum_{j=i}^n\lambda_{ij}(x\cdot e_i)(x\cdot e_j)
+\sum_{i=1}^n\lambda_i(x\cdot e_i) - \lambda,
\end{equation}
which we may recognize as the general equation of an $n$-dimensional quadric surface.

In practice, a computer program might take such a bivector of the form \eqref{equ_quadric_form}
and extract from it the coefficients of the quadric
polynomial \eqref{equ_quadric_polynomial_equation} it
represents.  It could then render the surface using traditional methods, such as those used
to render the traced surfaces in Figure~\ref{fig_invert_cylinder_in_sphere} far below.

Of course, using geometric algebra on paper, it might be undesirable and unnecessary to think of
quadrics in terms of polynomial equations.  A, perhaps, better way to think of quadrics is in terms
of an element, of a geometric algebra, whose decomposition
produces the parameters characterizing the quadric surface.  For example, many common
quadrics are the solution set in $\V^e$ of the equation
\begin{equation}
0 = -r^2 + (x-c)^2 + \lambda((x-c)\cdot v)^2,
\end{equation}
in the variable $x$.  (An explanation of the parameters $r$, $c$, $v$ and $\lambda$
was given in \cite{Parkin12}.)  Then, factoring out $-p\overline{p}$, we see that
the element $E\in\G(\W)$, given by
\begin{equation}\label{equ_canonical_form_of_common_quadric}
\Omega + \lambda v\overline{v}+2(c+\lambda(c\cdot v)v)\nvaib+
(c^2+\lambda (c\cdot v)^2-r^2)\nvai\nvaib
\end{equation}
is representative of this very same quadric by Definition~\ref{def_quadric},
where $\Omega$ is defined as
\begin{equation}
\Omega = \sum_{i=1}^n e_i\overline{e}_i.
\end{equation}
Canonical forms similar to \eqref{equ_canonical_form_of_common_quadric}
can be found for specific geometries, such as planes, spheres, plane-pairs,
circular cylinders, circular conical surfaces, and so on.

\section{Transformations Supported By The Model}

The main result of this section will depend upon the following lemma.
\begin{lem}\label{lma_versor_transfer}
For any versor $V\in\G(\W)$, and any four vectors $a,b,c,d\in\V$, we have
\begin{equation}
V^{-1}aV\wedge\overline{V^{-1}bV}\cdot c\wedge\overline{d} =
a\wedge\overline{b}\cdot V\overline{V}(c\wedge\overline{d})(V\overline{V})^{-1}.
\end{equation}
\end{lem}
\begin{proof}
We begin by first establishing that
\begin{align}
 & V^{-1}aV\wedge\overline{V^{-1}bV}\cdot c\wedge\overline{d} \\
=\;& -(V^{-1}aV\cdot c)(V^{-1}bV\cdot d) \\
=\;& -(a\cdot VcV^{-1})(b\cdot VdV^{-1}) \\
=\;& a\wedge\overline{b}\cdot VcV^{-1}\wedge\overline{VdV^{-1}}.
\end{align}
We now notice that
\begin{align}
& VcV^{-1} \\
=\;& V\overline{VV^{-1}}cV^{-1} \\
=\;& (-1)^m V\overline{V}c\overline{V^{-1}}V^{-1} \\
=\;& (-1)^m V\overline{V}c(V\overline{V})^{-1},
\end{align}
where $m$ is the number of vectors taken together in a geometric
product to form $V$.  We then notice that
\begin{align}
& \overline{VdV^{-1}} \\
=\;& VV^{-1}\overline{VdV^{-1}} \\
=\;& (-1)^{m^2}V\overline{V}V^{-1}\overline{dV^{-1}} \\
=\;&(-1)^{m^2+m}V\overline{Vd}V^{-1}\overline{V^{-1}} \\
=\;&(-1)^{2m^2+m}V\overline{VdV^{-1}}V^{-1} \\
=\;&(-1)^mV\overline{Vd}(V\overline{V})^{-1}.
\end{align}
It now follows that
\begin{equation}
a\wedge\overline{b}\cdot VcV^{-1}\wedge\overline{VdV^{-1}} =
a\wedge\overline{b}\cdot V\overline{V}(c\wedge\overline{d})(V\overline{V})^{-1}.
\end{equation}
\end{proof}
We're now ready to prove the main result as follows.
\begin{thm}\label{thm_quadric_transform}
Letting $E\in\G(\W)$ be a bivector of the form \eqref{equ_quadric_form},
$p,p'\in\V^o$ be a pair of points related by a versor $V\in\G(\V)$ by
the equation
\begin{equation}\label{equ_get_rid_ni}
p' = \nvao\cdot V^{-1}pV\wedge\nvai,
\end{equation}
and $E'\in\G(\W)$ a bivector given by
\begin{equation}\label{equ_transformed_surface}
E' = V\overline{V}E(V\overline{V})^{-1},
\end{equation}
the set of all points $p\in\V^\nvao$ such that
\begin{equation}\label{equ_wanted_variety}
0 = p'\wedge\overline{p}'\cdot E
\end{equation}
is exactly the set of all points $p\in\V^\nvao$ such that
\begin{equation}\label{equ_derived_variety}
0 = p\wedge\overline{p}\cdot E'.
\end{equation}
\end{thm}
\begin{proof}
The theorem goes through by the following chain of equalities.
\begin{align}
 & (\nvao\cdot V^{-1}pV\wedge\nvai)\wedge\overline{(\nvao\cdot V^{-1}pV\wedge\nvai)}\cdot E \\
=\;& V^{-1}pV\wedge\overline{V^{-1}pV}\cdot E \\
=\;& p\wedge\overline{p}\cdot(V\overline{V})E(V\overline{V})^{-1}.
\end{align}
The first equality holds by the fact that $E$ is of the form \eqref{equ_quadric_form},
while the second equality holds by Lemma~\ref{lma_versor_transfer}.
\end{proof}

The key motivation behind Theorem~\ref{thm_quadric_transform} is
the observation that the desired transformation of $E$ by $V$ is
given by the algebraic variety of equation \eqref{equ_wanted_variety}, because
an understanding of how $V^{-1}$ transforms $p$ gives us an understanding
of what type of geometry we get from equation \eqref{equ_wanted_variety} in terms of $E$ and $V$.
The theorem then
shows that this is also the algebraic variety of equation \eqref{equ_derived_variety}, thereby
giving us a means of performing desired transformations on elements representative
of quadric surfaces in $\G(\W)$, provided that $V$ is such a versor that $E'$ in
\eqref{equ_transformed_surface}, like $E$, is also bivector of the form \eqref{equ_quadric_form}.
If this is not the case, then $V\overline{V}$ simply represents a transformation not
closed in the set of all quadric surfaces.

We can now apply Theorem~\ref{thm_quadric_transform} to show
that the rigid body transformations are supported in our new variation
of the original model.
Letting $\pi\in\V$ be a dual plane of the conformal model, given by
\begin{equation}
\pi = v+(c\cdot v)\nvai,
\end{equation}
where $v\in\V^e$ is a unit-length vector indicating the norm of the plane,
and where $c\in\V^e$ is a vector representing a point on the plane,
we see that for any homogenized point $p\in\V^o$, we have
\begin{equation}
-\pi p\pi^{-1} = \nvao+x-2((x-c)\cdot v)v + \lambda\nvai,
\end{equation}
where $p=\nvao+x$ with $x\in\V^e$, and
where the scalar $\lambda\in\R$ is of no consequence.  Letting $V=\pi$,
the point $p'\in\V^o$ of consequence here is given by equation \eqref{equ_get_rid_ni},
from which we can recognize an orthogonal reflection about the plane $\pi$.
It now follows by Theorem~\ref{thm_quadric_transform} that $\pi\overline{\pi}$ is a versor
capable of reflecting any quadric surface about the plane $\pi$.
Being able to perform planar reflections of any quadric in any plane, it
now follows that we can always find a versor $V\in\G(\W)$ capable of performing
any rigid body motion applicable to any quadric surface.  The development
of the rigid body motions, (combinations of translations and rotations), by planar reflections,
is well known, and can be found in section 2.7 of \cite{LiRockwood}.

Notice that not all versors of the conformal model are applicable in
our variation of the original quadric model.  This is because they fail to
satisfy the condition for closure that $E'$ be
of the form \eqref{equ_quadric_form}.

In retrospect, what we have done to find the rigid body motions
of quadric surfaces is similar to what was done in \cite{Langer08}; and
according to \cite{Pfister95}, we can state more generally that what we
have done is at least similar to finding an isomorphism between quadratic spaces.

\section{Extending The New Model}

\begin{figure}
\includegraphics[scale=0.4]{InvertCylinderInSphere}
\caption{The inversion of a cylinder in a sphere.  Traces in various planes were
used to render the cylinder and its inversion.}
\label{fig_invert_cylinder_in_sphere}
\end{figure}
Interestingly, if we were not content with the rigid body motions of
quadrics, then we really could find what is, for example, the spherical
inversion of, say, an infinitely long cylinder in a sphere.  To do this, we change
Definition~\ref{def_quadric} into the following definition.
\begin{defn}\label{def_surface}
For any element $E\in\G(\W)$, we may refer to it as an $n$-dimensional
quartic surface as the set of all points $p\in\V^e$ such that
\begin{equation}\label{equ_surface_variety}
0 = P(p)\wedge\overline{P}(p)\cdot E,
\end{equation}
where $P:\V^e\to\V$ is the conformal mapping, defined as
\begin{equation}
P(p) = \nvao + p + \frac{1}{2}p^2\nvai.
\end{equation}
\end{defn}
A version of Theorem~\ref{thm_quadric_transform} is then easily found
such that if $V\in\G(\W)$ is any versor of the conformal model, and if $E$
is a surface under Definition~\ref{def_surface}, then the element $E'\in\G(\W)$,
given by equation \eqref{equ_transformed_surface}, must, by Definition~\ref{def_surface},
 be representative of the desired transformation of $E$ by the versor $V$.  The general
polynomial equation arising from the form of
such elements $E$ in Defintion~\ref{def_surface} is much more involved than
what we have in equation \eqref{equ_quadric_polynomial_equation}.  Nevertheless, it is possible to extract
a specific form of a quartic polynomial equation in
the vector components of $p$ from equation \eqref{equ_surface_variety}.
The result being unsightly, it will not be presented here.  Suffice it to say, a computerized
algebra system was used to find the polynomial form.  In any case, it is easy
to see from equation \eqref{equ_surface_variety} that the degree of the resulting
polynomial will be 4.

Now notice that under Definition~\ref{def_surface}, canonical forms
such as \eqref{equ_canonical_form_of_common_quadric} are still valid.
This is because
\begin{equation}
P(p)\wedge\overline{P}(p)\cdot E = (\nvao+p)\wedge\overline{(\nvao+p)}\cdot E
\end{equation}
in the case that $E$ is of the form \eqref{equ_quadric_form}.  This allows
us to use what we already know about quadrics in the new model with its extension.

Putting theory into practice, the author wrote a piece of computer
software that implements this conformal-like model for the special class of
quartic surfaces of equation \eqref{equ_surface_variety}.  Giving the program the following
script as input, the output of the program is given in Figure~\ref{fig_invert_cylinder_in_sphere}.
The script is easy for anyone to read, even if they are not familiar with its language.  It is given
here to illustrate how one might use the model with the aide a computer system.
%\begin{samepage}
{\small
\begin{verbatim}

/*
 * Calculate the surface that is the
 * inversion of a cylinder in a sphere.
 */
do
(
    /* Make the cylinder. */
    v = e2,
    c = -7*e1,
    r = 2,
    cylinder = Omega - v^bar(v) + 2*c*nib + (c.c - r*r)*ni^nib,
    bind_quadric(cylinder),
    geo_color(cylinder,0,1,0),
	
    /* Make the sphere. */
    c = 0,
    r = 6,
    sphere = no + c + 0.5*(c.c - r*r)*ni,
    bind_dual_sphere(sphere),
    geo_color(sphere,1,0,0,0.2),
	
    /* Make the inversion of the cylinder in the sphere. */
    V = sphere*bar(sphere),
    inversion = V*cylinder*V~,
    bind_conformal_quartic(inversion),
    geo_color(inversion,0,0,1),
)

\end{verbatim}
}
%\end{samepage}
The functions beginning with the word ``bind'' create and bind an entity to the given
element of the geometric algebra that is responsible for interpreting that element
as a surface under Definition~\ref{def_surface} or as a surface under the definition
given by the conformal model.  The computer program can then
use traditional methods to render the surface from the extracted polynomial equation.
For example, the polynomial equation in $x$, $y$ and $z$ for the inverted surface presented
in Figure~\ref{fig_invert_cylinder_in_sphere} is given by
\begin{equation}
\begin{split}
0 =\;& 28.8x^{2} + 11.2x^{3} + x^{4} + 11.2xy^{2} + 2x^{2}y^{2} + \\
 & 11.2xz^{2} + 2x^{2}z^{2} + y^{4} + 2y^{2}z^{2} + 28.8z^{2} + z^{4}.
\end{split}
\end{equation}
It is interesting how a bit of reasoning in geometric algebra has given us a means
to finding this polynomial equation.
Of course, while such equations lend themselves to computer algorithms, they
are not practical on paper.  This is where the canonical forms of elements might become
useful; although admittedly, even these forms have proven to be unwieldy
and impractical for the author, unlike their conterparts in the conformal
model of geometric algebra.

\section{A Companion Algebra $\G([\B(\W)])$ For $\G(\W)$}

In this section we show that the notions of dual and direct
geometries found in the conformal model of geometric algebra may
be likewise achieved in our model for quartic surface by using
a geometric algebra $\G([\B(\W)])$ in conjunction with $\G(\W)$.
To help differentiate between the elements of these two geometric
algebras, we will use a bold-face font for members of $\G([\B(\W)])$.

Let $\B(\W)$ denote the set of all bivectors $B\in\G(\W)$ of
the form
\begin{equation}
B = \sum_{i=1}^k a_i\overline{b}_i,
\end{equation}
where each of $\{a_i\}_{i=1}^k$ and $\{b_i\}_{i=1}^k$ is a sequence
of $k$ vectors taken from $\V$.  We observe now that $\B(\W)$ is
a linear space, and then seek a geometric algebra generated by this
space.  To that end, let $[]:\B(\W)\to[\B(\W)]$ be a linear function
mapping the bivectors of $\B(\W)$ into a vector space we'll denote by $[\B(\W)]$.
Being a linear function, $[]$ is determined entirely by how it maps a basis
of $\B(\W)$.  Given any basis of $\B(\W)$, therefore, we let $[]$ map
this basis to a set of vectors such that for any two blades $A,B\in\B(\W)$, we have
\begin{equation}
[A]\cdot[B]=A\cdot B.
\end{equation}
We can now let $\G([\B(\W)])$ denote the anti-Euclidean geometric algebra
generated by $[\B(\W)]$, define the function $\rho:\V^e\to[\B(\W)]$ as
\begin{equation}
\rho(p) = [P(p)\wedge\overline{P}(p)],
\end{equation}
and then give the following two definitions.
\begin{defn}\label{def_dual_surf}
For any blade $\mathbf{B}\in\G([\B(\W)])$, if we are refering to it as
a dual surface, then we are interpreting it as the surface that
is the set of all points $p\in\V^e$ such that
\begin{equation}
0 = \rho(p)\cdot\mathbf{B}.
\end{equation}
\end{defn}
\begin{defn}\label{def_direct_surf}
For any blade $\mathbf{B}\in\G([\B(\W)])$, if we are refering to it as a
direct surface, then we are interpreting it as the surface that
is the set of all points $p\in\V^e$ such that
\begin{equation}
0 = \rho(p)\wedge\mathbf{B}.
\end{equation}
\end{defn}
Letting $\mathbf{I}$ denote the unit psuedo-scalar of $\G([\B(\W)])$, it is
not hard to show that if a given blade $\mathbf{B}$ represents a surface
dually, then its positive or negative dual $\pm\mathbf{BI}$ does so directly,
and vice versa.\footnote{Note that it is sometimes useful to reinterpret a given
blade $\mathbf{B}$.  For example, while the dual intersection of two dual
surfaces may be imaginary, it may become a different yet real surface when interpreted
as a direct surface.}

We see now by Definition~\ref{def_dual_surf} that for any surface $E\in\B(\W)$,
the vector $[E]$ is its dual surface.  We can now benefit from the geometric
algebra $\G([\B(\W)])$ in two ways.  First, by realizing that the outer product
of any two dual surfaces, if non-zero, is a dual surface representative of the intersection
of the two surfaces taken in that product; and secondly, by realizing that the
outer product of any two direct surfaces, if non-zero, is a direct surface representative of a
surface that is at least the union of the surfaces taken in the product.  The proof
of these facts is well known, and so will not be presented here.

While it is clear what we get in an intersection of two dual surfaces, it is not immediately
clear what we get in the combination of two direct surfaces.  One way to begin to find out
may be to study the ability to fit surfaces to points as can be done in the conformal
model of geometric algebra, because in that model, all geometries, except the
flat points, can be factored as an outer product of points fitting the geometry, provided
the points satisfy specific constraints.

Given a set $\{p_i\}_{i=1}^k$, with
$1\leq k<\mbox{grade}(\mathbf{I})=(n+2)^2$, of $k$ points
taken from $\V^e$, if the set of vectors $\{\rho(p_i)\}_{i=1}^k$ is linearly
independent, then the direct surface $\mathbf{B}$, given by
\begin{equation}\label{equ_fit_points}
\mathbf{B} = \bigwedge_{i=1}^k\rho(p_i),
\end{equation}
must fit the $k$ points.  Assuming planar intersections, a $k$-blade $\mathbf{B}\in\G([\B(\W)])$
is dually representative
of a surface of dimension $n-k+1$, and therefore directly representative
of a surface of dimension $n-((n+2)^2-k)+1$, which brings up the odd
prospect of geometries with no or negative dimension.  It would seem that
$k=(n+2)^2-1$ points are needed to fit an $n$-dimensional surface; however,
it is possible to fit such a surface with just $k=(n+2)(n+3)/2-1$ points if
we restrict ourselves to the reduced form of surfaces $[\mathbf{E}]^{-1}\in\B(\W)$,
similar to what we did with equation \eqref{equ_quadric_reduced_form} above.

In any case, the question now remains: under what circumstances of the set $\{p_i\}_{i=1}^k$,
if any exist, is the set $\{\rho(p_i)\}_{i=1}^k$ linearly independent; and under those
circumstances,
what surface $\mathbf{B}$ do we get in equation \eqref{equ_fit_points}?  The
version of this question for the conformal model of geometric algebra is easy
to answer; but, not being able to find the answer here for our present model,
the author is compelled to leave it as an open question.

Lastly, given a versor $V\in\G(\V)$, we can see
that for any dual surface $\mathbf{E}\in[\B(\W)]$, if we can find a factorization
of this $k$-blade, such as $\mathbf{E}=\mathbf{E}_1\wedge\dots\wedge\mathbf{E}_k$,
then the
transformation $\mathbf{E}'$ of this surface $\mathbf{E}$ by $V$, is given by
\begin{equation}
\mathbf{E}' = \bigwedge_{i=1}^k[V\overline{V}[\mathbf{E}_i]^{-1}(V\overline{V})^{-1}].
\end{equation}
This illustrates one way that the algebras $\G(\W)$ and $\G([\B(\W)])$ might
work in concert with one another.  The problem of blade factorization is
given a great deal of treatment in \cite{}.

\nocite{Dorst07}
\bibliographystyle{amsplain}
\bibliography{Parkin_AVariationOfTheQuadricModelOfGA}

% cite http://www.math.ethz.ch/~knus/papers/campinas.pdf -> quad forms determine clifford algebras?
% cite http://www.maths.ed.ac.uk/~aar/books/dublin.pdf for history of quadratic forms
% cite wiki entry for algebraic variety?

\end{document}