\documentclass{birkjour}

\usepackage{amsmath}
\usepackage{amssymb}
\usepackage{amsthm}
\usepackage{graphicx}
\usepackage{float}

\newtheorem{thm}{Theorem}[section]
 \newtheorem{cor}[thm]{Corollary}
 \newtheorem{lem}[thm]{Lemma}
 \newtheorem{prop}[thm]{Proposition}
 \theoremstyle{definition}
 \newtheorem{defn}[thm]{Definition}
 \theoremstyle{remark}
 \newtheorem{rem}[thm]{Remark}
 \newtheorem*{ex}{Example}
 \numberwithin{equation}{section}

\newcommand{\G}{\mathbb{G}}
\newcommand{\V}{\mathbb{V}}
\newcommand{\Vb}{\mathbb{\overline{V}}}
\newcommand{\W}{\mathbb{W}}
\newcommand{\R}{\mathbb{R}}
\newcommand{\VS}{\mathbb{S}}

\newcommand{\Alpha}{A}
%\Omega is already defined

\newcommand{\nvao}{o}
\newcommand{\nvai}{\infty}
\newcommand{\nvaob}{\overline{o}}
\newcommand{\nvaib}{\overline{\infty}}

\newcommand{\eminus}{e_{-}}
\newcommand{\eplus}{e_{+}}
\newcommand{\eminusb}{\overline{e}_{-}}
\newcommand{\eplusb}{\overline{e}_{+}}

\begin{document}

\title{An Extension Of The Conformal Model Of Geometric Algebra}

\author{Spencer T. Parkin}
\email{spencer.parkin@gmail.com}

\numberwithin{equation}{section}

\subjclass{Primary 14J70; Secondary 14J29}

\keywords{Quadric Surface, Quartic Surface, Geometric Algebra, Quadric Model, Conformal Model}

%\dedicatory{To Melinda and Naomi}

\begin{abstract}
It is shown that with a trivial extension to the conformal
model of geometric algebra that the model can be made
to take on blades representative of a wider variety geometries.
(Without more than this, the paper isn't worth publishing.)
\end{abstract}

\maketitle

\section{Introduction}

In the papers \cite{} and \cite{}, models for quadric surfaces were proposed
where the underlying geometric algebra represented such surfaces with bivectors.
The problem with this approach, however, was the loss of features we have in a model
where the underlying geometric algebra represents its geometries with blades, which we may
think of as vector sub-spaces.
Using such sub-spaces to represent algebraic varieties in a way similar to that defined by the conformal
model of geometric algebra, we may therefore leverage the operations that can be
perform on such spaces to provide useful operations on geometric surfaces.  For example,
in the conformal model, the join of disjoint vector-subspaces allows for the intersection of any two
geometries represented in dual form.  This will be the principle motivation for the following
extension of the conformal model.  It is assumed here that the reader is already familiar
with the conformal model of geometric algebra, which may be found in \cite{}.

\section{The Extended Model}

We begin by introducing the the following vector spaces.
\begin{equation}
\begin{array}{ll}
\mbox{Notation} & \mbox{Basis} \\
\hline
\V^n & \{e_i\}_{i=1}^n \\
\V^{n+1,1} & \{\nvao,\nvai\}\cup\{e_i\}_{i=1}^n \\
\VS & \{s_{ij}|(i,j)\in[1,n]\times[1,n]\cap\mathbb{Z}\} \\
\V & \V^{n+1,1}\cup\VS
\end{array}
\end{equation}
Here, $\V^n$ denotes an $n$-dimensional Euclidean vector space, and it is this
space that we will use to represent all of $n$-dimensional Euclidean space.  Geometries
will be thought of as subsets of this space.  The vector space $\V^{n+1,1}$ generates
a geometric algebra that is the well-known Minkownski space upon which the conformal
model is based.  The $n^2$-dimensional Euclidean vector space $S$ is new, and we consider
it disjoint from $\V^{n+1,1}$.  Our extended model will be based in the geometric algebra
generated by $\V$.  The basis vector sets generating $\V^n$ and $\VS$ are 
sets of orthonormal vectors.

Owing to \cite{}, the conformal mapping $p:\V^n\to\V$ is given by
\begin{equation}
p(x)=\nvao+x+\frac{1}{2}x^2\nvai.
\end{equation}
To give impetus to our extension, let us illustrate why this mapping fails to provide
a means to representing the classes of geometries known as the quadric surfaces.
While there are many different derivations
of the vector forms of the conformal model that are dually representative of planes and spheres, the following
method will highlight where the failure occurs.  We begin with the vector equation in $\V^n$
for the plane.  It is the solution set of the equation
\begin{equation}\label{equ_plane}
0 = (x-c)\cdot v
\end{equation}
in the vector $x\in\V^n$, where $c\in\V^n$ is a point on the plane, and $v\in\V^n$ is
a unit-length direction vector normal to the plane.  We now make the observation that
$p(x)$ may be factored out of this equation in terms of the inner product, giving us
the equation
\begin{equation}
0 = p(x)\cdot(v+(c\cdot v)\nvai),
\end{equation}
which shows that $v+(c\cdot v)\nvai$ is dually representative of the plane.  Similarly,
given the equation
\begin{equation}\label{equ_sphere}
0 = -r^2+(x-c)^2,
\end{equation}
which we may recognize as yielding the solution set, in $x\in\V^n$, of all points
on the sphere centered at $c\in\V^n$ with radius $r\in\R$, we see that $-2p(x)$ factors
out of this equation in terms of the inner product as
\begin{equation}
0=p(x)\cdot\left(\nvao+x+\frac{1}{2}(c^2-r^2)\nvai\right),
\end{equation}
showing that $p(c)-\frac{1}{2}r^2\nvai$ is dually representative of the said sphere.
But now what about the following equation?
\begin{equation}\label{equ_common_quadric}
0 = -r^2 + (x-c)^2 + \lambda((x-c)\cdot v)^2
\end{equation}
For various choices of $\lambda\in\R$, this vector equation gives us a solution set
in $x$ for a variety of different $n$-dimensional quadric surfaces.  Our problem can
now be seen in the realization that $p(x)$ does not factor out of equation \eqref{equ_common_quadric}
in terms of the inner product.

As one can see, the only troubling term in equation \eqref{equ_common_quadric} is the
term $((x-c)\cdot v)^2$.  To remedy the situation, we will define a new mapping
$P:\V^n\to\V$ as
\begin{equation}
P(x) = p(x)+s(x,x),
\end{equation}
which can be factored out of equation \eqref{equ_common_quadric}.  The
function $s:\V^n\times\V^n\to\VS$ is given by
\begin{equation}
s(a,b) = \sum_{i=1}^n\sum_{j=1}^n(a\cdot e_i)(b\cdot e_j)s_{ij}.
\end{equation}
We now make the observation that for any two vectors $a,b\in\V^n$, we have
\begin{equation}
s(a,a)\cdot s(b,b) = (a\cdot b)^2.
\end{equation}
It then follows that $-2P(x)$ factors out of equation \eqref{equ_common_quadric} as
\begin{equation}
0 = P(x)\cdot\left(\nvao+c+\lambda(c\cdot v)v+\frac{1}{2}(c^2+\lambda(c\cdot v)-r^2)\nvai-
\frac{1}{2}\lambda s(v,v)\right).
\end{equation}

\bibliographystyle{amsplain}
\bibliography{Parkin_AnExtensionOfTheConformalModel}

\end{document}