\documentclass{article}

\usepackage{amsmath}
\usepackage{amssymb}
\usepackage{amsthm}

\addtolength{\oddsidemargin}{-.575in}
\addtolength{\evensidemargin}{-.575in}
\addtolength{\textwidth}{1.0in}
\addtolength{\topmargin}{-.575in}
\addtolength{\textheight}{1.25in}

\newcommand{\R}{\mathbb{R}}
\newcommand{\V}{\mathbb{V}}
\newcommand{\G}{\mathbb{G}}
\newcommand{\prl}{\parallel}
\newcommand{\prp}{\perp}
\newcommand{\nvao}{o}
\newcommand{\nvai}{\infty}
\newcommand{\grade}{\mbox{grade}}

%\swapnumbers
\newtheorem{theorem}{Theorem}[section]
\newtheorem{definition}{Definition}[section]
\newtheorem{corollary}{Corollary}[section]
\newtheorem{identity}{Identity}[section]
\newtheorem{lemma}{Lemma}[section]
\newtheorem{result}{Result}[section]

\title{A Treatment of the Conformal Model\\of\\Geometric Algebra--(Draft)}
\author{Spencer T. Parkin}

\begin{document}
\maketitle

\section{Introduction}

This paper is an attempt to rigorously prove results of the conformal model
of geometric algebra.  Specifically, those about the building up of geometries
from existing geometries through intersection and union-like operations.
We take a ground-up approach, assuming only a familiarity
with linear algebra, geometric algebra and the non-Euclidean geometric algebra used
by the conformal model.  We then go on to propose a method of
representing geometries not native to the conformal model.  Set in the background of
a developed model for geometry, we can properly screwtinize this new method of
representing geometry in a context that raises the right questions about its
practicality in geometric analysis.  Better models of geometry are contemplated.

\section{Lemmas and Theorems}\label{sec_lmas_and_thms}

For the remainder of this paper, we will let
$\G$ denote our geometric algebra, and $\V$ denote the vector space generating
$\G$.  We will let $\V^n$ denote the $n$-dimensional Euclidean vector sub-space
of $\V$.  If $\{e_k\}_{k=1}^n$ is any set of $n$ linearly independent vectors
taken from $\V^n$, then $\{\nvao,\nvai\}\cup\{e_k\}_{k=1}^n$ is a set of $n+2$
basis vectors generating $\V$, where $\nvao$ and $\nvai$ are the null-vectors at
origin and infinity, respectively.  I assume that the reader is already familiar
with these vectors.  We will let $I$ denote the unit psuedo-scalar of $\G$.

\subsection{Lemmas of Vector Algebra}\label{sec_thms_of_vec_alg}

In the proofs to follow, it is important to realize the distinction made between
position vectors and direction vectors.  As a position vector, a vector $v\in\V^n$
represents the position in space at the head of the vector when its tail is placed
at the origin.  As a direction vector, a vector $v\in\V^n$ represents a direction
with respect to other direction vectors, and it doesn't matter where it's placed.
Often direction vectors are placed with their tails in the same place so that their
relative directions can be more easily compared.

To further show the distinction between position and direction vectors, notice that
three vectors taken from $\V^n$, as position vectors, are always co-planar.  That is,
three points in space can always be contained within a single plane.
As direction vectors, however, three vectors taken from $\V^n$ are not always co-planar.
The co-planarity of three direction vectors is tested by placing their tails at
the same point in space, and then testing to see if the four points comprised of their
heads and common tail can be contained within a single plane.  Four points cannot
always be contained within a single plane.

As position vectors will play an important role in defining our geometries,
we will, unless stated otherwise, always be thinking about vectors taken from
$\V^n$ as position vectors.

We also need to talk about the descriptions of geometry used in this paper.
We will often make reference to $m$-dimensional hyper-spheres and $m$-dimensional hyper-planes.
A 0-dimensional hyper-plane is a point in space.  A 1-dimensional hyper-plane is
a line through space.  A 2-dimensional hyper-plane is what we would normally think of
as a plane through space.  A 3-dimensional hyper-plane takes up all of what we normally
think of as space, so hyper-planes of dimensions greater than 3 require a greater amount
of imagination.  A 0-dimensional hyper-sphere is a point in space.  A 1-dimensional hyper-sphere
can be thought of as a pair of points in space, both at a common distance to a center
point.  This idea, of course, generalizes to higher dimensions.  A 2-dimensional hyper-sphere
is a circle, which can be contained in a plane.  A 3-dimensional hyper-sphere is what
we normally think of as a sphere.  4-dimensional spheres require a stretch of the imagination.

All that said, we begin with an important definition that we will use quite often throughout
the paper.

\begin{definition}
A given set of $m$ position vectors $\{v_k\}_{k=1}^m$ taken from $\V^n$, with
$m\geq 2$, is said to be co-$(m-2)$-hyper-planar if for $m=2$, we have $v_1=v_2$;
for $m=3$, the vectors in $\{v_k\}_{k=1}^m$ are co-linear; for $m=4$, the vectors
in $\{v_k\}_{k=1}^m$ are co-planar; for $m=5$, the vectors $\{v_k\}_{k=1}^m$ are
co-hyper-planar; etc.
\end{definition}

\begin{lemma}\label{lma_co_hyper_planar_characterization}
If a set of $m$ distinct position vectors $\{v_k\}_{k=1}^m$ are co-$(m-2)$-hyper-planar,
then there exists distinct integers $i,j\in[1,m]$ such that
\begin{equation*}
v_i\in\mbox{span}\{v_j-v_k\}_{k=1,k\neq i,k\neq j}^m.
\end{equation*}
\end{lemma}
\begin{proof}
Proof pending.
\end{proof}

\begin{lemma}\label{lma_mm3_imp_mm2}
For any integer $m>2$, if a set of $m$ position vectors $\{v_k\}_{k=1}^m$
taken from $\V^n$ is co-$(m-3)$-hyper-planer, then this set of position vectors is
also co-$(m-2)$-hyper-planar.
\end{lemma}
\begin{proof}
Clearly for any $(m-3)$-dimensional hyper-plane there exists an $(m-2)$-dimensional
hyper-plane containing it.
\end{proof}

\begin{lemma}\label{lma_non_co_hyper_planar_imp_diff_set_lin_indep}
If a set of position vectors $\{v_k\}_{k=1}^m$ are non-co-$(m-2)$-hyper-planar,
then for any integer $w\in[1,m]$, the set of $m-1$ vectors $\{v_w-v_k\}_{k=1,k\neq w}^m$
is a linearly independent set of vectors.
\end{lemma}
\begin{proof}
We will prove the contrapositive.  If the set of $m-1$ vectors $\{v_w-v_k\}_{k=1,k\neq w}^m$
are linearly dependent, then the $(m-1)$-dimensional hyper-volume of the $(m-1)$-dimensional
parallel-piped determined by the $m-1$ vectors in $\{v_w-v_k\}_{k=1,k\neq w}^m$ is zero.
Then, at most, this flattened paralle-piped has $(m-1)$-dimensional hyper-area, which is
$(m-2)$-dimensional hyper-volume.  We can now see that if all directional vectors in
$\{v_w-v_k\}_{k=1,k\neq w}^m$ are confined to an $(m-2)$-dimensional hyper-plane, then,
as position vectors, the vectors in $\{v_k\}_{k=1}^m$ must all be co-$(m-2)$-hyper-planar.
Notice that if our flattened
parallel-piped has only volume in lower dimensions, or no volume in any dimension,
then our lemma here still goes through by lemma $\eqref{lma_mm3_imp_mm2}$.
\end{proof}

\begin{lemma}\label{lma_lin_combo_hyper_planar}
Let $m$ be an integer such that $m\geq 2$.
Then, if $\{\alpha_k\}_{k=1}^m$ is a set of $m$ scalars, not all zero, taken from $\R$,
and $\{v_k\}_{k=1}^m$ is a set of $m$ vectors taken from $\V^n$, such that
\begin{equation*}
\mbox{$0=\sum_{k=1}^m\alpha_k$ and $0=\sum_{k=1}^m\alpha_k v_k$,}
\end{equation*}
then the set of $m$ position vectors $\{v_k\}_{k=1}^m$ are co-$(m-2)$-hyper-planar.
\end{lemma}
\begin{proof}
Let us first make the observation that if for all integers $k\in[1,m]$, we have
$\alpha_k=0$, then we can come to no conclusion about the vectors in $\{v_k\}_{k=1}^m$.
Therefore, we must require that the scalars in $\{\alpha_k\}_{k=1}^m$ are not
all zero.

We now make the observation that if there exists an integer $i\in[1,m]$ such that
$\alpha_i\neq 0$, then there must exist an integer $j\in[1,m]$, where $i\neq j$, such
that $\alpha_j\neq 0$ also.
So without loss of generality, let $i=m$ so that $1\leq j<m$.
It now follows that the sum
\begin{equation*}
0 = \sum_{k=1}^{m-1}\alpha_k v_k - \left(\sum_{k=1}^{m-1}\alpha_k\right)v_m
 = \sum_{k=1}^{m-1}\alpha_k(v_k-v_m)
\end{equation*}
is a non-trivial linear combination of the vectors in $\{v_k-v_m\}_{k=1}^{m-1}$.
It then follows that $\{v_m-v_k\}_{k=1}^{m-1}$ is a linearly dependent set of vectors.
Therefore, by the contrapositive of lemma $\eqref{lma_non_co_hyper_planar_imp_diff_set_lin_indep}$,
we see the set of $m$ position vectors $\{v_k\}_{k=1}^m$ are co-$(m-2)$-hyper-planar,
which is what we wanted to show.
\end{proof}

Notice that the following lemma does not follow from the contrapositive of
lemma $\eqref{lma_lin_combo_hyper_planar}$, and that it is a stronger result.
The statement of lemma $\eqref{lma_lin_combo_hyper_planar}$, however, will make
a proof given later on more clear.

\begin{lemma}\label{lma_lin_indep_imp_non_co_hyper_planar}
If a set of position vectors $\{v_k\}_{k=1}^m$ is linearly independent, then
these vectors are also non-co-$(m-2)$-hyper-planar.
\end{lemma}
\begin{proof}
We will prove the contrapositive.  Assuming the position vectors in $\{v_k\}_{k=1}^m$
are co-$(m-2)$-hyper-planar, we see, by lemma $\eqref{lma_co_hyper_planar_characterization}$,
that there exist distinct integers $i,j\in[1,m]$, such that $v_i$ is a linear combination
of the vectors in the set $\{v_j-v_k\}_{k=1,k\neq i,\neq j}^m$.  It is now easy to see
that $v_i$ is also a linear combination of the vectors in the set $\{v_k\}_{k=1,k\neq i}^m$.
It follows that $\{v_k\}_{k=1}^m$ is a linearly dependent set of vectors.
\end{proof}

It is instructive to note that the converse of $\eqref{lma_lin_indep_imp_non_co_hyper_planar}$
is not true.
A simple counter example is given by two parallel vectors of different lengths.

\begin{lemma}\label{lma_lin_indep_not_lin_combo_diff_vecs}
Let $\{v_k\}_{k=1}^m$ be a set of $m$ linearly independent vectors taken from $\V^n$.
Then for any integer $w\in[1,m]$, the vector $v_w$ cannot be written as a linear combination
of the set of $m-1$ vectors in $\{v_w-v_k\}_{k=1,k\neq w}^m$.
\end{lemma}
\begin{proof}
We give a proof here by contradiction.  Suppose there does exist such a linear combination.
Then without loss of generality, choosing $w=m$, there exist $m-1$ scalars $\{\alpha_k\}_{k=1}^{m-1}$,
not all zero, such that
\begin{equation*}
v_m = \sum_{k=1}^{m-1}\alpha_k(v_m-v_k).
\end{equation*}
Solving for $v_m$, we find that
\begin{equation*}
v_m\left(1+\sum_{k=1}^{m-1}\alpha_k\right) = \sum_{k=1}^{m-1}\alpha_k v_k.
\end{equation*}
Suppose now that $-1\neq\sum_{k=1}^{m-1}\alpha_k$.  Then $v_m$ is a
linear combination of the vectors in $\{v_k\}_{k=1}^{m-1}$, which
contracicts the fact that $\{v_k\}_{k=1}^m$ is a linearly independent set.
If $-1=\sum_{k=1}^{m-1}\alpha_k$, then $0=\sum_{k=1}^{m-1}\alpha_k v_k$,
showing that there is a linearly dependent subset of $\{v_k\}_{k=1}^m$,
which also contradicts the fact that $\{v_k\}_{k=1}^m$ is a linearly
independent set.
\end{proof}

\subsection{Lemmas and Theorems of Geometric Algebra}\label{sec_thms_of_geo_alg}

\begin{lemma}\label{lma_psuedo_scalar_with_vec_commutativity}
The commutativity in the geometric product of vectors $v$ taken from $\V$ with the
unit psuedo-scalar $I$ is given by
\begin{equation*}
vI = -(-1)^nIv.
\end{equation*}
\end{lemma}
\begin{proof}
Let $v$ be a vector taken from $\V$.
Then, by definition, we have
\begin{equation*}
vI=v\cdot I+v\wedge I=v\cdot I=\frac{1}{2}(vI-(-1)^nIv).
\end{equation*}
Solving for $-(-1)^nIv$ now gives us the desired result.
\end{proof}

\begin{lemma}\label{lma_dot_wedge_identity}
For any $v\in\V$ and any blade $A\in\G$, we have
\begin{equation*}
(v\cdot A)I = v\wedge(AI).
\end{equation*}
\end{lemma}
\begin{proof}
Let $m=\grade(A)$.  Then, by application of lemma $\eqref{lma_psuedo_scalar_with_vec_commutativity}$,
we see that
\begin{align*}
2(v\cdot A)I &= (vA-(-1)^mAv)I \\
 &= vAI-(-1)^mAvI \\
 &= vAI+(-1)^{n+m}AIv \\
 &= vAI+(-1)^{n+2-m}AIv \\
 &= 2(v\wedge(AI)).
\end{align*}
Notice that $n+2-m=\grade(AI)$.
\end{proof}

\begin{lemma}\label{lma_inner_iff_outer_blade_vs_dual_version1}
For any vector $v\in\V$ and any blade $A\in\G$, we have
\begin{equation*}
v\cdot A=0 \iff v\wedge(AI)=0.
\end{equation*}
\end{lemma}
\begin{proof}
By lemma $\eqref{lma_dot_wedge_identity}$, we see that
\begin{equation*}
v\cdot A=0 \iff (v\cdot A)I=0 \iff v\wedge(AI)=0.
\end{equation*}
\end{proof}

\begin{lemma}\label{lma_inner_iff_outer_blade_vs_dual_version2}
For any vector $v\in\V$ and any blade $A\in\G$, we have
\begin{equation*}
v\wedge A=0 \iff v\cdot(AI)=0.
\end{equation*}
\end{lemma}
\begin{proof}
By lemma $\eqref{lma_inner_iff_outer_blade_vs_dual_version1}$, we see that
\begin{equation*}
v\wedge A=0 \iff -v\cdot(AI)=0 \iff v\cdot(AI)=0.
\end{equation*}
\end{proof}

\begin{lemma}\label{lma_vec_blade_inner_zero}
If $A\in\G$ is a non-zero blade and $v\in\V$ is a vector such that $v\cdot A=0$, then
for all vectors $w\in\V$ such that $w\wedge A=0$, we have $v\cdot w=0$.
\end{lemma}
\begin{proof}
We give a proof by induction.  The case when $A$ is a vector is easy to see.
Suppose the lemma holds for blades $A$ of grade $m-1$.  We will then show that the
lemma holds for blades $A$ of grade $m$.  Let $\{a_k\}_{k=1}^m$ be a set of $m$
vectors such that $A=\bigwedge_{k=1}^m a_k$.  Let $w$ be a vector such that
$w\wedge A=0$.  It follows that $w$ is a linear combination of the vectors in $\{a_k\}_{k=1}^m$.
Let $v$ be a vector such that $v\cdot A=0$.
\begin{equation*}
0 = v\cdot A = v\cdot \bigwedge_{k=1}^m a_k
 = \left(v\cdot\bigwedge_{k=1}^{m-1} a_k\right)\wedge b_m + (-1)^m(v\cdot b_m)\bigwedge_{k=1}^{m-1} b_k
\end{equation*}
From this we see that if $v\cdot A=0$, then $v\cdot\bigwedge_{k=1}^{m-1} a_k=0$ and $v\cdot a_m=0$.
Now, by inductive hypothesis, it follows that for all integers $k\in[1,m-1]$, we have $v\cdot a_k=0$.
Now since $w$ is a linear combination of the vectors in $\{a_k\}_{k=1}^m$, we see that $v\cdot w=0$.
\end{proof}

\begin{theorem}\label{thm_intersect}
Given any $v\in\V$ and any two blades $A,B\in\G$ such that $A\wedge B\neq 0$,
it follows that
\begin{equation*}
\mbox{$v\cdot A=0$ and $v\cdot B=0 \iff v\cdot(A\wedge B)=0$.}
\end{equation*}
\end{theorem}
\begin{proof}
Let $\{a_k\}_{k=1}^i$ be a set of $i$ vectors such that $A=\bigwedge_{k=1}^i a_k$.
Similarly, let $\{b_k\}_{k=1}^i$ be a set of $j$ vectors such that
$B=\bigwedge_{k=1}^j b_k$.  Then, using lemma $\eqref{lma_vec_blade_inner_zero}$ in both
directions of the proof, we have
$v\cdot A=0$ and $v\cdot B=0$ if and only if for all integers $k\in[1,i]$,
we have $v\cdot a_k=0$, and for all integers $k\in[1,j]$, we have $v\cdot b_k=0$, which,
in turn, is true if and only if $v\cdot(A\wedge B)=0$.  Notice that one direction of this
proof fails if $A\wedge B=0$.
\end{proof}

\begin{theorem}\label{thm_union_and_more}
Given any $v\in\V$ and any two blades $A,B\in\G$, we have
\begin{equation*}
\mbox{$v\cdot A=0$ or $v\cdot B=0\implies v\cdot C=0$,}
\end{equation*}
where $C=((AI)\wedge(BI))I$.
\end{theorem}
\begin{proof}
It follows by lemma $\eqref{lma_inner_iff_outer_blade_vs_dual_version1}$ that $v\wedge(AI)=0$ or $v\wedge(BI)=0$.  In turn, this implies
that $v\wedge(AI)\wedge(BI)=0$, which, by lemma $\eqref{lma_inner_iff_outer_blade_vs_dual_version2}$, implies that $v\cdot C=0$.
\end{proof}

It is instructive to note why the converse of theorem $\eqref{thm_union_and_more}$ does not hold.
If $v\wedge(AI)\wedge(BI)=0$, then $v$ may be contained within a vector sub-space of the vector
space represented by $(AI)\wedge(BI)$ that non-trivially intersects the vector sub-spaces represented
by $AI$ and $BI$, in which case $v\wedge(AI)\neq 0$ and $v\wedge(BI)\neq 0$.

\begin{theorem}\label{thm_reinterpret_outer}
Let $A,B\in\G$ be blades.  Then, for all vectors $v\in\V$, we have
\begin{equation*}
v\wedge A=0 \iff v\wedge B=0,
\end{equation*}
if and only if there exists a scalar $\lambda\in\R$ such that $A=\lambda B$.
\end{theorem}
\begin{proof}
This is easy to prove.  Do it here.
\end{proof}

\begin{theorem}\label{thm_reinterpert_inner}
Let $A,B\in\G$ be blades.  Then, for all vectors $v\in\V$, we have
\begin{equation*}
v\cdot A=0 \iff v\cdot B=0,
\end{equation*}
if and only if there exists a scalar $\lambda\in\R$ such that $A=\lambda B$.
\end{theorem}
\begin{proof}
This theorem follows directly from lemma $\eqref{lma_inner_iff_outer_blade_vs_dual_version1}$
and theorem $\eqref{thm_reinterpret_outer}$.
\end{proof}

\begin{theorem}\label{thm_basis_for_blade}
Let $\{v_k\}_{k=1}^m$ be a linearly independent set of vectors taken from $\V$ and
let $A$ be an $m$-blade taken from $\G$.
Then if for all integers $k\in[1,m]$, we have $v_k\wedge A=0$, then there
exists a scalar $\lambda\in\R$ such that
\begin{equation*}
A = \lambda\bigwedge_{k=1}^m v_k.
\end{equation*}
\end{theorem}
\begin{proof}
For a given integer $k\in[1,m]$, if $v_k\wedge A=0$, then $v_k$ is in the
vector sub-space represented by $A$.  Then, since all such $v_k$ are in the
vector sub-space represented by $A$, and since $\{v_k\}_{k=1}^m$ is a
linearly independent set of vectors, it follows that $\{v_k\}_{k=1}^m$
is also a set of basis vectors generating the vector sub-space represented by $A$.
Therefore, $\bigwedge_{k=1}^m v_k$ represents the same vector sub-space, and is
therefore a scalar multiple of $A$.
\end{proof}

Another way to think about theorem $\eqref{thm_basis_for_blade}$ is, after
realizing that $A$ and $\bigwedge_{k=1}^m v_k$ represent the same vector sub-space,
to realize that for all $v\in\V$, we have $v\wedge A=0$ if and only if
$v\wedge\bigwedge_{k=1}^m v_k=0$.  Theorem $\eqref{thm_basis_for_blade}$ now follows from
theorem $\eqref{thm_reinterpret_outer}$.

\subsection{Theorems of Conformal Geometric Algebra}\label{thms_of_conf_geo_alg}

\begin{definition}
We define the function $p:\V^n\to\G$ as
\begin{equation*}
p(v) = \nvao + v + \frac{1}{2}v^2\nvai.
\end{equation*}
\end{definition}

\begin{theorem}\label{thm_inf_lin_combo_hyper_planar}
Given a set of $m$ position vectors $\{v_k\}_{k=1}^m$ taken from $\V^n$,
if there exists a scalar $\lambda\in\R$ and a set of $m$ scalars
$\{\alpha_k\}_{k=1}^m$, not all zero, taken from $\R$ such that
\begin{equation*}
\lambda\nvai = \sum_{k=1}^m\alpha_k p(v_k),
\end{equation*}
then the set of $m$ position vectors in $\{v_k\}_{k=1}^m$ are co-$(m-2)$-hyper-planar.
\end{theorem}
\begin{proof}
By equating parts, it is easy to see that
\begin{equation*}
\mbox{$0 = \sum_{k=1}^m\alpha_k$ and $0=\sum_{k=1}^m\alpha_k v_k$.}
\end{equation*}
Our theorem now follows directly from lemma $\eqref{lma_lin_combo_hyper_planar}$.
\end{proof}

An important corollary of theorem $\eqref{thm_inf_lin_combo_hyper_planar}$ now follows.

\begin{corollary}\label{cor_non_hyper_planar_lin_indep}
If a given set of $m$ position vectors $\{v_k\}_{k=1}^m$ taken from $\V^n$
are non-co-$(m-2)$-hyper-planar, then the set of vectors $\{p(v_k)\}_{k=1}^m$
are linearly independent.
\end{corollary}
\begin{proof}
By the contrapositive of theorem $\eqref{thm_inf_lin_combo_hyper_planar}$,
there does not exist any scalar $\lambda\in\R$ nor set of scalars
$\{\alpha_k\}_{k=1}^m$, not all zero, such that
\begin{equation*}
\lambda\nvai = \sum_{k=1}^m\alpha_k p(v_k).
\end{equation*}
This is therefore also true when $\lambda=0$.
\end{proof}

Note that the converse of this corollary is not true.  A counter-example is easy to find once we
have developed the conformal geometries later on.

For sake of interest, we give now another proof of corollary $\eqref{cor_non_hyper_planar_lin_indep}$
that offers another perspective.

\begin{proof}
By lemma $\eqref{lma_non_co_hyper_planar_imp_diff_set_lin_indep}$, since
the set of vectors in $\{v_k\}_{k=1}^m$ are non-co-$(m-2)$-hyper-planar, the set of $m-1$
vectors in $\{v_m-v_k\}_{k=1}^{m-1}$ are linearly independent.  Then, by lemma
$\eqref{lma_lin_indep_not_lin_combo_diff_vecs}$, we see that the vector
$v_m-v_{m-1}$ cannot be written as a linear combination of the $m-2$ vectors in the
set $\{(v_m-v_{m-1})-(v_m-v_k)\}_{k=1}^{m-2}=\{v_k-v_{m-1}\}_{k=1}^{m-2}$.

We now prove the theorem by contradiction.  Suppose that $\{p(v_k)\}_{k=1}^m$
is a linearly dependent set of vectors.  Then there exists a set of
$m-1$ scalars $\{\alpha_k\}_{k=1}^{m-1}$ such that
\begin{equation*}
p(v_m)=\sum_{k=1}^{m-1}\alpha_k p(v_k).
\end{equation*}
By equating parts, we see that
\begin{equation*}
\mbox{$1 = \sum_{k=1}^{m-1}\alpha_k$ and $v_m = \sum_{k=1}^{m-1}\alpha_k v_k$.}
\end{equation*}
It now follows that
\begin{align*}
v_m-v_{m-1} &= -v_{m-1}+\sum_{k=1}^{m-1}\alpha_k v_k \\
 &= (\alpha_{m-1}-1)v_{m-1} + \sum_{k=1}^{m-2}\alpha_k v_k \\
 &= \left(\alpha_{m-1}-\sum_{k=1}^{m-1}\alpha_k\right)v_{m-1} + \sum_{k=1}^{m-2}\alpha_k v_k \\
 &= \alpha_{m-1}v_{m-1} - \alpha_{m-1}v_{m-1} - \sum_{k=1}^{m-2}\alpha_k v_{m-1} + \sum_{k=1}^{m-2}\alpha_k v_k \\
 &= -\sum_{k=1}^{m-2}\alpha_k(v_{m-1}-v_k),
\end{align*}
which is a contradiction to our earlier conclusion about the vector $v_m-v_{m-1}$.
\end{proof}

\begin{lemma}\label{lma_lin_dep_conf_vecs_imp_lin_dep_pos_vec}
Let $\{v_k\}_{k=1}^m$ be a set of $m$ vectors taken from $\V^n$.
Then if the set of vectors $\{p(v_k)\}_{k=1}^m$ is linearly dependent,
then so is the set of vectors $\{v_k\}_{k=1}^m$.
\end{lemma}
\begin{proof}
There must exist a set of $m-1$ scalars $\{\alpha_k\}_{k=1}^{m-1}$, such that
\begin{equation*}
p(v_m) = \sum_{k=1}^{m-1}\alpha_k p(v_k).
\end{equation*}
Then, by equating parts, it follows that
\begin{equation*}
v_m = \sum_{k=1}^{m-1}\alpha_k v_k,
\end{equation*}
showing that $\{v_k\}_{k=1}^m$ is a linearly dependent set of vectors.
\end{proof}

Alternatively, we could have proven lemma $\eqref{lma_lin_dep_conf_vecs_imp_lin_dep_pos_vec}$
using lemma $\eqref{lma_lin_indep_imp_non_co_hyper_planar}$
and corollary $\eqref{cor_non_hyper_planar_lin_indep}$.  It is worth mentioning this so that
we gain an insight into the necessary and sufficient conditions upon which $\{p(v_k)\}_{k=1}^m$
is a linearly independent set of vectors.

Note that the converse of lemma $\eqref{lma_lin_dep_conf_vecs_imp_lin_dep_pos_vec}$ is
not true.  Simply consider a unit-length vector $v\in\V^n$, and compare $p(v)$ to $p(2v)$.

\section{Results of the Conformal Model}\label{sec_app_of_thms}

We are now ready to develop the conformal model of geometric algebra.

\subsection{Geometric Representation}

\begin{definition}\label{def_norm_geo_rep}
A blade $A\in\G$ normally represents a geometry as the set of a
all position vectors $v\in\V^n$ such that $p(v)\cdot A=0$.
\end{definition}

\begin{definition}\label{def_dual_geo_rep}
A blade $A\in\G$ dually represents a geometry as the set of all
position vectors $v\in\V^n$ such that $p(v)\wedge A=0$.
\end{definition}

Our first result reveals the homogenous nature of our geometric representations.
\begin{result}\label{rslt_homog_rep}
If a blade $A\in\G$ normally represents a geometry, then so
does $\lambda A$ for any non-zero scalar $\lambda\in\R$.  Similarly,
if a blade $A\in\G$ dually represents a geometry, then so
does $\lambda A$ for any non-zero scalar $\lambda\in\R$.
\end{result}
\begin{proof}
This follows directly from definitions $\eqref{def_norm_geo_rep}$ and $\eqref{def_dual_geo_rep}$.
\end{proof}

It is important to realize that there is a dual relationship between the normal
and dual representations of a given geometry.
\begin{result}\label{rslt_dual_relationship}
If an $m$-blade $A\in\G$ normally represents a geometry, then the
$(n-m)$-blade $AI$ dually represents the same geometry.  Similarly,
if $A$ dually represents a geometry, then $AI$ normally represents
the same geometry.
\end{result}
\begin{proof}
This result follows directly from lemmas $\eqref{lma_inner_iff_outer_blade_vs_dual_version1}$
and $\eqref{lma_inner_iff_outer_blade_vs_dual_version2}$.
\end{proof}

\subsection{Generating all $m$-dimensional Hyper-Spheres of the Model}\label{sec_gen_all_spheres}

Here we begin with a development of the $n$-dimensional hyper-sphere.
All other lower-dimensional hyper-spheres can then be found by simply
intersecting such spheres together.  It is interesting to note that
even the intersection of two non-intersecting geometries gives a meaningful
result in the conformal model.

\begin{result}\label{rslt_sphere_def}
Define $s:\V^n\times\R\to\G$ as the function
\begin{equation*}
s(v,r)=p(v) - \frac{1}{2}r^2\nvai.
\end{equation*}
Then $s(v,r)$ normally represents a sphere positioned at $v$ with radius $r$.
\end{result}
\begin{proof}
Suppose $w\in\V^n$ is a vector such that $p(w)\cdot s(v,r)=0$.
This gives us
\begin{equation*}
\left(\nvao+w+\frac{1}{2}w^2\nvai\right)\cdot\left(\nvao+v+\frac{1}{2}(v^2-r^2)\nvai\right)=0.
\end{equation*}
When expanded, we get
\begin{equation*}
-\frac{1}{2}v^2+w\cdot v-\frac{1}{2}w^2=-\frac{1}{2}r^2.
\end{equation*}
It is then not hard to see that this reduces to $(v-w)^2=r^2$.
It follows that the set of all such vectors $w$ satisfying the
equation $p(w)\cdot s(v,r)=0$ are exactly those at a distance $r$ from $v$.
\end{proof}

Notice that from this we can deduce a normal representation of a single
point in space, and for any given position vector $v\in\V^n$, it is simply
$p(v)$, a sphere with radius zero.

\begin{result}\label{rslt_intersection}
Let $A,B\in\G$ be blades normally representative of any two types of geometries.
Then if $A\wedge B\neq 0$, then the blade $A\wedge B$ normally represents the
intersection of the geometries represented by $A$ and $B$.
\end{result}
\begin{proof}
This result follows directly from theorem $\eqref{thm_intersect}$.
Notice that both directions of the theorem are needed.
\end{proof}

\begin{result}\label{rslt_sphere_is_outer_prod_of_spheres}
If $\{s(v_k)\}_{k=1}^m$ is a set of $m$ vectors normally representative
of $n$-dimensional hyper-spheres such that their combined
intersection is non-trivial, (which is to say their intersection is some hyper-sphere of radius $r>0$),
then this hyper-sphere is of dimension $n-m+1$.
\end{result}
\begin{proof}
This follows from the fact that for any $m$-dimensional hyper sphere, with $1\leq m\leq n$,
a non-trivial intersection of this hyper-sphere with one of dimension $n$ is a hyper-sphere of
dimension $m-1$.
\end{proof}

There are two important take-home points about this section of the paper.
The first is that we can build up all spheres of dimensions 0 through $n$ as outer
products of vectors normally representative of $n$-dimensional hyper-spheres.
The second is that any blade normally representative of a non-degenerate sphere, (one of radius $r>0$),
can be thought of as an outer product of one or more vectors, each normally representative
of an $n$-dimensional hyper-sphere.  For a given hyper-sphere, there are uncountably many
such ways to characterize the sphere.  It is a powerful concept to be
able to choose any one of many different ways of representing a single geometry in an outer product
of elements representative of other geometries.
On the other hand, it is also powerful for us to be able to solve for an element
representative of a geometry that is the intersection, (or union
or what have you), of a set of given geometries.  These two ideas will surface again later.

\subsection{Generating all $m$-dimensional Hyper-Planes of the Model}\label{sec_gen_all_planes}

Each $m$-dimensional hyper-sphere uniquely determines an $m$-dimensional hyper-plane.
The main result of this section is to show how all $m$-dimensional hyper-planes
of the conformal model can be generated as a function of all $m$-dimensional hyper-spheres
of the model.  To do this, we begin with a result that,
in and of itself, is a very useful feature of the conformal model.

\begin{result}\label{rslt_sphere_factorization}
If $A\in\G$ is an $m$-blade that is dually representive of an
$(m-1)$-dimensional hyper-sphere of radius $r>0$, then
there exists a factorization of $A$ of the form
\begin{equation*}
A = \lambda\bigwedge_{k=1}^m p(v_k),
\end{equation*}
where $\lambda\in\R$, and each position vector in $\{v_k\}_{k=1}^m$ is a
point on the sphere represented by $A$.
\end{result}
\begin{proof}
%In section $\eqref{sec_gen_all_spheres}$
In the previous section
we proved the existence of blades normally representative of hyper spheres
of any of the dimensions 0 through $n$.  So, by result $\eqref{rslt_dual_relationship}$,
we know that there exist blades dually representative of all such hyper spheres.
We may therefore let the $m$-blade $A$ be such a blade.  The dimension of the
hyper sphere represented by $A$ is therefore $n-(n+2-m)+1=m-1$.

Now since $r>0$, there exists a set of $m$ distinct position vectors $\{v_k\}_{k=1}^m$, each being on the
sphere represented by $A$.  Then, interestingly, for all $2\leq m<4$, this implies their
non-co-$(m-2)$-hyper-planarity, but this does not follow for all $m\geq 4$.
In any case, it shouldn't be hard to see that $m$ points can be found on an $(m-1)$-dimensional
hyper-sphere of radius $r>0$ such that they are non-co-$(m-2)$-hyper-planar.

Then, by corollary $\eqref{cor_non_hyper_planar_lin_indep}$,
we see that $\{p(v_k)\}_{k=1}^m$ is a linearly independent set.
Now since for each integer $k\in[1,m]$, we have $p(v_k)\wedge A=0$, it follows from
theorem $\eqref{thm_basis_for_blade}$ that the scalar $\lambda\in\R$ above exists.
\end{proof}

There are at least two fascinating things about this result, and they're both important.
The first is that any
blade dually representative of a hyper-sphere can be thought of as any outer
product of vectors normally representative of a finite subset of distinct points on
that sphere that are also non-co-hyper-planar for a hyper-plane of the correct dimension.
The second is that the outer product of a given set of vectors normally representative of
points gives us a blade dually representative of the sphere containing
those points, provided, again, that those points are non-co-hyper-planar for a hyper-plane
of the right dimension.  This provides yet another way for us to generate all
hyper-spheres of the model, this time building higher-dimensional spheres with
blades of larger grade, as apposed to lower-dimensional spheres.

\pagebreak
\begin{result}\label{rslt_sphere_to_plane}
Let $A\in\G$ be an $m$-blade that is dually representive of an
$(m-1)$-dimensional hyper-sphere of radius $r>0$.
Then the $(m+1)$-blade $A\wedge\nvai$ is dually representative
of the $(m-1)$-dimensional hyper-plane containing the
hyper-sphere represented by $A$.
\end{result}
\begin{proof}
By result $\eqref{rslt_sphere_factorization}$, there exists a
scalar $\lambda\in\R$ and a set of $m$ vectors $\{p(v_k)\}_{k=1}^m$,
such that $A = \lambda\bigwedge_{k=1}^m p(v_k)$.
Now since the position vectors $\{v_k\}_{k=1}^m$ are non-co-$(m-2)$-hyper-planar,
we see that $\nvai$ is not any non-trivial linear combination of the
$m$ vectors $\{p(v_k)\}_{k=1}^m$ by theorem $\eqref{thm_inf_lin_combo_hyper_planar}$,
and therefore $A\wedge\nvai\neq 0$.

Now consider the set of all
vectors $v\in\V^n$ such that $p(v)\wedge A\wedge\nvai=0$.
This implies that if $p(v)\wedge A\neq 0$, then there
exists a set of $m+1$ scalars $\{\alpha_k\}_{k=1}^{m+1}$, not all zero,
such that
\begin{equation*}
\nvai = \sum_{k=1}^m \alpha_k p(v_k) + \alpha_{m+1}p(v).
\end{equation*}
If such a non-trivial linear combination does not exist, then $p(v)\wedge A=0$,
and so $v$ is on the hyper-sphere, which is in the desired hyper-plane.
Therefore, what we need to show is that the remainder of the hyper-plane
is given by the set of all position vectors $v\in\V^n$ for which this non-trivial linear
combination does exist.  This follows directly from theorem $\eqref{thm_inf_lin_combo_hyper_planar}$.
\end{proof}

It is now clear that we can generate all the hyper-planes of the conformal
model by simply taking all dual representatives of the hyper-spheres
in the outer product with the null-vector at infinity.
Notice, however, that result $\eqref{rslt_intersection}$ is not limited in its applicability
to intersecting spheres, but can apply to any geometry normally represented in the model.
The normal representation of a hyper-plane can be found using $\eqref{rslt_dual_relationship}$.
Hyper-planes can then be intersected to generate lower dimensional hyper-planes, but notice
that we cannot generate all $m$-dimensional hyper-planes of the model this way, because
the interesection of two $n$-dimensional hyper-planes is always an $n$-dimensional hyper-plane.
(This is because all $n$-dimensional hyper-planes are parallel.)
At least one of the two hyper-planes intersected must be of dimension less than $n$.

\begin{result}
Let $A\in\G$ be an $m$-blade normally representative of an $(n-m+1)$-dimensional
hyper-sphere.  Then $A\cdot\nvai$ is normally representative of the $(n-m+1)$-dimensional
hyper-plane containing this hyper-sphere represented by $A$.
\end{result}
\begin{proof}
Clearly $AI$ is an $(n+2-m)$-blade dually representative of the hyper-sphere.
Then $(AI)\wedge\nvai$ is dually representative of the hyper-plane of lowest
possible dimension that still contains this hyper-sphere.  Then
$((AI)\wedge\nvai)(-I)=A\cdot\nvai$ is a normal representation of this hyper-plane.
\end{proof}

\begin{lemma}
If $A\in\G$ is dually representative of an $(m-1)$-dimensional hyper-sphere, then for any
point $v\in\V^n$ not on this sphere, but in the $(m-1)$-dimensional hyper-plane containing
the sphere, the $(m+1)$-blade $A\wedge p(v)$ is dually representative of the $(m-1)$-dimensional
hyper-plane containing the sphere.
\end{lemma}
\begin{proof}
Proof pending.
\end{proof}

We'll later be able to show that if two blades represent the same plane in the same
way, then they're scalar multiples of one another.  It will then follow that after
formulating a plane in terms of 3 co-circular points (any 3 points that are non-co-linear)
and $\nvai$, we can find a factorization
of that plane in terms of any 4 co-planar points on that plane that are also non-co-circular.

We will need the following lemma in a later section, but while we're on the
subject of hyper-planes, it belongs in this section.
\begin{lemma}\label{lma_planes_contain_nvai}
If $A\in\G$ is a blade dually representative a hyper-plane of any dimension,
then $A\wedge\nvai=0$.
\end{lemma}
\begin{proof}
Proof pending.
\end{proof}

\subsection{Finding Fittings in the Conformal Model}

Let $C$ be a blade in $\G$ normally representative of a circle and
let $P$ in $\V$ be a vector normally representative of a point.
From what we now know, the blade $S$ normally representative
of the sphere, if any, containing the circle and point is given by
\begin{equation*}
S = (P\wedge(CI))(-I).
\end{equation*}
Here we have simply taken a dual of $C$ as $CI$, which we know is
dually representative of the circle, and may be thought of as the
outer product of any three distinct points on the circle.  Then, the
point $P$, when taken in the outer product with $CI$, gives us a
blade dually representative of the sphere containing the point and
circle, provided $P\wedge(CI)\neq 0$.  We then take a final dual
to formulate $S$ as a normal representative of the sphere.
Interestingly, as the reader can check, this reduces nicely to
the formula
\begin{equation*}
S = P\cdot C.
\end{equation*}

Now let $C$ be a blade in $\G$ dually representative of a circle and
let $S$ be a sphere normally representative of a sphere.
A blade $P$ dually representative of a pair of points is then
given by
\begin{equation*}
P = (S\wedge(CI))(-I) = S\cdot C.
\end{equation*}
Taking a dual of $C$, we get a blade that may be thought of as
the outer product of two blades normally representative of spheres.
Then, by forming the outer product of three spheres, the blade
$S\wedge(CI)$ is normally representative of a 1-dimensional sphere,
which is a pair of points.  Taking a final dual, we get the
dual representation of this point-pair, which may also be
thought of as the outer product of the vectors normally representative
of those points.

\subsection{Geometric Analysis}

As noted earlier, given a circle, there are uncountably many ways such a geometry
can be formulated as the intersection between two spheres.  There is, howerver, up to scale,
a unique representation of this circle in terms of the outer product between a blade
representative of a plane, and a blade representative of a sphere centered on that plane.
This latter presentation is easy to decompose, giving us the circle's characteristics,
its center and radius.  These characteristics, however, are not so easily gleaned from
the latter representation.  Therefore, if we wish to intersect two spheres, and then
analyze the result to determine the resulting circle, it would be helpful to know
what, if any, replationship their might be between this, (or any other), representation
of the circle, and the unique representation we have found which
lends itself easily to decomposition.

\begin{lemma}\label{lma_equiv_dual_reps_of_spheres}
If $A,B\in\G$ are blades dually representative of the same $m$-dimensional hyper-sphere,
then there exists $\lambda\in\R$ such that $A=\lambda B$.
\end{lemma}
\begin{proof}
By result $\eqref{rslt_sphere_factorization}$, let $\alpha\in\R$ be a scalar
and $\{v_k\}_{k=1}^{m+1}$ be a set of $m+1$ vectors taken from $\V^n$ such that
$A=\alpha\bigwedge_{k=1}^{m+1}p(v_k)$.  Now, since $B$ represents the same
hyper-sphere, it follows that for all integers $k\in[1,m+1]$, we
have $p(v_k)\wedge B=0$.  It follows then, by theorem $\eqref{thm_basis_for_blade}$, that
there must exist $\beta\in\R$ such that $B=\beta\bigwedge_{k=1}^{m+1}p(v_k)$.
We now see that $\lambda=\alpha/\beta$.
\end{proof}

\begin{lemma}\label{lma_equiv_norm_reps_of_spheres}
If $A,B\in\G$ are blades normally representative of the same $m$-dimensional hyper-sphere,
then there exists $\lambda\in\R$ such that $A=\lambda B$.
\end{lemma}
\begin{proof}
Clearly $AI$ and $BI$ are dually representative of the same $m$-dimensional hyper-sphere.
So it follows from lemma $\eqref{lma_equiv_dual_reps_of_spheres}$ that there exists $\lambda\in\R$
such that $AI=-\lambda BI\implies A=\lambda B$.
\end{proof}

\begin{lemma}\label{lma_equiv_dual_reps_of_planes}
If $A,B\in\G$ are blades dually representative of the same $m$-dimensional hyper-plane,
then there exists $\lambda\in\R$ such that $A=\lambda B$.
\end{lemma}
\begin{proof}
Choose any $m$-dimensional hyper-sphere contained within the $m$-dimensional hyper-plane.
Then by result $\eqref{rslt_sphere_factorization}$, there exists a scalar $\alpha\in\R$
and a set of $m+1$ vectors $\{v_k\}_{k=1}^{m+1}$ such that $\alpha\bigwedge_{k=1}^{m+1}p(v_k)$
is dually representative of this $m$-dimensional hyper-sphere.  Furthermore, by
result $\eqref{rslt_sphere_to_plane}$, $\alpha\nvai\wedge\bigwedge_{k=1}^{m+1}p(v_k)$ is dually representative
of the $m$-dimensional hyper-plane.  By theorem $\eqref{thm_basis_for_blade}$, we need now only
make the observation that we have
found a common basis for the vector sub-spaces represented by $A$ and $B$, since
for all integers $k\in[1,m+1]$, we have $p(v_k)\wedge A=p(v_k)\wedge B=0$, and, by lemma $\eqref{lma_planes_contain_nvai}$,
we have $\nvai\wedge A=\nvai\wedge B=0$.
\end{proof}

\begin{lemma}\label{lma_equiv_norm_reps_of_planes}
If $A,B\in\G$ are blades normally representative of the same $m$-dimensional hyper-plane,
then there exists $\lambda\in\R$ such that $A=\lambda B$.
\end{lemma}
\begin{proof}
This is easily proved from lemma $\eqref{lma_equiv_dual_reps_of_planes}$ just as
we proved lemma $\eqref{lma_equiv_norm_reps_of_spheres}$ from lemma $\eqref{lma_equiv_dual_reps_of_spheres}$.
\end{proof}

We now come to the main result of this section.

\begin{result}\label{rslt_reinterp}
If $A,B\in\G$ are blades both normally or both dually representative of the same
$m$-dimensional hyper-sphere or $m$-dimensional hyper-plane, then there exists
a scalar $\lambda\in\R$ such that $A=\lambda B$.
\end{result}
\begin{proof}
This follows directly from lemmas $\eqref{lma_equiv_dual_reps_of_spheres}$,
$\eqref{lma_equiv_norm_reps_of_spheres}$, $\eqref{lma_equiv_dual_reps_of_planes}$ and
$\eqref{lma_equiv_norm_reps_of_planes}$.
\end{proof}

The question now arrises, if we have two blades $A,B\in\G$ representative, in the same way,
(both normally or both dually), of the same geometry, then does there exist a scalar
$\lambda\in\R$ such that $A=\lambda B$?  Well, if spheres and planes of various dimensions
are the only types of geometries representable by blades in the conformal model, then the
answer is yes.  While it may be possible to show that blades in the geometric algebra of
the conformal model are only dually or normally representative of spheres or planes,
(or that perhaps this claim can be refuted with a counter-example), we will be content
to end this section now with this as an open question, as the result shown above
is sufficient to perform analyices of intersections and fittings taken in the
conformal model between spheres and planes of various dimensions.

\section{Standard Forms}

Standard forms for planes, lines and circles are given here along
with their decomposition formulas.

\subsection{Planes}

A standard form for the normal representation of a plane is given by
\begin{equation*}
\pi(n,c) = n+(c\cdot n)\nvai,
\end{equation*}
where $n$ is a unit-length vector taken from $\V^n$ and $c$ is a position vector taken from $\V^n$.
As the reader can check, the set of all $v\in\V^n$ such that $p(v)\cdot\pi(n,c)=0$ forms a plane.
To decompose this geometry, we have the following.
\begin{align*}
n &= \nvao\cdot(\nvai\wedge\pi(n,c)) \\
c &= (\nvao\cdot\pi(n,c))n
\end{align*}
Notice that the position vector $c$ calculated here will always be the point on the plane
closest to the origin.  If given a vector $v$ normally representative of a plane,
calculating $w=\nvao\cdot(\nvai\wedge v)$ gives a vector orthogonal to the plane, but it
may not be of unit length.  The decomposition formulas above, therefore, can be applied
to the vector $v/|w|$.

\subsection{Circles}

A standard form for the normal representation of a circle may be given by
\begin{equation*}
C(n,c,r) = \pi(n,c)\wedge s(c,r),
\end{equation*}
where $n$ is a unit-length vector taken from $\V^n$, $c$ is a position vector taken from $\V^n$
and $r$ is a scalar taken from $\R$.  Its decomposition formulas are as follows.
\begin{align*}
n &= (\nvai\wedge\nvao)\cdot(C(n,c,r)\wedge\nvai) \\
c &= [(\nvao\wedge\nvai)\cdot[(\nvai\nvao)\wedge C(n,c,r)]]n \\
r^2 &= 2[(\nvai\wedge\nvao)\cdot[C(n,c,r)\wedge\nvao]-(c\cdot n)c]n+c^2
\end{align*}
Again, for a given 2-blade normally representative of a circle, the vector $n$
calculated above may not be of unit-length, in which case the 2-blade needs to
be divided through by the length of $n$ before the rest of the decomposition
formulas are applied.  In other words, these decomposition formulas apply to
the homogenized version of the 2-blade.

\subsection{Lines}

\section{Representing Other Types of Geometry}

So far we have seen that blades of the conformal model of geometric algebra
may be representative of hyper-spheres and hyper-planes.  Here we investigate
what other types of geometries are representable in the conformal model using
blade-valued functions of vectors in $\V^n$.

\begin{definition}
A function $g:\V^n\to\G$ mapping vectors in $\V^n$ to blades in $\G$ is
normally representative of some geometry as the set of all vectors $v\in\V^n$
such that $p(v)\cdot g(v)=0$.  Similarly, this function is dually representative
of some geometry as the set of all vector $v\in\V^n$ such that $p(v)\wedge g(v)=0$.
\end{definition}
We will refer to such functions $g$ as geometric functions.

\subsection{Tubes}

The Euclidean vector form of the equation for a tube is given by
\begin{equation*}
(v-c)^2-[(v-c)\cdot n]^2-r^2 = 0.
\end{equation*}
This geometry is normally represented by the geometric function
\begin{equation*}
g(v)=\nvao+f(v)+\frac{1}{2}[f^2(0)-r^2]\nvai,
\end{equation*}
where $f(v)=c+[(v/2-c)\cdot n]n$.

\subsection{Right-Circular Conical Surfaces}

The Euclidean vector form of the equation for the right-circular conical surface is given by
\begin{equation*}
(v-c)^2-2[(v-c)\cdot n]^2=0.
\end{equation*}
This geometry is normally represented by the geometric function
\begin{equation*}
g(v)=\nvao+f(v)+\frac{1}{2}[f^2(v)-2(c\cdot n)^2]\nvai,
\end{equation*}
where $f(v)=c+[(v-2c)\cdot n]n$.

\subsection{The General Conical Surface}

This geometry is normally represented by the geometric function
\begin{equation*}
g(v) = [1-f(v)]^2\nvao + [1-f(v)](c-f(v)e)+\frac{1}{2}[(c-f(v)e)^2-r^2]\nvai,
\end{equation*}
where $f(v)$ is defined as
\begin{equation*}
f(v) = \frac{(v-e)\cdot(v-c)}{(v-e)^2}.
\end{equation*}
Here, $e$ is the center of the geometry, $c$ is the center of a sphere aiding
in the composition of the geometry, and $r$ is the radius of that sphere.

Replacing $e$ with $e+\lambda n$, where $n$ is a unit-length vector in $\V^n$,
taking the limit of $g(v)$ as $\lambda$ approaches infinity gives us a geometric
for the tube as
\begin{equation*}
g(v) = \nvao + f(v) + \frac{1}{2}[f^2(v)-r^2]\nvai,
\end{equation*}
where $f(v)=c+[(v-c)\cdot n]n$.

\subsection{Special-Case Ellipsoids}

The Euclidean vector form of the equation for a special-case ellipsoid is given by
\begin{equation*}
(v-c)^2+[(v-c)\cdot a]^2-r^2=0.
\end{equation*}
This geometry is normally represented by the geometric function
\begin{equation*}
g(v) = \nvao + c + [(c-v/2)\cdot a]a + \frac{1}{2}[c^2+(c\cdot a)^2-r^2]\nvai.
\end{equation*}

Another special-case ellipsoid is given by the following equation involving
Euclidean vectors and a Euclidean 2-blade $P$.
\begin{equation*}
(v-c)^2+[(v-c)\cdot P]^2-r^2 = 0.
\end{equation*}
A geometric function normally representative of this geometry is given by
\begin{equation*}
g(v) = \nvao+c+\frac{1}{2}[c^2-[(v-c)\cdot P]^2-r^2]\nvai.
\end{equation*}

\subsection{The General Ellipsoid}

An equation for the general ellipsoid in $n$-dimensional space is given by
\begin{equation*}
\sum_{k=1}^n((v-c)\cdot a_k)^2 - r^2 = 0,
\end{equation*}
where $c$ is the center of the ellipsoid and each $a_k$ is a
distortion axis.  The scalar $r$ is a radius for the special case of ellipsoids that
are spheres.  This happens when the vectors in $\{a_k\}_{k=1}^n$ form an orthonormal
basis, in which case, it is easy to see that our equation reduces to $(v-c)^2-r^2=0$.
The vectors in $\{a_k\}_{k=1}^n$ need not be pair-wise orthogonal, but there is no
loss in generality if we assume this requirement, and they do need to form a linearly
independent set of vectors.  In the case that they are pair-wise orthogonal,
it is easy to see how the lengths of these vectors
determine the distortion of a sphere forming the desired ellipsoid, and to see
how that ellipsoid is oriented in space.
A geometric function normally representative of the ellipsoid is given by
\begin{equation*}
g(v) = \sum_{k=1}^n[(v-2c)\cdot a_k]a_k + \left[r^2-\sum_{k=1}^n(c\cdot a_k)^2\right]\nvai.
\end{equation*}

\section{Analyzing Geometric Functions}

Yet new geometries can be represented when we take the inner and outer
products of geometric functions.  The major problem, however, is that of
relating one geometric function to another when they are known to represent
the same geometry.  In other words, is there an equivilant to result
$\eqref{rslt_reinterp}$ that applies to geometric functions?  This is an
important question, because the taking of an intersection, for example,
between two geometries becomes useless to us if we cannot decompose the result.
Interesecting two ellipsoids, can we interpret this as the intersection between
a plane and an ellipsoid centered on that plane?  The latter intersection may
have a set of decomposition steps, while the former does not.

\section{References}

Give them here.

\end{document}