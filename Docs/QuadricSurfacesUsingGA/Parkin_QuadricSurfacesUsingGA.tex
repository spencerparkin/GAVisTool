%-----------------------------------------------------------------------
% Beginning of ecgd-l-template.tex
%-----------------------------------------------------------------------
%
%     This is a topmatter template file for ECGD for use with AMS-LaTeX.
%
%     Templates for various common text, math and figure elements are
%     given following the \end{document} line.
%
%%%%%%%%%%%%%%%%%%%%%%%%%%%%%%%%%%%%%%%%%%%%%%%%%%%%%%%%%%%%%%%%%%%%%%%%

%     Remove any commented or uncommented macros you do not use.

\documentclass{ecgd-l}

%     If you need symbols beyond the basic set, uncomment this command.
\usepackage{amssymb}

%     If your article includes graphics, uncomment this command.
\usepackage{graphicx}

%     If the article includes commutative diagrams, ...
%\usepackage[cmtip,all]{xy}


%     Update the information and uncomment if AMS is not the copyright
%     holder.
%\copyrightinfo{2009}{American Mathematical Society}

\newcommand{\G}{\mathbb{G}}
\newcommand{\V}{\mathbb{V}}
\newcommand{\Vb}{\mathbb{\overline{V}}}
\newcommand{\W}{\mathbb{W}}
\newcommand{\R}{\mathbb{R}}

\newtheorem{theorem}{Theorem}[section]
\newtheorem{lemma}[theorem]{Lemma}

\theoremstyle{definition}
\newtheorem{definition}[theorem]{Definition}
\newtheorem{example}[theorem]{Example}
\newtheorem{xca}[theorem]{Exercise}

\theoremstyle{remark}
\newtheorem{remark}[theorem]{Remark}

\numberwithin{equation}{section}

\begin{document}

% \title[short text for running head]{full title}
\title{A Model For Quadric Surfaces\\Using\\Geometric Algebra}

%    Only \author and \address are required; other information is
%    optional.  Remove any unused author tags.

%    author one information
% \author[short version for running head]{name for top of paper}
\author{Spencer T. Parkin}
\address{102 West 500 South, Salt Lake City, Utah  84101}
\curraddr{}
\email{spencer.parkin@disney.com}
\thanks{}

%    author two information
%\author{}
%\address{}
%\curraddr{}
%\email{}
%\thanks{}

% TODO: Figure out how to fill in the subject class.
%    \subjclass is required.
\subjclass[2010]{Primary }

\date{}

\dedicatory{}

%    Abstract is required.
\begin{abstract}
Inspired by the conformal model of geometric algebra,
a similar model of geometry is developed for the set
of all quadric surfaces in $n$-dimensional space.  Bivectors of the geometric algebra
are found to be representative of quadric surfaces.  Coordinate free canonical forms
of such bivectors are found for common quadric surfaces.  The model is investigated
for utility and compared to the conformal model.
\end{abstract}

\maketitle

%    Text of article.

\section{The Construction Of The Model}

The stage for this model of $n$-dimensional quadric surfaces is set in the geometric
algebra we'll denote by $\G$ that is generated by a vector space $\W$ of dimension
$2(n+1)$.  Letting $\{e_i\}_{i=0}^{2n+1}$ be an orthonormal set of basis vectors
generating $\W$, we let $\{e_i\}_{i=0}^n$ be such a set of vectors generating
the $(n+1)$-dimensional vector sub-space $\V$ of $\W$ in which we'll impose the
usual interpretation of $(n+1)$-dimensional homogeneous space.  Specifically,
a vector $v\in\V$ with $v\cdot e_0\neq 0$ represents the point given by\footnote{Throughout this
paper we let the outer product take precedence over the inner product, and the geometric product
take precedence over both the inner and outer products.}
\begin{equation}
e_0\cdot\frac{e_0\wedge v}{e_0\cdot v}
\end{equation}
in $n$-dimensional Euclidean space, imposing the usual correlation between $n$ dimensional
vectors and $n$-dimensional points\footnote{The correlation between
vectors and points spoken of here is that of having a vector represent the point
at its tip when its tail is placed at the origin.}.  We will take the liberty of letting vectors $v\in\V$ with $v\cdot e_0=0$
represent points under the same interpretation of which has been just spoken, as
well as pure directions with magnitude.  The intended interpretation will be made clear
in the context of our usage.  We will refer to all vectors $v\in\V$ with $v\cdot e_0\neq 0$
as projective points, and such vectors with $v\cdot e_0=0$ as non-projective points.

We now introduce a function defined on $\G$ having the outermorphic property.
This means it is a linear function and that it preserves the outer product.  We will
use over-bar notation to denote the use of this function.  Doing so, for any
element $E\in\G$, we define $\overline{E}$ as
\begin{equation}
\overline{E} = RE\tilde{R},
\end{equation}
where the rotor $R$ is given by
\begin{equation}
R = \frac{1}{2^{n/2}}\prod_{i=0}^n\left(1-e_ie_{i+n+1}\right).
\end{equation}
As the reader can check, for any integer $i\in[0,n]$, we have $\overline{e_i}=e_{i+n+1}$.
The rotor $R$ simply rotates any $k$-vector taken from the geometric algebra generated
by $\V$ and rotates it into the identical geometric algebra generated by the vector
space complement to $\V$ with respect to $\W$.  This idea can be found in \cite{DoranHestenes93}.
We will find the over-bar notation convenient when perform algebraic manipulations in our model.

We are now ready to give the definition by which we will interpret bivectors in $\G$
as $n$-dimensional quadric surfaces.
\begin{definition}\label{def_quadric}
For any element $E\in\G$, we say that $E$ is representative of the $n$-dimensional
quadric surface generated by the set of all projective points $v\in\V$ such that
\begin{equation}\label{equ_quadric_condition}
0 = p\wedge\overline{p}\cdot E.
\end{equation}
\end{definition}
Notice that when $\mbox{grade}(E)>1$, there is no ambiguity, despite the non-associativity
of the inner product, in rewriting equation
\eqref{equ_quadric_condition} as
\begin{equation}
0 = p\cdot E\cdot\overline{p},
\end{equation}
which resembles a sort of conjugation of $E$ by $p$.  This may perhaps be a more
familiar form for readers familiar with the study of quadric surfaces in projective geometry.
Also notice that we have not required that $E$ be a bivector in Definition~\ref{def_quadric},
because we may find this condition useful and meaningful for any element of $\G$.  For now,
however, we will restrict our attention to the case when $E$ is a bivector.

To see why Definition~\ref{def_quadric} works, simply notice that when $E$ is a bivector, we have
\begin{equation}\label{equ_homogeneous_polynomial}
p\wedge\overline{p}\cdot E=\sum_{i=0}^n\sum_{j=i}^n \lambda_{ij}(p\cdot e_i)(p\cdot e_j),
\end{equation}
which we can recognize as a homogeneous polynomial of degree 2 in the vector components of $p$.
The scalars $\lambda_{ij}$, with $0\leq i\leq j\leq n$, may be formulated in terms of $E$ by
\begin{equation}\label{equ_quadric_components}
\lambda_{ij} = \left\{\begin{array}{ll}
e_i\overline{e_j}\cdot E & \mbox{if $i=j$,} \\
\left(e_i\overline{e_j}-\overline{e_i}e_j\right)\cdot E & \mbox{if $i\neq j$.}
\end{array}\right.
\end{equation}
It should be noted that bivectors do not uniquely represent quadric surfaces, not even up to scale.
This is apparent from equation \eqref{equ_quadric_components} when we see that for $i\neq j$,
we can freely choose certain components of the bivector without changing the represented
quadric so long as that their sum is still $-\lambda_{ij}$.

An important difference to point out here between this model and the conformal model is that,
unlike what we can analogously expect from the point-definition of the conformal model,
here the 2-blade form $a\wedge\overline{a}$ found in Definition~\ref{def_quadric}, for
any projective point $a\in\V$, does not represent the projective point $a$ under Definition~\ref{def_quadric}.
In homogenized form, the projective point represented by $a\wedge\overline{a}$ is given by
\begin{equation}
e_0 - \left(e_0\cdot\frac{e_0\wedge a}{e_0\cdot a}\right)^{-1},
\end{equation}
which is the reflection about the origin of the spherical reflection of the projective point $a$
about the unit-sphere centered at the origin.  The only point that represents itself
in the form $a\wedge\overline{a}$ appears to be $e_0$.

\section{The Construction Of Quadric Surfaces In The Model}

Having constructed our model, we are now ready to find canonical forms of bivectors
representing a variety of well-known quadric surfaces.  Let us begin with the
spheroid, a special case of ellipsoid.  Such a surface may be characterized by
the non-projective point solution set of the equation
\begin{equation}\label{equ_spheroid}
r^2 = (x-c)^2 - ((x-c)\cdot v)^2
\end{equation}
in the non-projective point $x\in\V$, where $c\in\V$ is a non-projective
point denoting the center of the spheroid, $v\in\V$ is a direction
vector with $0\leq |v|<1$, indicating the direction and amount of bulge in the spheroid, and the scalar $r\in\R$
is the radius of the spheroid about the axis $v$.  To better see that this
is indeed a spheroid, consider the two cases $(x-c)\cdot v=0$ and $(x-c)\wedge v=0$.
In the first case, equation \eqref{equ_spheroid} becomes
\begin{equation}\label{equ_spheroid_smallest}
r^2 = (x-c)^2,
\end{equation}
which is clearly a sphere at $c$ with radius $r$.  In the second case, equation \eqref{equ_spheroid} becomes
\begin{equation}\label{equ_spheroid_largest}
\frac{r^2}{1-v^2} = (x-c)^2,
\end{equation}
which is yet another sphere at $c$, this time having a radius larger than $r$.  In neither
case, we may consider our equation to be an interpolation between equations
\eqref{equ_spheroid_smallest} and \eqref{equ_spheroid_largest}.

Expanding equation \eqref{equ_spheroid}, we get
\begin{equation}
0 = x^2 - (x\cdot v)^2 + 2x\cdot ((c\cdot v)v - c) + c^2 - (c\cdot v)^2 - r^2,
\end{equation}
from which it is possible to factor out $p\wedge\overline{p}$
in terms of the inner product, where $p=e_0+x$
is a homogenized projective point.  Doing so, we see that the bivector
$E$ given by
\begin{equation}\label{equ_spheroid_bivector}
E = -\Omega + v\wedge\overline{v} - 2((c\cdot v)v - c)\wedge\overline{e_0} - (c^2-(c\cdot v)^2-r^2)A,
\end{equation}
is representative of the spheroid by Definition~\ref{def_quadric}, where the constant
$\Omega$ is defined as
\begin{equation}
\Omega=\sum_{i=1}^n e_i\overline{e_i},
\end{equation}
and $A$ is the constant defined as $A=e_0\overline{e_0}$.  We will find each of these useful as
frequently recurring constants in our calculations.

Such forms as that in equation \eqref{equ_spheroid_bivector} are useful, not only
for composition, but especially decomposition in the cases
where we have formulated what may, for example, be a spheroid by some other means.
This gives the model power as an analytical tool.  If we can solve a problem whose solution
is a bivector representative of a spheroid, then we can use this canonical form to answer
questions about that spheroid.  Where is its center?  What is its axis?  What is its radius
about that axis?  As is often the case in mathematics, decomposition is
harder than composition.
Referring to equation \eqref{equ_spheroid_bivector}, we can deduce...give
a decomposition of the spheroid once I know the form is correct.  Consider $E^2$.

%    Bibliographies can be prepared with BibTeX using amsplain,
%    amsalpha, or (for "historical" overviews) natbib style.
\bibliographystyle{amsplain}
%    Insert the bibliography data here.

\bibliography{Parkin_QuadricSurfacesUsingGA}

\end{document}

%%%%%%%%%%%%%%%%%%%%%%%%%%%%%%%%%%%%%%%%%%%%%%%%%%%%%%%%%%%%%%%%%%%%%%%%

%    Templates for common elements of a journal article; for additional
%    information, see the AMS-LaTeX instructions manual, instr-l.pdf,
%    included in the ECGD author package, and the amsthm user's guide,
%    linked from http://www.ams.org/tex/amslatex.html .

%    Section headings
\section{}
\subsection{}

%    Ordinary theorem and proof
\begin{theorem}[Optional addition to theorem head]
% text of theorem
\end{theorem}

\begin{proof}[Optional replacement proof heading]
% text of proof
\end{proof}

%    Figure insertion; default placement is top; if the figure occupies
%    more than 75% of a page, the [p] option should be specified.
\begin{figure}
\includegraphics{filename}
\caption{text of caption}
\label{}
\end{figure}

%    Mathematical displays; for additional information, see the amsmath
%    user's guide, linked from http://www.ams.org/tex/amslatex.html .

% Numbered equation
\begin{equation}
\end{equation}

% Unnumbered equation
\begin{equation*}
\end{equation*}

% Aligned equations
\begin{align}
  &  \\
  &
\end{align}

%-----------------------------------------------------------------------
% End of ecgd-l-template.tex
%-----------------------------------------------------------------------
