\documentclass{birkjour}

\usepackage{amsmath}
\usepackage{amssymb}
\usepackage{amsthm}
\usepackage{graphicx}
\usepackage{float}

\newtheorem{thm}{Theorem}[section]
\newtheorem{cor}[thm]{Corollary}
\newtheorem{lem}[thm]{Lemma}
\newtheorem{prop}[thm]{Proposition}
\theoremstyle{definition}
\newtheorem{defn}[thm]{Definition}
\theoremstyle{remark}
\newtheorem{rem}[thm]{Remark}
\newtheorem*{ex}{Example}
\numberwithin{equation}{section}

\newcommand{\G}{\mathbb{G}}
\newcommand{\V}{\mathbb{V}}
\newcommand{\Vb}{\mathbb{\overline{V}}}
\newcommand{\W}{\mathbb{W}}
\newcommand{\R}{\mathbb{R}}
\newcommand{\Alpha}{A}
\newcommand{\nvao}{o}
\newcommand{\nvai}{\infty}
\newcommand{\nvaob}{\overline{o}}
\newcommand{\nvaib}{\overline{\infty}}
\newcommand{\eminus}{e_{-}}
\newcommand{\eplus}{e_{+}}
\newcommand{\eminusb}{\overline{e}_{-}}
\newcommand{\eplusb}{\overline{e}_{+}}

\begin{document}

\title{An Extension Of The Quadric Model}

\author{Spencer T. Parkin}
\address{%
2113 S. Claremont Dr.\\
Bountiful, Utah  84010\\
USA}
\email{spencer.parkin@gmail.com}

\numberwithin{equation}{section}

\subjclass{Primary 14J70; Secondary 14J29}

\keywords{Quadric Surface, Geometric Algebra, Quadric Model}

%\dedicatory{To Melinda and Naomi}

\begin{abstract}
An extension is found for the model set forth in \cite{Parkin12}
which expands the set of all transformations that can be applied
to quadric surfaces to the set of all conformal transformations.
\end{abstract}

\maketitle

\section{The Expansion Of $\G$ to $\G^*$}

We assume here from the beginning that the reader is familiar with
all definitions and results set forth in \cite{Parkin12} as this paper
will make use of that material without recounting it.
That said, we begin with an extension of $\G$ to the geometric
algebra $\G^*$.  We will let $\G$ be a proper sub-algebra of $\G^*$
by adding the following four basis vectors.
\begin{equation}\label{equ_null_basis_vecs}
\begin{array}{ll}
\nvao & \mbox{The null-vector at the origin.} \\
\nvai & \mbox{The null-vector at infinity.} \\
\nvaob & \mbox{The friend of $\nvao$.} \\
\nvaib & \mbox{The friend of $\nvai$.}
\end{array}
\end{equation}
What was referred to in \cite{Parkin12} as the counter-part of a vector will be referred
to here in this paper as the friend of a vector.  In group theory terms, the vectors
$\nvao$ and $\nvaob$ are conjugates of one another, which is always a mutual
relationship.  Friendship, however, as it will be defined shortly, and as we'll find,
is not always such a relationship.

At the moment, the over-bar notation used in table \eqref{equ_null_basis_vecs}
is nothing more than notation.  In \cite{Parkin12}, the over-bar notation refers to the application
of an outermorphic function.  We will see shortly that we can overload this notation to
also refer to an extension of this outermorphic function.

The following is an inner product table for the basis vectors in table \eqref{equ_null_basis_vecs}.
\begin{equation}
\begin{array}{c|cccc}
\cdot & \nvao & \nvai & \nvaob & \nvaib \\
\hline
\nvao & 0 & -1 & 0 & 0 \\
\nvai & -1 & 0 & 0 & 0 \\
\nvaob & 0 & 0 & 0 & -1 \\
\nvaib & 0 & 0 & -1 & 0
\end{array}
\end{equation}

We will let $\V^*$ contain $\V$ as a proper vector-subspace, adding to it the
basis vectors $\nvao$ and $\nvai$.  Similarly, we will let $\overline{\V}^*$ contain
$\overline{\V}$ as a proper vector-space, adding to it the basis vectors $\overline{\nvao}$
and $\overline{\nvai}$.  We will let $\W^*$ denote the smallest vector space containing
$\V^*$ and $\overline{\V}^*$ as vector sub-spaces.
For all vectors $v\in\W$, (not $v\in\W^*$), we will define $0=v\cdot b$, where
$b$ is any basis vector in table \eqref{equ_null_basis_vecs}.

What we have now with $\G^*$ is simply a geometric algebra containing
two isomorphic Minkownski sub-algebras $\G(\V^*)$ and $\G(\overline{\V}^*)$.
To preserve the use of the over-bar notation in our extended model, we will need
to extend its use an outermorphic function to the new model.  To that end, we will
find it useful to refer to \cite{LiRockwood} in defining the following vectors.
\begin{align}
\eminus &= \frac{1}{2}\nvai + \nvao \\
\eplus &= \frac{1}{2}\nvai - \nvao
\end{align}
As the reader can check, $\eminus$ is a unit-length anti-Euclidean vector, (having an inner-product
square of $-1$), while $\eplus$ is a unit-length Euclidean vector.  We will define
$\eminusb$ and $\eplusb$ similarly with $\nvaob$ and $\nvaib$.  We can now
define, for any element $E\in\G^*$, the friend $\overline{E}$ of $E$ as
\begin{equation}
\overline{E} = S^*E\tilde{S^*},
\end{equation}
where $S^*$ is defined in terms of $S$ as
\begin{equation}
S^* = \frac{1}{2}(1+\eminus\eminusb)(1-\eplus\eplusb)S.
\end{equation}
It now follows that for any vector $v\in\V^*$, the vector $\overline{v}$ is the
friend of $v$ in $\overline{V}^*$.  Similarly, for any vector $v\in\overline{\V}^*$,
the vector $\overline{v}$ as the friend of $v$ in $\V^*$.
We must be careful, however, because it does not now follow, as it did in our
original model, that the friend of $v\in\V^*$ in $\overline{\V}^*$ is also the
friend of $v$, and vice-versa.
This is because $\nvao$ is not the friend
of $\nvaob$, nor is $\nvai$ the friend of $\nvaib$.  Rather, $-\nvao$
is the friend of $\nvaob$, and $-\nvai$ is the friend of $\nvaib$.
In the extended model, friendship is not always a mutual relationship as it
was in the original model.
This will not present a problem for us, however, if we only use the over-bar
function in one direction; from $\V^*$ to $\overline{\V}^*$.

Furthermore, it should be noted here that the over-bar function
is no longer an isomorphism between the two principle halves
of our model; namely, $\G(\V)$ and $\G(\V^*)$, but we never
relied upon this property to begin with, nor will we need it moving forward.
The only important fact is that $\G(\V)$ and $\G(\V^*)$ are indeed isomorphic.
It may actually possible to remove our definition of the over-bar function
in terms of $S^*$, and simply use it as pure notation.

We now introduce the conformal mapping $P:\V\to\V^*$ as
\begin{equation}
P(p) = \nvao + p + \frac{1}{2}p^2\nvai,
\end{equation}
and then realize that for any bivector $E\in\G$ representative of an $n$-dimensional
quadric surface by our original model, that we have
\begin{equation}
P(p)\wedge\overline{P(p)}\cdot E = p\wedge\overline{p}\cdot E
\end{equation}
showing that the bivectors of the form $E$ in \cite{Parkin12} are 
conveniently the very bivectors in our extended model that are also presentative
of $n$-dimensional quadric surfaces.  To see this, it is convenient
to make use of the vectors $\eminus$ and $\eplus$; rewriting the conformal
mapping in terms of them as
\begin{equation}
P(p) = \alpha\eminus + p + \beta\eplus,
\end{equation}
where $\alpha=\frac{1}{2}(p^2+1)$ and $\beta=\frac{1}{2}(p^2-1)$.
Doing so, we see that
\begin{align}
P(p)\wedge\overline{P(p)}
 &= (\alpha\eminus + \beta\eplus)\wedge\overline{(\alpha\eminus + \beta\eplus)}\label{equ_e_with_e} \\
 &+  (\alpha\eminus + \beta\eplus)\wedge\overline{p}\label{equ_e_with_p} \\
 &+ p\wedge\overline{(\alpha\eminus + \beta\eplus)}\label{equ_p_with_e} \\
 &+ p\wedge\overline{p}.\label{equ_p_with_p}
\end{align}
It is now easy to see that $E$, when taken in the inner product with
each of \eqref{equ_e_with_e}, \eqref{equ_e_with_p} and \eqref{equ_p_with_e}, vanishes to zero,
leaving just \eqref{equ_p_with_p}.

\section{Transformations Of The Extended Model}

At this point we have extended the framework of the quadric model to a higher dimensional
algebra $\G^*$ in which all previously known results of $\G$ are preserved.  In this extended framework
we can now discover a larger set of transformations applicable to quadrics as versors.  Indeed, what
we'll now show is that the entire set of conformal transformations are available to us in the extended model.
To see this, we start by making the simple observation that for any versor $V\in\G(\V^*)$, we can
recognize the algebraic variety generated by the set of all projective points $p\in\V$, such that
\begin{equation}\label{equ_quadric_transformed}
0 = V^{-1}P(p)V\wedge\overline{V^{-1}P(p)V}\cdot E,
\end{equation}
as the transformation of the quadric $E\in\G$ by the versor $V$, provided
that $e_0=V^{-1}e_0V$.  Indeed, what we'll find
is that the transformation $E'$ of $E$ by $V$ is given by
\begin{equation}
E' = V\overline{V}E(V\overline{V})^{-1}.
\end{equation}
To see this, let us first write $E$ in the form
\begin{equation}
E = \sum_{i=1}^k a_i\wedge\overline{b_i},
\end{equation}
where each of $\{a_i\}_{i=1}^k$ and $\{b_k\}_{k=1}^i$ is a sequence of $k$ vectors taken from $\V$.
Then, by the linearity of all the products of geometric algebra, there is no loss in generality here
if we, for convenience, consider only the case $k=1$, and write $E$ as simply the 2-blade
\begin{equation}
E = a\wedge\overline{b},
\end{equation}
where $a,b\in\V$.  Having done this, it is easy to establish that the quadric represented by $E'$
is the very quadric represented in equation \eqref{equ_quadric_transformed} by the equality of
\eqref{equ_start} with \eqref{equ_finish}.
\begin{align}
 & V^{-1}P(p)V\wedge\overline{V^{-1}P(p)V}\cdot a\wedge\overline{b}\label{equ_start} \\
=\;& -(V^{-1}P(p)V\cdot a)(V^{-1}P(p)V\cdot b) \\
=\;& (P(p)\cdot VaV^{-1})(P(p)\cdot VbV^{-1}) \\
=\;& P(p)\wedge\overline{P(p)}\cdot VaV^{-1}\wedge\overline{VbV^{-1}}\label{equ_before_versor_invariance} \\
=\;& P(p)\wedge\overline{P(p)}\cdot
V\overline{V}a(V\overline{V})^{-1}\wedge V\overline{Vb}(V\overline{V})^{-1}\label{equ_after_versor_invariance} \\
=\;& P(p)\wedge\overline{P(p)}\cdot V\overline{V}(a\wedge\overline{b})(V\overline{V})^{-1}\label{equ_finish}
\end{align}
Seeing that the step from \eqref{equ_before_versor_invariance} to \eqref{equ_after_versor_invariance} may
require more clarity, let us proceed to justify this step.  To begin, realize
that for any vector $v\in\V$, the conjugation of $\overline{v}$ by $V$ leaves $\overline{v}$ invariant, up
to scale; and likewise, the conjugation of $v$ by $\overline{V}$ leaves $v$ invariant, up to scale.
A change in sign depends upon the parity of the versor $V$, but it doesn't matter,
because only zero or two sign changes, if any, will happen, a cancelation occuring in the latter case.
To see this, realize that
\begin{align}
 & VaV^{-1} \\
=\;& V\overline{VV^{-1}}aV^{-1} \\
=\;& (-1)^mV\overline{V}a\overline{V}^{-1}V^{-1} \\
=\;& (-1)^m V\overline{V}a(V\overline{V})^{-1},\label{equ_a_with_m}
\end{align}
where $m$ is the number of vectors composing $V$ in a geometric product,
while on the other side of the outer product, we have
\begin{align}
 & \overline{VbV^{-1}} \\
=\;& VV^{-1}\overline{VbV^{-1}} \\
=\;& (-1)^{m^2}V\overline{V}V^{-1}\overline{bV^{-1}} \\
=\;& (-1)^{m^2+m}V\overline{Vb}V^{-1}\overline{V}^{-1} \\
=\;& (-1)^{2m^2+m}V\overline{VbV^{-1}}V^{-1} \\
=\;& (-1)^m V\overline{Vb}(V\overline{V})^{-1}.\label{equ_b_with_m}
\end{align}
It is clear now that the $(-1)^m$ in \eqref{equ_a_with_m} will cancel the $(-1)^m$ in \eqref{equ_b_with_m}
as we see each of these appearing in \eqref{equ_after_versor_invariance}.

Of course, the requirement that $V$ keep $e_0$ invariant under versor conjugation is only necessary
if we wish to easily visualize the newly transformed point $V^{-1}P(p)V$ in Euclidean space.
Removing this constraint, a versor $V$ transforms points in homogeneous space, the results
of which are harder to visualize, but which may provide us with the ability to project the quadrics.

\section{Computational Verification}

``For example'' is not a proof\footnote{Of course, ``for example'' may certainly be a disproof, which
is a type of proof; but the meaning of the phrase is still well understood.},
as the Jewish proverb goes.  Nevertheless, making practical use
of the new model for quadric surfaces can be a proof of concept.  Here we give the resulting
output of a computer program, developed by the author, and designed to help users visualize
the elements of a geometric algebra, (such as our $\G(\W^*)$), under certain
interpretations.

\section{Concluding Remarks}

The biggest gap that seems to remain between our extended model and
the conformal model is the lack of conformal operations such as intersecting geometries, fitting geometries
to a set of points, and so on.  It isn't too surprising that that these features do not
naturally present themselves, however, because the quadrics are not closed under
the intersection operation, and there may not be a unique quadric fitting a given set
of points in a certain way.  In any case, the jury is still out on what the
best model for quadrics is, but until a better model comes along, this one appears
to show some promise.

\bibliographystyle{amsplain}
\bibliography{Parkin_AnExtensionOfTheQuadricModel}

\end{document}