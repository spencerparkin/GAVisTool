\documentclass{birkjour}

\usepackage{amsmath}
\usepackage{amssymb}
\usepackage{amsthm}
\usepackage{graphicx}
\usepackage{float}

\newtheorem{thm}{Theorem}[section]
\newtheorem{cor}[thm]{Corollary}
\newtheorem{lem}[thm]{Lemma}
\newtheorem{prop}[thm]{Proposition}
\theoremstyle{definition}
\newtheorem{defn}[thm]{Definition}
\theoremstyle{remark}
\newtheorem{rem}[thm]{Remark}
\newtheorem*{ex}{Example}
\numberwithin{equation}{section}

\newcommand{\G}{\mathbb{G}}
\newcommand{\V}{\mathbb{V}}
\newcommand{\Vb}{\mathbb{\overline{V}}}
\newcommand{\W}{\mathbb{W}}
\newcommand{\R}{\mathbb{R}}
\newcommand{\Alpha}{A}
\newcommand{\nvao}{o}
\newcommand{\nvai}{\infty}
\newcommand{\nvaob}{\overline{o}}
\newcommand{\nvaib}{\overline{\infty}}
\newcommand{\eminus}{e_{-}}
\newcommand{\eplus}{e_{+}}
\newcommand{\eminusb}{\overline{e_{-}}}
\newcommand{\eplusb}{\overline{e_{+}}}

\begin{document}

\title{An Extension Of The Quadric Model}

\author{Spencer T. Parkin}
\address{%
2113 S. Claremont Dr.\\
Bountiful, Utah  84010\\
USA}
\email{spencer.parkin@gmail.com}

\numberwithin{equation}{section}

\subjclass{Primary 14J70; Secondary 14J29}

\keywords{Quadric Surface, Geometric Algebra, Quadric Model}

%\dedicatory{To Melinda and Naomi}

\begin{abstract}
An extension is found for the model set forth in \cite{Parkin12}
which expands the set of all transformations that can be applied
to quadric surfaces to the set of all conformal transformations.
\end{abstract}

\maketitle

\section{The Expansion Of $\G$ to $\G^*$}

We assume here from the beginning that the reader is familiar with
all definitions and results set forth in \cite{Parkin12} as this paper
will make use of that material without recounting it.
That said, we begin with an extension of $\G$ to the geometric
algebra $\G^*$.  We will let $\G$ be a proper sub-algebra of $\G^*$
by adding the following four basis vectors.
\begin{equation}\label{equ_null_basis_vecs}
\begin{array}{ll}
\nvao & \mbox{The null-vector at the origin.} \\
\nvai & \mbox{The null-vector at infinity.} \\
\nvaob & \mbox{The conjugate of $\nvao$.} \\
\nvaib & \mbox{The conjugate of $\nvai$.}
\end{array}
\end{equation}
Recall that the conjugate\footnote{What we have referred to here
as the conjugate of $v$ was referred to in \cite{Parkin12} as the counter-part of $v$.
The term conjugate is more appropriate.}
of any vector $v\in\V$ is the vector $\overline{v}\in\overline{\V}$,
and vice-versa.  At the moment, however, the over-bar notation used in table \eqref{equ_null_basis_vecs}
is nothing more than notation.  In \cite{Parkin12}, the over-bar notation refers to the application
of an outermorphic function.  We will see shortly that we can overload this notation to
also refer to an extension of this outermorphic function.

The following is an inner product table for the basis vectors in table \eqref{equ_null_basis_vecs}.
\begin{equation}
\begin{array}{c|cccc}
\cdot & \nvao & \nvai & \nvaob & \nvaib \\
\hline
\nvao & 0 & -1 & 0 & 0 \\
\nvai & -1 & 0 & 0 & 0 \\
\nvaob & 0 & 0 & 0 & -1 \\
\nvaib & 0 & 0 & -1 & 0
\end{array}
\end{equation}

We will let $\V^*$ contain $\V$ as a proper vector-subspace, adding to it the
basis vectors $\nvao$ and $\nvai$.  Similarly, we will let $\overline{\V}^*$ contain
$\overline{\V}$ as a proper vector-space, adding to it the basis vectors $\overline{\nvao}$
and $\overline{\nvai}$.  We will let $\W^*$ denote the smallest vector space containing
$\V^*$ and $\overline{\V}^*$ as vector sub-spaces.
For all vectors $v\in\W$, (not $v\in\W^*$), we will define $0=v\cdot b$, where
$b$ is any basis vector in table \eqref{equ_null_basis_vecs}.

What we have now with $\G^*$ is simply a geometric algebra containing
two isomorphic Minkownski sub-algebras $\G(\V^*)$ and $\G(\overline{\V}^*)$.
To preserve the use of the over-bar notation in our extended model, we will want to develop it
as an outermorphic isomorphism between these two sub-algebras.  To that end, we will
find it useful to refer to \cite{LiRockwood} in defining the following vectors.
\begin{align}
\eminus &= \frac{1}{2}\nvai + \nvao \\
\eplus &= \frac{1}{2}\nvai - \nvao
\end{align}
As the reader can check, $\eminus$ is a unit-length anti-Euclidean vector, (having an inner-product
square of $-1$), while $\eplus$ is a unit-length Euclidean vector.  We will define
$\eminusb$ and $\eplusb$ similarly with $\nvaob$ and $\nvaib$.  We can now
define, for any element $E\in\G^*$, the element $\overline{E}$ conjugate to $E$ as
\begin{equation}
\overline{E} = S^*E\tilde{S^*},
\end{equation}
where $S^*$ is defined in terms of $S$ as
\begin{equation}
S^* = \frac{1}{2}(1+\eminus\eminusb)(1-\eplus\eplusb)S.
\end{equation}
It now follows that for any vector $v\in\V^*$, the vector $\overline{v}$ is the
conjugate of $v$ in $\overline{V^*}$.

We now introduce the conformal mapping $P:\V\to\V^*$ as
\begin{equation}
P(p) = \nvao + p + \frac{1}{2}p^2\nvai,
\end{equation}
and then realize that for any bivector $E\in\G$ representative of an $n$-dimensional
quadric surface by our original model, we have
\begin{equation}
P(p)\wedge\overline{P(p)}\cdot E = p\wedge\overline{p}\cdot E
\end{equation}
showing that the bivectors of the form $E$ in \cite{Parkin12} are 
conveniently the very bivectors in our extended model that are also presentative
of $n$-dimensional quadric surfaces.  To see this, it is convenient
to make use of the vectors $\eminus$ and $\eplus$; rewriting the conformal
mapping in terms of them as
\begin{equation}
P(p) = \alpha\eminus + p + \beta\eplus,
\end{equation}
where $\alpha=\frac{1}{2}(p^2+1)$ and $\beta=\frac{1}{2}(p^2-1)$.
Doing so, we see that
\begin{align}
P(p)\wedge\overline{P(p)}
 &= (\alpha\eminus + \beta\eplus)\wedge\overline{(\alpha\eminus + \beta\eplus)}\label{equ_e_with_e} \\
 &+  (\alpha\eminus + \beta\eplus)\wedge\overline{p}\label{equ_e_with_p} \\
 &+ p\wedge\overline{(\alpha\eminus + \beta\eplus)}\label{equ_p_with_e} \\
 &+ p\wedge\overline{p}.
\end{align}
It is now easy to see that $E$, when taken in the inner product with
each of \eqref{equ_e_with_e}, \eqref{equ_e_with_p} and \eqref{equ_p_with_e}, vanishes to zero.

\section{Transformations Of The Extended Model}

At this point we have extended the framework of the quadric model to a higher dimensional
algebra $\G^*$ in which all previously known results of $\G$ are preserved.  In this extended framework
we can now discover a larger set of transformations applicable to quadrics as versors.  Indeed, what
we'll now show is that the entire set of conformal transformations are available to us in the extended model.
To see this, we start by making the simple observation that for any versor $V\in\G(\V^*)$, we can
recognize the algebraic variety generated by the set of all projective points $p\in\V$, such that
\begin{equation}\label{equ_quadric_transformed}
0 = V^{-1}P(p)V\wedge\overline{V^{-1}P(p)V}\cdot E,
\end{equation}
as the transformation of the quadric $E\in\G$ by the versor $V$, provided
that $e_0=V^{-1}e_0V$.  Indeed, what we'll find
is that the transformation $E'$ of $E$ by $V$ is given by
\begin{equation}
E' = V\overline{V}E(V\overline{V})^{-1}.
\end{equation}
To see this, let us first write $E$ in the form
\begin{equation}
E = \sum_{i=1}^k a_i\wedge\overline{b_i},
\end{equation}
where each of $\{a_i\}_{i=1}^k$ and $\{b_k\}_{k=1}^i$ is a sequence of $k$ vectors taken from $\V$.
Then, by the linearity of all the products of geometric algebra, there is no loss in generality here
if we, for convenience, consider only the case $k=1$, and write $E$ as simply the 2-blade
\begin{equation}
E = a\wedge\overline{b},
\end{equation}
where $a,b\in\V$.  Having done this, it is easy to establish that the quadric represented by $E'$
is the very quadric represented in equation \eqref{equ_quadric_transformed} by the equality of
\eqref{equ_start} with \eqref{equ_finish}.
\begin{align}
 & V^{-1}P(p)V\wedge\overline{V^{-1}P(p)V}\cdot a\wedge\overline{b}\label{equ_start} \\
=\;& -(V^{-1}P(p)V\cdot a)(V^{-1}P(p)V\cdot b) \\
=\;& (P(p)\cdot VaV^{-1})(P(p)\cdot VbV^{-1}) \\
=\;& P(p)\wedge\overline{P(p)}\cdot VaV^{-1}\wedge\overline{VbV^{-1}}\label{equ_before_versor_invariance} \\
=\;& P(p)\wedge\overline{P(p)}\cdot
V\overline{V}a(V\overline{V})^{-1}\wedge V\overline{Vb}(V\overline{V})^{-1}\label{equ_after_versor_invariance} \\
=\;& P(p)\wedge\overline{P(p)}\cdot V\overline{V}(a\wedge\overline{b})(V\overline{V})^{-1}\label{equ_finish}
\end{align}

To see the step from \eqref{equ_before_versor_invariance} to \eqref{equ_after_versor_invariance},
realize that for any vector $v\in\V$, the conjugation of $\overline{v}$ by $V$ leaves $\overline{v}$ invariant, up
to scale; and likewise, the conjugation of $v$ by $\overline{V}$ leaves $v$ invariant, up to scale.
A change in sign depends upon the parity of the versor $V$, but it doesn't matter,
because only zero or two sign changes, if any, will happen, a cancelation occuring in the latter case.

Of course, the requirement that $V$ keep $e_0$ invariant under versor conjugation is only necessary
if we wish to easily visualize the newly transformed point $V^{-1}P(p)V$ in Euclidean space.
Removing this constraint, a versor $V$ transforms points in homogeneous space, the results
of which are harder to visualize, but which may provide us with the ability to project the quadrics.

\section{Concluding Remarks}

The biggest gap that seems to remain between our extended model and
the conformal model is the lack of conformal operations such as intersecting geometries, fitting geometries
to a set of points, and so on.  It isn't too surprising that that these features do not
naturally present themselves, however, because the quadrics are not closed under
the intersection operation, and there may not be a unique quadric fitting a given set
of points in a certain way.  In any case, the jury is still out on what the
best model for quadrics is, but until a better model comes along, this one appears
to show some promise.

\bibliographystyle{amsplain}
\bibliography{Parkin_AnExtensionOfTheQuadricModel}

\end{document}