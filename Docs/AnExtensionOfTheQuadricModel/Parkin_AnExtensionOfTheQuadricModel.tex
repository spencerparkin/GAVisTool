\documentclass{birkjour}

\usepackage{amsmath}
\usepackage{amssymb}
\usepackage{amsthm}
\usepackage{graphicx}
\usepackage{float}

\newtheorem{thm}{Theorem}[section]
 \newtheorem{cor}[thm]{Corollary}
 \newtheorem{lem}[thm]{Lemma}
 \newtheorem{prop}[thm]{Proposition}
 \theoremstyle{definition}
 \newtheorem{defn}[thm]{Definition}
 \theoremstyle{remark}
 \newtheorem{rem}[thm]{Remark}
 \newtheorem*{ex}{Example}
 \numberwithin{equation}{section}

\newcommand{\G}{\mathbb{G}}
\newcommand{\V}{\mathbb{V}}
\newcommand{\Vb}{\mathbb{\overline{V}}}
\newcommand{\W}{\mathbb{W}}
\newcommand{\R}{\mathbb{R}}

\newcommand{\Alpha}{A}
%\Omega already defined

\newcommand{\nvao}{o}
\newcommand{\nvai}{\infty}
\newcommand{\nvaob}{\overline{o}}
\newcommand{\nvaib}{\overline{\infty}}

\newcommand{\eminus}{e_{-}}
\newcommand{\eplus}{e_{+}}
\newcommand{\eminusb}{\overline{e}_{-}}
\newcommand{\eplusb}{\overline{e}_{+}}

\begin{document}

\title{An Extension Of The Quadric Model Of Geometric Algebra}

\author{Spencer T. Parkin}
\address{%
2113 S. Claremont Dr.\\
Bountiful, Utah  84010\\
USA}
\email{spencer.parkin@gmail.com}

\numberwithin{equation}{section}

\subjclass{Primary 14J70; Secondary 14J29}

\keywords{Quadric Surface, Geometric Algebra, Quadric Model}

%\dedicatory{To Melinda and Naomi}

\begin{abstract}
What can be seen as an extension of the model set forth
in \cite{Parkin12}, a new model of $n$-dimensional quadric surfaces is
found in which all conformal transformations are supported in the
form of versors.
\end{abstract}

\maketitle

\section{The Geometric Algebra}

An explanation of any model of geometry based upon geometric algebra must
begin with a description of the structure of the geometric algebra upon
which the model is imposed.  The model set forth in this paper is set in a
geometric algebra containing the following nested vector spaces.
\begin{equation}\label{equ_vector_spaces}
\begin{array}{ll}
\mbox{Notation} & \mbox{Basis} \\
\hline
\V^e & \{e_i\}_{i=1}^n \\
\V^o & \{\nvao\}\cup\{e_i\}_{i=1}^n \\
\V & \{\nvao\}\cup\{e_i\}_{i=1}^n\cup\{\nvai\}
\end{array}
\end{equation}
The set of vectors $\{e_i\}_{i=1}^n$ forms an orthonormal set of basis
vectors for the $n$-dimensional Euclidean vector space $\V^e$, which we'll
use to represent $n$-dimensional Euclidean space.
The vectors $\nvao$ and $\nvai$ are the familiar null-vectors at origin and infinity
taken from the conformal model of geometric algebra.  An inner-product
table for these basis vectors is given as follows.
\begin{equation}
\begin{array}{c|ccc}
\cdot & \nvao & e_i & \nvai \\
\hline
\nvao & 0 & 0 & -1 \\
e_i & 0 & 1 & 0 \\
\nvai & -1 & 0 & 0
\end{array}
\end{equation}
We will now let $\G(\V)$ denote the Minkowski geometric algebra generated by $\V$.
For each vectors space in table \eqref{equ_vector_spaces}, we will let an over-bar
above this vector space denote an identical copy of that vector space.  The vector
space $\W$ will denote the smallest vector space containing each $\V$ and $\Vb$
as vector subspaces.  In symbols, one may write
\begin{equation}
\G(\W) = \G(\V)\oplus\G(\Vb)
\end{equation}
to illustrate the structure of $\G(\W)$ in terms of its two isomorphic Minkowski
geometric sub-algebras $\G(\V)$ and $\G(\Vb)$.

We will use over-bar notation to distinguish between vectors taken from $\V$
with vectors taken from $\Vb$.  Though not necessary, we can work exclusively
in $\G(\V)$ by defining the over-bar notation as an outermorphic ismorphism between
$\G(\V)$ and $\G(\Vb)$.  Doing so, we see that for any element $E\in\G(\V)$,
we may define $\overline{E}\in\G(\Vb)$ as
\begin{equation}
\overline{E} = SE\tilde{S},
\end{equation}
where $S$ is the versor given by
\begin{equation}\label{equ_isomorphism_versor}
S = 2^{-n/2}(1+\eminus\eminusb)(1-\eplus\eplusb)\prod_{i=0}^n(1-e_i\overline{e}_i).
\end{equation}
This definition is non-circular if we let the over-bars in equation \eqref{equ_isomorphism_versor}
be purely notation.  The vectors $\eminus$ and $\eplus$, taken from \cite{LiRockwood},
are defined as
\begin{align}
\eminus &= \frac{1}{2}\nvai + \nvao \\
\eplus &= \frac{1}{2}\nvai - \nvao.
\end{align}
The vectors $\eminusb$ and $\eplusb$ are defined similarly in terms of $\nvaob$ and $\nvaib$.
Defined this way, it is important to realize that, unlike the over-bar function defined in \cite{Parkin12},
here we do not have the property that for any vector $w\in\W$, we have $\overline{\overline{w}}=w$.
This is because $\overline{\nvaob}=-\nvao$ and $\overline{\nvaib}=-\nvai$.

\section{The Form Of Quadric Surfaces In $\G(\W)$}

We now give a formal definition under which elements $E\in\G(\W)$
are representative of $n$-dimensional quadric surfaces.
\begin{defn}\label{def_quadric}
Referring to an element $E\in\G(\W)$ as a quadric surface, it is representative of such an $n$-dimensional
surface as the set of all points $p\in\V^o$ such that
\begin{equation}\label{equ_quadric_equation}
0 = p\wedge\overline{p}\cdot E.
\end{equation}
\end{defn}
From this definition it can be seen that the general form of a quadric $E\in\G(\W)$ is given by
\begin{equation}\label{equ_quadric_form}
E = \sum_{i=1}^n\sum_{j=1}^n\lambda_{ij}e_i\overline{e}_j+
\sum_{i=1}^n\lambda_i(e_i\nvaib + \nvai\overline{e}_j)+
\lambda\nvai\nvaib.
\end{equation}
This is because an element of the form \eqref{equ_quadric_form}, when
substituted into equation \eqref{equ_quadric_equation}, produces a polynomial
equation of degree 2 in the vector components of $p-\nvao$.
Doing so with $p=\nvao+x$, where $x\in\V^e$, we get equation
\begin{equation}
0 = -\sum_{i=1}^n\sum_{j=1}^n\lambda_{ij}(x\cdot e_i)(x\cdot e_j)+
\sum_{i=1}^n 2\lambda_i(x\cdot e_i) - \lambda,
\end{equation}
which we may recognize as the equation for an $n$-dimensional quadric surface.
Of course, using geometric algebra, it is undesirable and unecessary to think of
quadrics in terms of polynomial equations.  A better way to think of quadrics is in terms
of an element of the algebra whose decomposition
produces the parameters characterizing the quadric surface.  For example, many common
quadrics are the solution set in $\V^e$ of the equation
\begin{equation}
0 = -r^2 + (x-c)^2 + \lambda((x-c)\cdot v)^2,
\end{equation}
in the variable $x$.  (An explanation of the parameters $r$, $c$, $v$ and $\lambda$
was given in \cite{Parkin12}.)  Then, factoring out $-p\wedge\overline{p}$, we see that
the element $E\in\G(\W)$, given by
\begin{equation}
\Omega + \lambda v\overline{v}+2(c+\lambda(c\cdot v)v)\nvaib+
(c^2+\lambda (c\cdot v)^2-r^2)\nvai\nvaib
\end{equation}
is representative of this very same quadric by Definition~\ref{def_quadric},
where $\Omega$ is defined as
\begin{equation}
\Omega = \sum_{i=1}^n e_i\overline{e}_i.
\end{equation}
Give table of canonical forms here...

\section{Transformations Of The Extended Quadric Model}

Here we begin with the following lemma to be used in the main result
of this paper, which shows that all conformal transformations exist
as versors applicable to quadrics in the extended model.
\begin{lem}\label{lma_versor_transfer}
For any versor $V\in\G(\W)$, and any four vectors $a,b,c,d\in\V$, we have
\begin{equation}
V^{-1}aV\wedge\overline{V^{-1}bV}\cdot c\wedge\overline{d} =
a\wedge\overline{b}\cdot V\overline{V}(c\wedge\overline{d})(V\overline{V})^{-1}.
\end{equation}
\end{lem}
\begin{proof}
Give proof here...
\end{proof}
The main result, as promised, can now be given as follows.
\begin{thm}
Let $E\in\G(\W)$ be a bivector of the form \eqref{equ_quadric_form}.
Then, for any versor $V\in\G(\V)$ of the conformal model, the element $E'$, given by
\begin{equation}
E' = V\overline{V}E(V\overline{V})^{-1},
\end{equation}
is representative of the $n$-dimensional quadric surface by Definition~\ref{def_quadric}
that is the transformation of $E$ by $V$.
\end{thm}
\begin{proof}
Let $P:\V^e\to\V$ be the conformal mapping defined by
\begin{equation}
P(v) = \nvao + v + \frac{1}{2}v^2\nvai.
\end{equation}
We then make the observation that if $p'\in\V^o$ is representative of
the point that is the transformation of $p\in\V^o$ by the transformation
of $V^{-1}$, then the transformation of $E$ by $V$ is described as the set of
all points $p\in\V^o$ such that $0=p'\wedge\overline{p}'\cdot E$.
We then see that
\begin{align}
 & \frac{p'\wedge\overline{p}'}{(\nvai\cdot p')^2}\cdot E \\
=\;& P\left(\frac{p'}{-\nvai\cdot p'}-\nvao\right)
\wedge\overline{P}\left(\frac{p'}{-\nvai\cdot p'}-\nvao\right)\cdot E \\
=\;& V^{-1}P\left(\frac{p}{-\nvai\cdot p}-\nvao\right)V
\wedge\overline{V^{-1}P}\left(\frac{p}{-\nvai\cdot p}-\nvao\right)\overline{V}\cdot E\label{equ_before_step} \\
=\;& P\left(\frac{p}{-\nvai\cdot p}-\nvao\right)\wedge
\overline{P}\left(\frac{p}{-\nvai\cdot p}-\nvao\right)\cdot E'\label{equ_after_step} \\
=\;& \frac{p\wedge\overline{p}}{(\nvai\cdot p)^2}\cdot E',
\end{align}
showing that $E'$ is representative of this very same set of
points under Definition~\ref{def_quadric}.  Lemma~\ref{lma_versor_transfer}
was employed in the step taken from \eqref{equ_before_step} to \eqref{equ_after_step}.
\end{proof}

\section{Computational Verification}

In this section we present the result of implementing the above
model with a computer program.

\bibliographystyle{amsplain}
\bibliography{Parkin_AnExtensionOfTheQuadricModel}

\end{document}