\documentclass{beamer}
\mode<presentation>
\usetheme{Warsaw}
\usecolortheme{lily}

\title{An Intro to CGA}
\subtitle{Conformal Geometric Algebra}
\author{Spencer T. Parkin}
\institute{Avalanche Software}

\newcommand{\G}{\mathbb{G}}
\newcommand{\V}{\mathbb{V}}
\newcommand{\R}{\mathbb{R}}
\newcommand{\B}{\mathbb{B}}
\newcommand{\nvao}{o}
\newcommand{\nvai}{\infty}
\newcommand{\grade}{\mbox{grade}}

\begin{document}

\frame{\titlepage}

\begin{frame}
\frametitle{Presentation Outline}
In this presentation, we will...
\begin{itemize}
\item Introduce concepts from GA only as necessary,
\item Introduce the generalized homogeneous model of geometry over GA,
\item Define the specific conformal model of GA,
\item Find the forms for all geometric primitives of the CGA model,
\item Discuss the fundamental transformations of CGA.
\end{itemize}
\end{frame}

\begin{frame}
\frametitle{Blades}
Let $\V^n$ denote an $n$-dimensional vector space.

Let $\{b_k\}_{k=1}^m$ be a set of $m$ vectors taken from $\V^n$.
\begin{definition}
We say the blade $B$, given by
\begin{equation*}
B = \bigwedge_{k=1}^m b_k = b_1\wedge\dots\wedge b_m,
\end{equation*}
is a non-zero $m$-blade if and only if $\{b_k\}_{k=1}^m$
is a linearly independent set of vectors.
\end{definition}
\end{frame}
%====================================
%====================================

\begin{frame}
\frametitle{Visualizing Euclidean Blades}
Imagine an \alert{infinite} $m$-dimensional hyper-plane.

Think of $B$ as a \alert{finite} $m$-dimensional hyper-plane.

\alert{Non-Euclidean} blades require more imagination!

Our geometric arguments will not require us to visualize the homogeneous representation space.
\end{frame}
%====================================
% The word blade suggests something flat.
%====================================

\begin{frame}
\frametitle{Building Intuition About Euclidean Blades}
Let $v_{\parallel}$ denote the orthogonal \alert{projection} of $v$ down onto $B$.

Let $v_{\perp}=v-v_{\parallel}$ denote the orthogonal \alert{rejection} of $v$ from $B$.

For any vector $v\in\V^n$, we have
\begin{align*}
v\wedge B &= (v_{\parallel} + v_{\perp})\wedge B = v_{\perp}\wedge B, \\
v\cdot B &= (v_{\parallel} + v_{\perp})\cdot B = v_{\parallel}\cdot B.
\end{align*}
\begin{align*}
\grade(v\wedge B) &= \grade(B) + 1 \\
\grade(v\cdot B) &= \grade(B) - 1
\end{align*}
\end{frame}
%====================================
% Explain the orthogonal projection and rejection.
% The outer product builds up dimension, while the inner product removes it.
%====================================

\begin{frame}
\frametitle{Blades May Represent Vector Sub-Spaces}
Recall that $B = b_1\wedge\dots\wedge b_m$.
\begin{definition}
For any $v\in\V^n$, we say that
\begin{equation*}
\mbox{$v\in B$ if and only if $v\in\mbox{span}\{b_k\}_{k=1}^m$}.
\end{equation*}
\end{definition}
\begin{definition}
If $v\not\in B$, then $v\in B^*$, which represents the complement $(\V^n-\mbox{span}\{b_k\}_{k=1}^m)\cup\{0\}$.
\end{definition}
\end{frame}
%====================================
% No need to use unit psuedo-scalar in this presentation; we will use this notation.
%====================================

\begin{frame}
\frametitle{Membership in Vector Spaces and Dual Vector Spaces}
If $B\neq 0$, then $v\in B$ if and only if $v\wedge B=0$.
\begin{proof}
The set $\{b_k\}_{k=1}^m$ is linearly independent while the set $\{v\}\cup\{b_k\}_{k=1}^m$ is linearly dependent.
\end{proof}
If $B\neq 0$, then $v\in B^*$ if and only if $v\cdot B=0$.
\begin{proof}
Notice that $0=v\cdot B=(v\wedge B^*)^*$ if and only if $v\wedge B^*=0$.
\end{proof}
\end{frame}
%====================================
%====================================

% when the time comes, use V^{n+2} to denote the rep space
\begin{frame}
\frametitle{Blades May Represent Geometries}
Let $\R^n$ denote $n$-dimensional Euclidean space.
Let $p:\R^n\to\G(\V^n)$ be a vector-valued function of a Euclidean point.
\begin{definition}
We say that $B$ \alert{directly} represents a geometry as the
set of all points
\begin{equation*}
G(B) = \{x\in\R^n|p(x)\in B\}.
\end{equation*}
\end{definition}
\begin{definition}
We say that $B$ \alert{dually} represents a geometry as the
set of all points
\begin{equation*}
G^*(B) = \{x\in\R^n|p(x)\in B^*\}.
\end{equation*}
\end{definition}
\end{frame}

\begin{frame}
\frametitle{We Can Combine Geometries}
For any two blades $A,B\in\G(\V^n)$ such that $A\wedge B\neq 0$, we have
\begin{equation*}
G(A)\cup G(B)\subseteq G(A\wedge B).
\end{equation*}
\begin{proof}
\begin{align*}
 & \mbox{$p(x)\in A$ or $p(x)\in B$} \\
\implies & \mbox{$p(x)\in A\wedge B$}
\end{align*}
\end{proof}
Let $C\subseteq A\wedge B$ represent the smallest vector sub-space such that $p(x)\in C$.
Then we might have $C\not\subseteq A$ and $C\not\subseteq B$.
\end{frame}
%====================================
% The outer product of two blades, if non-zero, gives us at least
% direct representation of the geometry that is the union of the
% geometries directly represented by the blades taken in that product.
%====================================

\begin{frame}
\frametitle{We Can Intersect Geometries}
\begin{lemma}
For any two blades $A,B\in\G(\V^n)$ such that $A\wedge B\neq 0$,
we have
\begin{equation*}
G^*(A)\cap G^*(B)=G^*(A\wedge B).
\end{equation*}
\end{lemma}
\begin{proof}
\begin{align*}
 & \mbox{$p(x)\in A^*$ and $p(x)\in B^*$} \\
\mbox{iff}\;\; & \mbox{$p(x)\not\in A$ and $p(x)\not\in B$} \\
\mbox{iff}\;\; & \mbox{$p(x)\not\in A\wedge B$} \\
\mbox{iff}\;\; & \mbox{$p(x)\in(A\wedge B)^*$}
\end{align*}
\end{proof}
\end{frame}
%====================================
% The outer product of two blades, if non-zero, gives us the
% dual representation of the geometry that is the intersection of the
% geometries dually represented by the blades taken in that product.
%====================================

\begin{frame}
\frametitle{The Homogeneous Nature Of The Model}
For any non-zero scalar $\lambda$, we have $G(B)=G(\lambda B)$.

For any blade $B$, there is a scalar $\lambda$ such that $\lambda B$ is a homogenized form.

If $B$ is the result of some geometric operations, then such a $\lambda$ has geometric signficance
WRT to that operation.
\end{frame}

\begin{frame}
\frametitle{The Geometric Product}
\begin{definition}
For any vector $v\in\V^n$ and any blade $B\in\G(\V^n)$, we define
\begin{equation*}
vB = v\cdot B + v\wedge B.
\end{equation*}
\end{definition}
\end{frame}
%====================================
% Something should be mentioned about how fundamental
% this product is to GA.
%====================================

\begin{frame}
\frametitle{Versors}
Let $\{v_k\}_{k=1}^m$ be any set of $m$ vectors.
\begin{definition}
We say the element $V\in\G(\V^n)$, given by
\begin{equation*}
V = \prod_{k=1}^m v_k,
\end{equation*}
is a versor if and only if for all $k$, the vector $v_k^{-1}$ exists.
\end{definition}
\end{frame}
%====================================
% Versors are invertible by definition.
%====================================

\begin{frame}
\frametitle{The Inverse And The Reverse Of Versors}
\begin{definition}
Given the versor $V=v_1\dots v_m$, we define
\begin{equation*}
\tilde{V} = \prod_{k=1}^m v_{m-k+1}.
\end{equation*}
\end{definition}
The inverse $V^{-1}$ of $V$ is therefore given by
\begin{equation*}
V^{-1} = \frac{\tilde{V}}{V\tilde{V}}.
\end{equation*}
\end{frame}

\begin{frame}
\frametitle{The Versor Group}
Versors form a group under the geometric product.
\begin{proof}
\alert{Associativity} follows from the associativity of the geometric product.

The scalar $1$ is the \alert{identity} versor.

For every versor $V$, there exists an \alert{inverse} $V^{-1}$ such that $VV^{-1}=1$.
\end{proof}
\end{frame}

\begin{frame}
\frametitle{Properties Of Versors}
Conjugation by versors is \alert{outermorphic}!

Recall that $B=b_1\wedge\dots\wedge b_m$.
We then have
\begin{equation*}
VBV^{-1} = \bigwedge_{k=1}^m Vb_kV^{-1}.
\end{equation*}

Conjugation by versors is \alert{grade preserving}!

For any vector $v\in\V^n$, we have $VvV^{-1}\in\V^n$,
therefore, we have $\grade(B)=\grade(VBV^{-1})$.

\end{frame}
%====================================
% The proof isn't too hard, but much to long for a slide,
% and would detract from the rest of this presentation.
%====================================

\begin{frame}
\frametitle{Versors May Represent Transformations}
It follows that versors may be used to represent transformations
of geometry as versors conjugated with blades representative of geometry.
\end{frame}

\begin{frame}
\frametitle{The Specifics Of The Conformal Model}
Replace $\R^n$ with $\V^n$.

Embed $\V^n$ in $\V^{n+2}$ as a Euclidean vector sub-space.

Let $\nvao,\nvai\in\V^{n+2}$ be vectors such that $\nvao\cdot\nvao=\nvai\cdot\nvai=0$
and $\nvao\cdot\nvai=\nvai\cdot\nvao=-1$ and for all $v\in\V^n$, we have $v\cdot\nvao=v\cdot\nvai=0$.

\begin{definition}
Define $p:\V^n\to\G(\V^{n+2})$ as
\begin{equation*}
p(x) = \nvao + x + \frac{1}{2}x^2\nvai.
\end{equation*}
\end{definition}
Having \alert{invented} this specific model, what we are now able to \alert{discover} about it is almost endless!
\end{frame}
%====================================
% There is motivation behind this definition, but we will
% skip it so as not to detract from proof of the more general
% usage of the model...that's not put right.
%====================================

%\begin{frame}
% Give n-d spheres and (n-1)-d planes here.
%\end{frame}

% With p(x) defined as it is, we get an important lemma...

% compare outermorphic property of versors to linear transformations
% being determined by how they transform basis.  versors are determined
% in CGA by how they transform points.

\end{document}