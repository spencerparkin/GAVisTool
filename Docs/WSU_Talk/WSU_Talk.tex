\documentclass{beamer}
\mode<presentation>
\usetheme{Warsaw}
\usecolortheme{lily}

\title{An Intro to CGA}
\subtitle{Conformal Geometric Algebra}
\author{Spencer T. Parkin}
\institute{Avalanche Software}

\newcommand{\G}{\mathbb{G}}
\newcommand{\V}{\mathbb{V}}
\newcommand{\R}{\mathbb{R}}
\newcommand{\B}{\mathbb{B}}
\newcommand{\nvao}{o}
\newcommand{\nvai}{\infty}

\begin{document}

\frame{\titlepage}

%\begin{frame}
%\frametitle{Outline}
%\tableofcontents
%\end{frame}

% Throughout the talk, it is important to question the applicability
% of a result based upon whether the GA is Euclidean or not.

%\section{Understanding Geometric Algebra}

\begin{frame}
\frametitle{The Outer Product}
Let $\V^n$ be a vector space of dimension $n$.
\pause
\begin{definition}
The product $\bigwedge_{k=1}^m v_k$ is what we call a non-zero $m$-blade if and only if
$\{v_k\}_{k=1}^m\subset\V^n$ is a linearly independent set.
\end{definition}
\pause
\begin{definition}
Letting $\{w_k\}_{k=1}^m$ be the Gram-Schmidt orthonormalization of $\{v_k\}_{k=1}^m$,
we have
\begin{equation*}
\bigwedge_{k=1}^m v_k = \det\left[
\begin{array}{ccc}
v_1\cdot e_1 & \dots & v_1\cdot e_m \\
\vdots & \ddots & \vdots \\
v_m\cdot e_1 & \dots & v_m\cdot e_m
\end{array}
\right]\bigwedge_{k=1}^m w_k,
\end{equation*}
where $\{e_k\}_{k=1}^m$ is any orthonormal basis for the $m$-dimensional
vector sub-space represented by this blade.
\end{definition}
\end{frame}
%============================================================
% Notes for slide:
% Clearly we must have m <= n.
% The outer product is associative and anti-commutative and left/right distributive over addition.
% An (m+1)-dimentional person sees an m-dimentional blade as flat.
% Blade factorizations are not unique!
%============================================================

\begin{frame}
\frametitle{Visualizing Euclidean Blades}
An $m$-blade is characterized by...\pause
\begin{itemize}
\item Its \alert{attitude},\pause
\item Its $m$-dimensional \alert{hyper-volume},\pause
\item Its \alert{handedness}.
\end{itemize}
\end{frame}
%============================================================
% Notes for slide:
% Handedness is a characteristic manifested in the comparison of two blades, or the sign of the determinant.
% The absolute value of the determinant is an $m$-dimensional hyper-volume.
% Non-Euclidean blades are outer products involving non-Euclidean vectors.  Can we visualize them?
%============================================================

\begin{frame}
\frametitle{Generating all Elements of a Geometric Algebra}
Let $\{e_k\}_{k=1}^n$ be any set of basis vectors for $\V^n$.\pause

The general element (multivector) is then any linear combination of (\alert{in red})...\pause
\begin{itemize}
\item $\binom{n}{0}$ ways to choose \alert{1} as a 0-blade,\pause
\item $\binom{n}{1}$ ways to choose \alert{$e_i$} as a 1-blade,\pause
\item $\binom{n}{2}$ ways to choose \alert{$e_i\wedge e_j$} as a 2-blade,\pause
\item ...and so on...\pause
\item $\binom{n}{n}$ ways to choose \alert{$I$} as an $n$-blade.\pause
\end{itemize}
The geometric algebra $\G(\V^n)$ is of dimension $2^n$.
\end{frame}
%============================================================
% Notes for slide:
% Here, "I" is the unit psuedo-scalar.
%============================================================

\begin{frame}
\frametitle{Adding Blades Together}
\begin{definition}
For any $E\in\G(\V^n)$, we let $\langle E\rangle_k$ denote the grade $k$ part of $E$,
and so we may write $E = \sum_{k=0}^n\langle E\rangle_k$.
\end{definition}\pause
The element $\langle E\rangle_k$ is called a $k$-vector.\pause

All $k$-blades are $k$-vectors, but not all $k$-vectors are $k$-blades!\pause
\begin{example}
The following 2-vector cannot be written as a 2-blade.
\begin{equation*}
e_1\wedge e_2 + e_3\wedge e_4
\end{equation*}
\end{example}
\end{frame}
%============================================================
% Notes for slide:
% The first example comes only when we go above 3-dimensinal space.
%============================================================

% talk about join and meet after talking about the inner product

\begin{frame}
\frametitle{The Inner Product}
\begin{definition}
In a Euclidean geometric algebra, we define for all integers $i$ and $j$,
\begin{equation*}
e_i\cdot e_j = \delta_{ij},
\end{equation*}
where here, $\delta_{ij}$ is the Kronecker delta.
\end{definition}
\begin{definition}
If for any vector $v\in\V^n$, we have $v\cdot v=0$, we call $v$ a
null vector.
\end{definition}
\end{frame}
%============================================================
% Notes for slide:
% How we define the inner product of basis vectors is the signature of our geometric algebra.
%============================================================

\begin{frame}
\frametitle{The Inner Product (Continued)}
\begin{definition}
For any vector $v\in\V^n$ and any $m$-blade $B\in\G(\V^n)$, we define
\begin{equation*}
v\cdot B = -\sum_{i=1}^m (-1)^i (v\cdot b_i)\bigwedge_{j=1,j\neq i}^m b_j,
\end{equation*}
where $B = \bigwedge_{k=1}^m b_k$.  We also define
\begin{equation*}
B\cdot v = -(-1)^n v\cdot B.
\end{equation*}
\end{definition}
\end{frame}
%============================================================
% Notes for slide:
% Notice that the inner product is grade lowering while the outer product is grade raising.
% There is a recursive definition which looks nicer, but it is worth memorizing this formula.
%============================================================

\begin{frame}
\frametitle{The Inner Product (Continued)}
\begin{example}
Consider $v\cdot B$, where $B$ is a 2-blade.  WLOG, choose $a,b\in\V^n$ such that
$B=a\wedge b$, $a\cdot b=0$, $|b|=1$ and $v\cdot b=0$.  We then have
\begin{equation*}
v\cdot B = (v\cdot a)b - (v\cdot b)a = |B|\frac{v\cdot a}{|a|}b.
\end{equation*}
\end{example}
\end{frame}
%============================================================
% Notes for slide:
% Here, the result is the part of B with the dimension of the orthogonal projection of v down onto it removed.
% With the outer product, we're adding dimension.  With the inner product, we're removing them.
%============================================================

\begin{frame}
\frametitle{The Inner Product (Continued)}
\begin{definition}
For any two blades $A,B\in\G(\V^n)$ of grades $i$ and $j$, respectively, we define
\begin{equation*}
A\cdot B = \left\{
\begin{array}{ll}
a_1\cdot\dots\cdot a_i\cdot B & \mbox{if $i\leq j$, (R to L assoc.)} \\
A\cdot b_1\cdot\dots\cdot b_j & \mbox{if $i\geq j$, (L to R assoc.)}
\end{array}
\right.
\end{equation*}
where $A=\bigwedge_{k=1}^i a_k$ and $B=\bigwedge_{k=1}^j b_j$.
\end{definition}
\end{frame}
%============================================================
% Notes for slide:
% We also define the inner product as left and right distributive over addition.
%============================================================

\begin{frame}
\frametitle{The Geometric Product}
\begin{definition}
For any vector $v\in\V^n$ and any $m$-blade $B\in\G(\V^n)$, we define
\begin{equation*}
vB = v\cdot B + v\wedge B,
\end{equation*}
and similarly, $Bv = B\cdot v + B\wedge v$.
\end{definition}
\begin{example}
For any two vectors $a,b\in\V^n$, we have
\begin{equation*}
ab = a\cdot b + a\wedge b = |a||b|\cos\theta + B|a||b|\sin\theta = |a||b|\exp(\theta B),
\end{equation*}
where $B = \frac{a\wedge b}{|a\wedge b|}$.
\end{example}
\end{frame}
%============================================================
% Notes for slide:
% The geometric product is an invertible product!
% It can be shown that the square of a unit 2-blade is -1.
%============================================================

\begin{frame}
\frametitle{The Geometric Product (Continued)}
\begin{example}
It can be shown that
\begin{equation*}
v\cdot B = \frac{1}{2}(vB - (-1)^m Bv),
\end{equation*}
and
\begin{equation*}
v\wedge B = \frac{1}{2}(vB + (-1)^m Bv).
\end{equation*}
\end{example}
\end{frame}

% Show the inverse of a blade here WRT the GP.
%\begin{frame}
%\frametitle{The Geometric Product (Continued)}
%\begin{lemma}
%
%\end{lemma}
%\end{frame}

\begin{frame}
\frametitle{The Geometric Product (Continued)}
\begin{definition}
For any set of $m$ vectors $\{v_k\}_{k=1}^m\subset\V^n$, a product
\begin{equation*}
\prod_{k=1}^m v_k
\end{equation*}
is called a versor if for all integers $k$, $v_k^{-1}$ exists.
\end{definition}
If there exists an integer $k$ such that $v_k^{-1}$ does not exist, I call it a pseudo versor.
\begin{lemma}
In a \alert{Euclidean} geometric algebra, any blade can be written as a versor
by the Gram-Schmidt orthogonalization process.
\end{lemma}
\end{frame}

\begin{frame}
\frametitle{The Geometric Product (Continued)}
\begin{example}
For any two multivectors $A,B\in\G(\V^n)$, we have
\begin{equation*}
AB = f^{-1}(f(A)f(B)),
\end{equation*}
where $f$ is a function mapping a multivector to its multi-psuedo-versor form.
\end{example}
\begin{lemma}
The Gram-Schimdt process cannot always be used on blades taken from
a \alert{non-Euclidean} geometric algebra!
\end{lemma}
\begin{proof}
Consider $a\wedge b$.  If $a\cdot b\neq 0$ and $a,b$ are null, then there does not exist a
scalar $\lambda$ such that $a\cdot(b+\lambda a)=0$ or $(a+\lambda b)\cdot b=0$.
\end{proof}
\end{frame}
%============================================================
% Notes for slide:
% The function f here is just the identity function, but what this does is
% illustrate how you can take multivectors in the geometric product.
% This algorithm is only necessary in non-Euclidean geometric algebras!
% In Euclidean GAs, we can use the Gram-Schmidt process.
%============================================================

\begin{frame}
\frametitle{Blade to Multi-Psuedo-Versor Form}
Let $B\in\G(\V^n)$ be a blade of grade $m>1$ where $B=\bigwedge_{k=1}^m b_k$.
We then have
\begin{align*}
f(B) &= B = b_1 B^{(1)} - b_1\cdot B^{(1)} \\
& = b_1 f(B^{(1)}) - \sum_{k=2}^m(-1)^i (b_1\cdot b_i)f(B^{(1)(i)}),
\end{align*}
where $B^{(i)}$ is notation for the $(m-1)$-blade $\bigwedge_{k=1,k\neq i}^m b_k$.
\begin{example}
For the blade $a\wedge b$, we have $f(a\wedge b)=a\wedge b=ab-a\cdot b$.
\end{example}
\end{frame}

\begin{frame}
\frametitle{Psuedo-Versor to Multivector Form}
Let $V\in\G(\V^n)$ be a versor of size $m>1$ where $V=\prod_{k=1}^m v_k$.
We then have
\begin{align*}
 & f^{-1}(V) = V = v_1\sum_{k=0}^m\langle V^{(1)}\rangle_k = \\
 & \langle f^{-1}(V^{(1)})\rangle_0 v_1+\sum_{k=2}^m\left(
v_1\wedge\langle f^{-1}(V^{(1)})\rangle_k + v_1\cdot\langle f^{-1}(V^{(1)})\rangle_k\right),
\end{align*}
where $V^{(i)}$ is notation for the $(m-1)$-sized psuedo-versor $\prod_{k=1,k\neq i}^m v_k$.
\begin{example}
For the versor $ab$, we have $f(ab)=ab=a\cdot b+a\wedge b$.
\end{example}
\end{frame}

\begin{frame}
\frametitle{The Geometric Product (Again)}
\begin{example}
For any two blades $A,B\in\G(\V^n)$ of grades $i$ and $j$, respectively, it
can be shown that
\begin{equation*}
A\cdot B = \langle AB\rangle_{|i-j|},
\end{equation*}
and
\begin{equation*}
A\wedge B = \langle AB\rangle_{i+j}.
\end{equation*}
\end{example}
\end{frame}

\begin{frame}
\frametitle{The Reverse}
\begin{definition}
For any $m$-sized versor $V\in\G(\V^n)$ where $V=\prod_{k=1}^m v_k$,
we define
\begin{equation*}
\tilde{V} = \prod_{k=1}^m v_{m-k+1}.
\end{equation*}
\end{definition}
We can extend this definition to any multivector if we let
the reverse operator distribute over addition.
\begin{definition}
For any multivector $E\in\G(\V^n)$, we may write
\begin{equation*}
\tilde{E} = f^{-1}(\tilde{f}(E)).
\end{equation*}
\end{definition}
\end{frame}

\begin{frame}
\frametitle{The Inverse}
\begin{lemma}
For any $m$-sized versor $V\in\G(\V^n)$ where $V=\prod_{k=1}^m v_k$, we have
\begin{equation*}
V^{-1} = \left(\prod_{k=1}^m |v_k|\right)^{-1}\tilde{V}.
\end{equation*}
\end{lemma}
\begin{lemma}
For any \alert{Euclidean} $m$-blade $B\in\G(\V^n)$ where $B=\bigwedge_{k=1}^m b_k$, we have
\begin{equation*}
B^{-1} = (-1)^{n(n-1)}\tilde{B}.
\end{equation*}
\end{lemma}
\begin{example}
If $v\in\V^n$ is a null vector, then $v^{-1}$ does not exist.
\end{example}
\end{frame}
%============================================================
% Notes for slide:
% Euclidean blades can always be written as versors using Gram-Schmidt.
% Not all non-Euclidean blades have an inverse WRT the GP.
% Finding the inverse of a general multivector is equivalant to solving a system of linear equations.
%============================================================

\begin{frame}
\frametitle{Conjugation by Versors}
\begin{lemma}
Conjugation by versors is outermorphic.  That is, for any versor $V=\prod_{k=1}^i v_k$,
and any blade $B=\bigwedge_{k=1}^j b_k$, we have
\begin{equation*}
VBV^{-1} = \bigwedge_{k=1}^j Vb_kV^{-1}.
\end{equation*}
\end{lemma}
The proof of this is not too hard to get, but too big to put here.
\begin{example}
A given rotor $R\in\G(\V^n)$ is a versor that rotates points $v\in\V^n$
by versor conjugation.  It therefore rotates blades as well!
\end{example}
\end{frame}

%\section{The Generalized Model}

\begin{frame}
\frametitle{How Blades Can Represent Geometry}
Let $\V^n$ denote a \alert{Euclidean} vector space.

Let $\V$ denote any other vector space.

Let $p:\V^n\to\G(\V)$ be a blade-valued function of points.
\begin{definition}
Given any blade $B\in\G(\V)$, we say that $B$ \alert{directly} represents
the geometry that consists of all points
\begin{equation*}
G(B) = \{x\in\V^n|p(x)\in B\}.
\end{equation*}
\end{definition}
\begin{definition}
Given any blade $B\in\G(\V)$, we say that $B$ \alert{dually} represents
the geometry that consistent of all points
\begin{equation*}
G^*(B) = \{x\in\V^n|p(x)\in B^*\}.
\end{equation*}
\end{definition}
\end{frame}
%============================================================
% Notes for slide:
% Here, a vector is in the vector sub-space represented by a blade iff their outer product is zero.
% Here, * takes a dual, which is multiplying by +/-I.
% A vector is in the vector sub-space represented by a blade's dual iff their inner product is zero.
%============================================================

% Put in slides that explain union-like ops and intersection ops.

\begin{frame}
\frametitle{Finally, The Conformal Model}
Let $\V^n$ be a vector-subpace of $\V$.

If $\{e_k\}_{k=1}^n$ is any basis for $\V^n$, let $\{e_k\}_{k=1}^n\cup\{\nvao,\nvai\}$
be a basis for $\V^n$.
\begin{definition}
For any vector $v\in\V^n$, we define $v\cdot\nvao=v\cdot\nvai=0$.  We
define $\nvao\cdot\nvai=\nvai\cdot\nvao=-1$.  Each of $\nvao$ and $\nvai$
are defined as null.
\end{definition}
\begin{definition}
We define $p:\V^n\to\G(\V)$ as
\begin{equation*}
p(x) = \nvao + x + \frac{1}{2}x^2\nvai.
\end{equation*}
\end{definition}
What we can now discover about this model of geometry is almost endless!
\end{frame}

\end{document}