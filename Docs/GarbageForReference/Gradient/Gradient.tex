\documentclass[12pt]{article}

\usepackage{amsmath}
\usepackage{amssymb}
\usepackage{amsthm}

\title{Gradient}
\author{Spencer T. Parkin}

\newcommand{\G}{\mathbb{G}}
\newcommand{\V}{\mathbb{V}}
\newcommand{\R}{\mathbb{R}}
\newcommand{\A}{\mathbb{A}}
\newcommand{\B}{\mathbb{B}}
\newcommand{\ob}{\overline}
\newcommand{\ub}{\underline}

\newtheorem{theorem}{Theorem}[section]
\newtheorem{definition}{Definition}[section]
\newtheorem{corollary}{Corollary}[section]
\newtheorem{identity}{Identity}[section]
\newtheorem{lemma}{Lemma}[section]
\newtheorem{result}{Result}[section]

\begin{document}

\section*{Examination of Definition 5.2}

I assume that a function $F$ defined on open set in $\R^n$ that is multi-vector valued has
the property that for any $x\in\R^n$, we have $F(x)\in\G(\R^n)$, where $\G(\R^n)$
is the geometric algebra generated by $\R^n$ when thought of as a vector space.
Is this right?  We have to be able to relate the input space and the output space of $F$ for
definition 5.2 to make sense, because we are combining each $e_i$ in some product with
elements in the image of $\partial_i F(x)$.

For any integer $i\in[1,n]$, to understand definition 5.2, we must understand
the meaning of $e_i\partial_i F(x)$, where $x= \sum_{i=1}^n x_i e_i$ and each $x_i\in\R$.  If I understand correctly,
we have
\begin{equation*}
\partial_i F(x) = \lim_{\epsilon\to 0}\frac{F(x + \epsilon e_i)-F(x)}{\epsilon},
\end{equation*}
which, like $F$, is also a multivector-valued function defined on an open set in $\R^n$.
It follows that $\nabla F(x)$, as defined in definition 5.2, may be written as
\begin{align*}
\nabla F(x) &= \sum_{i=1}^n e_i\partial_i F(x) \\
 &= \sum_{i=1}^n\left( e_i\cdot\partial_i F(x) + e_i\wedge\partial_i F(x)\right) \\
 &= \sum_{i=1}^n e_i\cdot\partial_i F(x) + \sum_{i=1}^n e_i\wedge\partial_i F(x),
\end{align*}
if I am not mistaken in my understanding that the product between each $e_i$ and
$\partial_i F(x)$ in definition 5.2 is the geometric product.  If we then define,
(and we need a definition here because I do not believe it follows from anything
we have done thus far), $\nabla\cdot F(x)=\sum_{i=1}^n e_i\cdot\partial_i F(x)$
and $\nabla\wedge F(x)=\sum_{i=1}^n e_i\wedge\partial_i F(x)$, then we may write
\begin{equation*}
\nabla F(x) = \nabla\cdot F(x) + \nabla\wedge F(x),
\end{equation*}
which is my first experience seeing the operator $\nabla$ behave as a vector.
(Does it behave generally like a vector?  That remains to be seen.)  But I do
not really know how $\nabla\cdot F(x)$ and $\nabla\wedge F(x)$ are defined,
or if they can be inferred from more fundamental definitions and results.
I would think not, since the combining of an operator with a function in one of
the inner or outer products is something that must be defined first to make any sense.

I can see that for all integers $i\in[1,n]$, if $\partial_i F(x)$ is a scalar,
then $\nabla F(x)$ needs not be thought of as behaving like a vector,
because it is a vector!  But this is not always the case.

If I'm not off my rocker yet, the linearity of the $\nabla$ operator
follows from the linearity of the $\partial_i$ operator and the
distributivity of the geometric product over addition.

\section*{Examination of Exercise 5.3a}

What does Alan mean by $\nabla x_i=e_i$?
If we let $x=\sum_{i=1}^n x_ie_i$, then perhaps he means $\nabla F(x)$, where
\begin{equation*}
F(x) = \sum_{i=1}^n x\cdot e_i = \sum_{i=1}^n x_i.
\end{equation*}
Applying definition 5.2, we then have
\begin{equation*}
\nabla F(x) = \sum_{i=1}^n e_i\partial_i F(x) = \sum_{i=1}^n e_i\partial_i\sum_{j=1}^n x_j = \sum_{i=1}^n e_i.
\end{equation*}

If, on the other hand, Alan means $\nabla F(x)$, where $F(x)=x\cdot e_i=x_i$
for some fixed integer $i\in[1,n]$, then
\begin{equation*}
\nabla F(x) = \sum_{j=1}^n e_j\partial_j x_i = e_i.
\end{equation*}
I believe that either interpretation is okay, since he's using an implicit summation notation.

\section*{Examination of Exercise 5.3b}

What in the world does Alan mean by $e_i\cdot\nabla$?!  How do we take
a vector and an operator in the inner product?  This just can't make sense,
because the inner product is not defined over a space that includes operators.
It does make sense, however, to write
\begin{equation*}
e_i\cdot\nabla F(x) = \sum_{j=1}^n e_i\cdot\left(e_j\partial_j F(x)\right).
\end{equation*}
Now, if for all integers $j\in[1,n]$, we have $\partial_j F(x)\in\R$, then
it is easy to see that
\begin{equation*}
e_i\cdot\nabla F(x) = \sum_{j=1}^n(e_i\cdot e_j)\partial_j F(x) = \partial_i F(x).
\end{equation*}
But again, in general, we do not have $\partial_j F(x)$ as a scalar-valued function,
so we would have to be very careful, I believe, to lazily rewrite
$e_i\cdot\nabla F(x) = \partial_i F(x)$ as $e_i\cdot\nabla = \partial_i$.
For example, let's suppose that $\partial_j F(x)$ was always a vector valued function.
We then have
\begin{align*}
e_i\cdot\nabla F(x) &= \sum_{j=1}^n e_i\cdot(e_j\cdot \partial_j F(x) + e_j\wedge\partial_j F(x)) \\
 &= \sum_{j=1}^n (e_j\cdot\partial_j F(x))e_i + \sum_{j=1}^n (e_i\cdot e_j)\partial_j F(x) - \sum_{j=1}^n (e_i\cdot \partial_j F(x))e_j \\
 &= \partial_i F(x) + \sum_{j=1}^n\left( (e_j\cdot\partial_j F(x))e_i - (e_i\cdot\partial_j F(x))e_j\right) \\
 &= \partial_i F(x) + \sum_{j=1}^n \partial_j F(x)\cdot(e_j\wedge e_i).
\end{align*}
In this case we can only claim that $e_i\cdot\nabla F(x)=\partial_i F(x)$ when for all $i\neq j$,
we have $\partial_j F(x)$ as zero or as some vector orthogonal to both $e_i$ and $e_j$.
What does this say about $F$?  I'm not sure without further analysis.  In any case,
I am not convinced that an identity such as $e_i\cdot\nabla=\partial_i$, if we can
write it that way, holds generally.

\section*{Misc. Stuff}

Let's suppose $F(x)=x=\sum_{i=1}^n x_ie_i$ is the identity function.  What is $\nabla F$ in this case?

Well, let's start by showing that
\begin{equation*}
\partial_i F(x) = \lim_{\epsilon\to 0}\frac{(x_i+\epsilon)e_i-x_ie_i}{\epsilon} = e_i,
\end{equation*}
which, admittedly, can be done much faster using rules of differentiation.  We then have
\begin{equation*}
\nabla F(x) = \sum_{i=1}^n e_i\partial_i F(x)
 = \sum_{i=1}^n e_ie_i
 = \sum_{i=1}^n 1
 = n.
\end{equation*}

\end{document}