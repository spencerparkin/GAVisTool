\documentclass[12pt]{article}

\usepackage{amsmath}
\usepackage{amssymb}
\usepackage{amsthm}

\title{In Search Of\\The Change Of Basis Transformation\\Using\\Geometric Algebra}
\author{Spencer T. Parkin}

\newcommand{\G}{\mathbb{G}}
\newcommand{\V}{\mathbb{V}}
\newcommand{\R}{\mathbb{R}}
\newcommand{\A}{\mathbb{A}}
\newcommand{\B}{\mathbb{B}}
\newcommand{\C}{\mathbb{C}}

\newtheorem{theorem}{Theorem}[section]
\newtheorem{definition}{Definition}[section]
\newtheorem{corollary}{Corollary}[section]
\newtheorem{identity}{Identity}[section]
\newtheorem{lemma}{Lemma}[section]
\newtheorem{result}{Result}[section]

\begin{document}
\maketitle

\begin{abstract}
One failing of geometric algebra is its inability to
represent the change of basis transformation as
is represented by matrices by design.  Consequently,
geometric algebra may not be the best tool for working with, applying and
representing shear and non-uniform scale transformations.
Two construction methods for performing the change of basis
transformation in geometric algebra are presented in this paper and evaluated for
their usefulness.  The main requirement for usefulness is the
ease, if any, at which the construction lends itself to the problem
of finding the inverse change of basis transformation.
\end{abstract}

Here we will consider only 2-dimensional change of basis transformations for two reasons.
The first is that the generalizations of the constructions given here will be obvious enough.
The second is that by considering the 2-dimensional case only we greatly unobfuscate the
points being made.  That said, we will begin by defining the change of basis transformation.

Letting $\A$ be a 2-dimensional Euclidean vector space, choose any two
linearly independent vectors $x,y\in\A$.  Then for any vector $a\in\A$,
the change of basis transformation of $a$ with respect to the set of
basis vectors $\{x,y\}$ transforms $a$ written as $ (a\cdot e_0)e_0 + (a\cdot e_1)e_1$
into $(a\cdot e_0)x + (a\cdot e_1)y$, where $\{e_0,e_1\}$ is any
set of orthonormal basis vectors for $\A$.  Matrix multiplication performs this type
of transformation by definition, but it is not so easily reproduced in geometric algebra.
It is worth investigating, because this type of transformation allows us to represent
shear and non-uniform scale transformations.  This paper concerns itself with
finding a useful construction in geometric algebra that performs the change of basis
transformation.

Our first construction barrows from ideas set forth in $\cite{Hestenese1993}$.
Letting $\B$ be a 2-dimensional Euclidean vector space disjoint from $\A$, we
will work in the geometric algebra $\G(\A\cup\B)$ where $\{e_0,e_1\}$ and $\{e_2,e_3\}$
are any sets of orthonormal basis vectors spanning $\A$ and $\B$, respectively.  We wish to find
a function $f:\A\to\A$ that performs the change of basis transformation
described earlier.  To that end, we define a function $R$ as follows.
\begin{equation*}
R(u,v) = \frac{\sqrt{2}}{2}\left(1 - uv\right),
\end{equation*}
where $u$ and $v$ are vectors.  When $u$ and $v$ are orthogonal, and each of unit-length, the element $R(u,v)$ is
a rotor transforming $u$ into $v$, but leaving all unit vectors orthogonal to
$u$ and $v$ invarient.  The reader can check that $R(u,v)u\tilde{R}(u,v)=v$
and that for any unit vector $w$ orthognal to $u$ and $v$, we have $R(u,v)w\tilde{R}(u,v)=w$.
Using this function, we can easily define an outermorphism $g$ between $\A$
and $\B$ as $g(w)=Rw\tilde{R}$, where $R=R(e_1,e_3)R(e_0,e_2)$.  It now
follows that a bivector of the form $M=e_2x+e_3y$ may be used to represent
our change of basis transformation and that we may define $f$ as
\begin{equation*}
f(a) = g(a)\cdot M.
\end{equation*}
Now, by linear algebra, we know that $f^{-1}$ exists, because $x$ and $y$ are
linearly independent vectors.  Unfortunately, however, this construction in
geometric algebra is useless to us, because the inner product is not invertible.
Furthermore, even if we were using the geometric product here, $M$ has no
inverse with respect to the geometric product.  In fine, though there must exist
a bivector $M'$ such that $f^{-1}(a) = g(a)\cdot M'$, geometric algebra, unlike
matrix algebra, offers no obvious method for finding it.

Leaving this construction as a dead end, we will now turn our attention to
a similar construction that may prove more promising.  For this new construction,
however, we will have to give up non-uniform scale directly.  This may be acceptable,
because non-uniform scale can be performed indirectly as uniform-scale followed by shear followed by rotation.
In any case, giving up non-uniform scale brings up the requirement that $|x|=|y|=1$.
Shears are still possible, because we do not require that $x\cdot y=0$, only that $x\wedge y\neq 0$.

We start by extending
our geometric algebra to $\G(\A\cup\B\cup\C)$ where $\{e_4,e_5\}$ is any
set of orthonormal basis vectors spaning the 2-dimensional Euclidean vector
space $\C$, which we require to be disjoint from both $\A$ and $\B$.
We now redefine the function $g$ as $g(w)=Rw\tilde{R}$, where
\begin{equation*}
R=R\left(e_1,y_{\C}\right)R\left(e_0,x_{\B}\right).
\end{equation*}
Here we are using the notation
$x_{\B}$ to denote $Rx\tilde{R}$ where $R=R(e_1,e_3)R(e_0,e_2)$ and
$y_{\C}$ to denote $Ry\tilde{R}$ where $R=R(e_1,e_5)R(e_0,e_4)$.
It then follows that $R$, as it is defined for the function $g$, represents our
change of basis transformation, and we can define $f$ as
\begin{equation*}
f(a) = g(a)C,
\end{equation*}
where $C$ is the constant bivector $e_2e_0+e_4e_0 + e_3e_1 + e_5e_1$.
We appear to have made some progress!  Unfortunately, however, the
inverse of $C$ does not exist.  The function that $C$ plays here as a multiplier,
on the other hand, is an invertible function.

% can we use reciprical basis stuff here?

\bibliographystyle{plain}
\bibliography{ChangeOfBasis}

\end{document}