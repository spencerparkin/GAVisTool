\documentclass[12pt]{article}

\usepackage{amsmath}
\usepackage{amssymb}
\usepackage{amsthm}
%\usepackage{biblatex}

\title{Notes on Affine Geometry}
\author{Spencer T. Parkin}

\newcommand{\G}{\mathbb{G}}
\newcommand{\V}{\mathbb{V}}
\newcommand{\R}{\mathbb{R}}
%\newcommand{\a}{\vec{a}}
%\newcommand{\b}{\vec{b}}

\newtheorem{theorem}{Theorem}[section]
\newtheorem{definition}{Definition}[section]
\newtheorem{corollary}{Corollary}[section]
\newtheorem{identity}{Identity}[section]
\newtheorem{lemma}{Lemma}[section]
\newtheorem{result}{Result}[section]

\begin{document}
\maketitle

Geometries are usually described either directly or indirectly
as a set of points in a space.  The setting of affine geometry
takes place in what is called an affine space.  So we must begin
with a notion of what such a space is.

Loosely speaking, an affine space is a set of points where
the spacial relationships between these points is described
by vectors in a vector space.  Notice that there is no need
to designate any one point in an affine space as the origin so
that we can realize the positions of all points in the space
relative to it, such as we might in the usual interpretation
of space using a vector space.
In an affine space, position is a characteristic of a given point only in relation
to some other point.  In this sense, we can forget about the origin.

An affine space is therefore a set of points $P$ accompanied by
a vector space $\V$ where for any point $a\in P$, we may find any
other point $b\in P$ as $b=a+\vec{v}$, where $\vec{v}\in\V$ is
a non-zero vector, and
we have defined the addition of a point and a vector as
another point.  By this it is tempting to define the subtraction
of two points as a vector, but we have to be careful about
how we take points in a linear combination if we wish such
combinations to remain invariant under a change of origin.
And we do wish this, because what we're trying to do is create a structure
of space that is independent of origin, just as a structure of
space modeled by a vector space is independent of what
basis we choose to span that space.

Realizing that the structure of $P$ must somehow depend upon the
structure of $\V$, we procede to define the structure of $P$
in terms of $\V$.  Linear combinations of vectors taken from
$\V$ are well understood, as is the concept of linear independence.
What then will it mean to take linear combinations of points in $P$?
As stated earlier, our answer will be motived by a desire to make
affine spaces independent of the concept of an origin.

Let $\{a_k\}_{k=1}^m$ be a set of $m$ points taken from $P$.
We then define
\begin{equation}\label{equ_lin_combo_points_def}
\sum_{k=1}^m \alpha_k a_k = o + \sum_{k=1}^m \alpha_k\vec{a}_k,
\end{equation}
where $o\in P$ is a common origin of all points in $\{a_k\}_{k=1}^m$,
and therefore, for each integer $k\in[1,m]$, we have $a_k=o + \vec{a}_k$, where $\vec{a}_k\in\V$.
Right away this definition imposes a restriction on the set of scalars $\{\alpha_k\}_{k=1}^n\subset\R$.
To see why, let $p\in P$ be a point such for some non-zero vector $\vec{v}\in\V$, we have $p+\vec{v}=o$.
Equation $\eqref{equ_lin_combo_points_def}$ then becomes
\begin{equation}\label{equ_lin_combo_wrt_o}
\sum_{k=1}^m \alpha_k(p+\vec{v}+\vec{a}_k) = p+\vec{v}+\sum_{k=1}^m\alpha_k\vec{a}_k.
\end{equation}
But by associativity, applying our definition $\eqref{equ_lin_combo_points_def}$ to this, we must have
\begin{equation}\label{equ_lin_combo_wrt_p}
\sum_{k=1}^n \alpha_k(p+\vec{v}+\vec{a}_k) = p+\sum_{k=1}^m \alpha_k(\vec{v}+\vec{a}_k).
\end{equation}
It follows that $\vec{v} = \sum_{k=1}^m\alpha_k\vec{v}$, and therefore
\begin{equation}\label{equ_lin_combo_restrict}
\sum_{k=1}^m\alpha_k = 1.
\end{equation}
Here, $o$ and $p$ were two different origins used to express the same
set of points.  When we took a linear combination of those points in
the context of one origin versus the other, we found that the resulting
points in each context $\eqref{equ_lin_combo_wrt_o}$ and $\eqref{equ_lin_combo_wrt_p}$
would differ, (making our definition $\eqref{equ_lin_combo_points_def}$
inconsistent), unless we imposed the restriction $\eqref{equ_lin_combo_restrict}$.

A 2-dimensional example helps to demonstrate this.  Let $o_1,o_2\in P$ be two
distinct origins.  Let $a,b\in P$ be two distinct points such that
\begin{align*}
a&=o_1+\vec{a}_1=o_2+\vec{a}_2 \\
b&=o_1+\vec{b}_1=o_2+\vec{b}_2,
\end{align*}
for vectors $\vec{a}_1,\vec{a}_2,\vec{b}_1,\vec{b}_2\in\V$.  If we were then to define $a+b$ as
$o_1+\vec{a}_1+\vec{b}_1$,
then we would reach a constradiction in the fact that $a+b$ is also
$o_2+\vec{a}_2+\vec{b}_2$, yet $\vec{a}_1+\vec{b}_1\neq \vec{a}_2+\vec{b}_2$.
Requiring the pattern in equation $\eqref{equ_lin_combo_points_def}$, we therefore see that it does not make
sense to add points in $P$.  An acceptable linear combination of the points $a$ and $b$, however,
can be found as a linear interpolation of these points.  That is, for any scalar $\lambda\in\R$,
we may write
\begin{equation*}
\lambda a+(1-\lambda)b = o_1 + \lambda a_1+(1-\lambda)b_1 = o_2+\lambda a_2+(1-\lambda )b_2.
\end{equation*}
(Make a figure for this.)

As for subtracting two points $a,b\in P$, we need to be careful with notation.
If $a-b$ is short-hand for $(1)a+(-1)b$, then clearly this is unacceptable as $1-1\neq 1$.
For any two points $a,b\in P$, however, there always exists a unique vector $\vec{v}\in\V$
such that $a=b+\vec{v}$.  If we want to be lazy, then it may be acceptable to define
$b-a=\vec{v}$, but I will refraign from doing this to avoid any confusion.

Notice that there is no additive inverse of a point $a\in P$.

\end{document}