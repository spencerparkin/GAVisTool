\documentclass{birkjour}

\usepackage{amsmath}
\usepackage{amssymb}
\usepackage{amsthm}
\usepackage{graphicx}
\usepackage{float}

\newtheorem{thm}{Theorem}[section]
\newtheorem{cor}[thm]{Corollary}
\newtheorem{lem}[thm]{Lemma}
\newtheorem{prop}[thm]{Proposition}
\theoremstyle{definition}
\newtheorem{defn}[thm]{Definition}
\theoremstyle{remark}
\newtheorem{rem}[thm]{Remark}
\newtheorem*{ex}{Example}
\numberwithin{equation}{section}

\newcommand{\G}{\mathbb{G}}
\newcommand{\V}{\mathbb{V}}
\newcommand{\Vb}{\mathbb{\overline{V}}}
\newcommand{\W}{\mathbb{W}}
\newcommand{\R}{\mathbb{R}}

\newcommand{\nvao}{o}
\newcommand{\nvai}{\infty}
\newcommand{\nvaob}{\overline{o}}
\newcommand{\nvaib}{\overline{\infty}}

\newcommand{\eminus}{e_{-}}
\newcommand{\eplus}{e_{+}}
\newcommand{\eminusb}{\overline{e}_{-}}
\newcommand{\eplusb}{\overline{e}_{+}}

\begin{document}

\title{The Quadratic Form In Geometric Algebra}

\author{Spencer T. Parkin}
\email{spencer.parkin@gmail.com}

\numberwithin{equation}{section}

\subjclass{Primary 14J70; Secondary 14J29}

\keywords{Quadric Surface, Quadratic Form, Geometric Algebra}

%\dedicatory{To Melinda and Naomi}

\begin{abstract}
Blah.
\end{abstract}

\maketitle

\section{Finding The Quadratic Form}

Let $\V^n$ be an $n$-dimensional Euclidean vector space, and identify vectors
in this space with points in $n$-dimensional Euclidean space.  That is,
for any vector $v\in\V^n$, identify this vector with the point at its tip when
its tail is placed at origin.  Letting any subset of $\V^n$ be what we refer to
as a geometry, the goal of this paper is to use geometric algebra in the study of all
such geometries that occur as the zero set of one or more quadratic forms.\footnote{In
algebraic geometry, the zero set of one or more polynomials is called an affine variety.}
A quadratic form $q:\V^n\to\R$ is a quadratic polynomial in the vector components of any
vector $v\in\V^n$.  Specifically, we have
\begin{equation}\label{equ_quadratic_form}
q(v) = C + \sum_{i=1}^n C_i(v\cdot e_i) + \sum_{i=1}^n\sum_{j=1}^n C_{ij}(v\cdot e_i)(v\cdot e_j),
\end{equation}
where $C$, each of $C_i$ and each of $C_{ij}$ are scalars in $\R$.  The coeficients $C$, $C_i$ and $C_{ij}$ collectively determine the geometry that is the zero set of $q$.  Adding a Euclidean vector $e_0$
representative of the origin to $\V^n$ to obtain the $(n+1)$-dimensional Euclidean
vector space $\V^{n+1}$, we see that the quadratic form $q$ is determined by a
symmetric billinear form $B:\V^{n+1}\times\V^{n+1}\to\R$ as
\begin{align}
q(v) &= B(e_0+v,e_0+v) \\
 &= B(e_0,e_0)+2\sum_{i=1}^n B(e_0,e_i)(v\cdot e_i)
 + \sum_{i=1}^n\sum_{j=1}^n B(e_i,e_j)(v\cdot e_i)(v\cdot e_j),
\end{align}
if we let $B(e_0,e_0)=C$, each of $B(e_0,e_i)=B(e_i,e_0)=\frac{1}{2}C_i$ and each of
$B(e_i,e_j)=C_{ij}$.
In turn, we see that the symmetric billinear form $B$ is determined entirely by how it
maps a basis of $\V^{n+1}$.

To find the quadratic form $q$ in geometric algebra, it is clear now that one approach
is to go about looking for the symmetric bilinear form $B$.  Two instances of this form
are found and detailed in the following two sections.

\section{The Quadratic Form In $\G(\V^{n+1}\oplus\Vb^{n+1})$}

Here we let $\Vb^{n+1}$ be an $(n+1)$-dimensional Euclidean vector space
isomorphic to $\V^{n+1}$, and then define the over-bar notation on elements
of the geometric algebra $\G(\W)$, with $\W=\V^{n+1}\oplus\Vb^{n+1}$, as an
outermorphic isomorphism between $\G(\V^{n+1})$ and $\G(\Vb^{n+1})$.
That is, for any element $E\in\G(\W)$, we have
\begin{equation}
\overline{E} = QEQ^{-1},
\end{equation}
where the rotor $Q$ is given by
\begin{equation}
Q = \prod_{i=0}^n(1-e_i\overline{e}_i).
\end{equation}
Having defined $\G(\W)$, we will introduce the function $S:\V^{n+1}\to\G(\W)$, given by
\begin{equation}
S(p) = p\overline{p},
\end{equation}
and then see that for all vectors $p\in\V^{n+1}$, the symmetric bilinear
form $B$ occurs in $\G(\W)$ as
\begin{equation}\label{equ_quadric_transformable}
B(p,p) = -S(p)\cdot\sum_{i=0}^n\sum_{j=0}^n B(e_i,e_j)e_i\overline{e}_j,
\end{equation}
showing that the bivectors $E\in\G(\W)$ are representative of $n$-dimensional
quadric surfaces as the set of all vectors $v\in\V^n$ such that $q(v)=0$, where
\begin{equation}\label{equ_quadric_form_in_ga}
q(v)=S(e_0+v)\cdot E.
\end{equation}
This approach is especially advantageous in the realization that for
any versor $V\in\G(\V^{n+1})$, we have
\begin{equation}
S(V^{-1}pV)\cdot E=S(p)\cdot V\overline{V}E(V\overline{V})^{-1}
\end{equation}
by the fact that $S$ has the property
\begin{equation}
S(V^{-1}pV)=(V\overline{V})^{-1}S(p)V\overline{V},
\end{equation}
which shows that if we understand how $V$ transforms homogeneous
points $p\in\V^{n+1}$ as $V^{-1}pV$, then we also understand
how $V$ transforms quadric surfaces $E\in\G(\W)$ as $(V\overline{V})E(V\overline{V})^{-1}$.
In a variation of this approach that uses the geometric algebra $\G(\V^{n+1,1}\oplus\V^{n+1,1})$,
the versors of the conformal model of geometric algebra may be used to transform quadric surfaces.

A down-side to this approach, however, is in the fact that we're not using blades
to represent quadric surfaces in the same way that blades are representive of geometries
in the conformal model of geometric algebra.  Consequently, we cannot similarly benefit
from the meet and join operations.
We will attempt to remedy this problem in the next section.

\section{The Quadratic Form In $\G(\V^{(n+1)^2})$}

Notice that in the previous method, the Euclidean space $\V^n$ was
embedded in the representation space $\G(\W)$.  For the method to
follow, we show that this need not be the case.  Specifically, we
do not let $\V^n$ be a vector sub-space of the $(n+1)^2$-dimensional
anti-Euclidean vector space $\V^{(n+1)^2}$.\footnote{To be anti-Euclidean means
that the inner product square of any two vectors is a non-positive scalar.}
We will, however, continue to let $\V^n$ be a proper vector sub-space of $\V^{n+1}$.

Letting $\{e_{ij}\}$ be a set
of orthonormal basis vectors spanning $\V^{(n+1)^2}$, we reintroduce the
function $S:\V^{n+1}\to\V^{(n+1)^2}$ as
\begin{equation}
S(p) = p\otimes p,
\end{equation}
where $\otimes:\V^{n+1}\times\V^{n+1}\to\V^{(n+1)^2}$ is a commutative
bilinear and binary operator, defined as
\begin{equation}
x\otimes y = \sum_{i=0}^n\sum_{j=0}^n(x\cdot e_i)(y\cdot e_j)e_{ij},
\end{equation}
and then find that for all vectors $p\in\V^{n+1}$, the symmetric billinear form
$B$ in $\G(\V^{(n+1)^2})$ is found as
\begin{equation}\label{equ_quadric_intersectable}
B(p,p) = -S(p)\cdot\sum_{i=1}^n\sum_{j=1}^n B(e_i,e_j)e_{ij},
\end{equation}
showing that the vectors $E\in\G(\V^{(n+1)^2})$ are representative
of $n$-dimensional quadric surfaces as the set of all vectors $v\in\V^n$
such that $q(v)=0$, where $q$ is again given by equation \eqref{equ_quadric_form_in_ga}.

Immediately we see that the advantage to this approach is that a
non-zero blade $E\in\G(\V^{(n+1)^2})$ of grade $k$ is representative
of an $(n+1-k)$-dimensional quadric surface.  To see this,
let $E=E_1\wedge\dots\wedge E_k$, and realize that
\begin{equation}
0 = S(p)\cdot\bigwedge_{i=1}^k E_i = \sum_{i=1}^k (S(p)\cdot E_i)\bigwedge_{j=1,j\neq i}^k E_j
\end{equation}
if and only if for all integers $i\in[1,k]$, we have $S(p)\cdot E_i=0$.  In other words,
$E$ represents the affine variety generated by the set of all quadradic polynomials
determined by each $E_i$.

We will refer to $E$ as a dual quadric if we are interpreting it as being representative
of a quadric surface in terms of the equation
\begin{equation}\label{equ_dual_quadric}
0 = S(e_0+v)\cdot E.
\end{equation}
Similarly, we will refer to $E$ as a direct quadric if we are interpreting it as being
representative of such a surface in terms of the equation
\begin{equation}\label{equ_direct_quadric}
0 = S(e_0+v)\wedge E.
\end{equation}
To see that this is also the previously mentioned affine variety, simply realize that
\begin{equation}
\mbox{$0 = S(e_0+v)\wedge E$ if and only if $0 = S(e_0+v)\cdot EI$,}
\end{equation}
where $I$ is the unit-psuedo scalar of $\G(\V^{(n+1)^2})$.

Notice that any single blade $E\in\G(\V^{(n+1)^2})$ is
simultaneously representative of both a dual and direct quadric, which are distinct
pieces of geometry.\footnote{A dual quadric is directly represented by its dual, and
a direct quadric is dually represented by its dual.  As a given blade simultaneously
represents two geometries, (one dually, the other directly), a single given geometry is
simultaneously represented by two blades, (which are duals of one another).}
It is sometimes useful to reinterpret a dual quadric
as a direct quadric, or vice versa.  For example, if the dual intersection of
two dual quadrics is imaginary, the imaginary intersection may be a real quadric
in direct form.

%The immediate down-side to this approach, however, is that we lose
%the versor transformation properties of the prior method.  To see this,
%realize that for any vector $v\in\V^{n+1}$, we have
%\begin{equation}
%S(v^{-1}pv) = 2(p\cdot v^{-1})v\otimes y - S(p),
%\end{equation}
%and that for any vector $w\in\V^{(n+1)^2}$, we have
%\begin{equation}
%w^{-1}S(p)w = \frac{2}{w^2}\sum_{i=1}^n\sum_{j=1}^n(w\cdot e_{ij})^2(p\cdot e_i)^2e_{ij} - S(p).
%\end{equation}
%Given a vector $v\in\V^{n+1}$, there is not always a choice of $w\in\V^{(n+1)^2}$ such
%that $S(v^{-1}pv)=w^{-1}S(p)$w.  (Needs more work.)

\section{Point Fitting Quadrics In $\G(\V^{(n+1)^2})$}

Suppose $E\in\G(\V^{(n+1)^2})$ is a direct quadric of grade $k$, and
that $\{p_i\}_{i=1}^k$ is a set of $k$ homogeneous points taken from $\V^{n+1}$
such that for all integers $i\in[1,k]$, we have $S(p_i)\wedge E=0$.  Then,
if $\{S(p_i)\}_{i=1}^k$ is a linearly independent set, it follows that
there exists a scalar $\lambda\in\R$ such that
\begin{equation}
\bigwedge_{i=1}^k S(p_i) = \lambda E.
\end{equation}
What this shows is that, given a set of $k$ points $\{p_i\}_{i=1}^k$, we can find a quadric $E$
that fits the $k$ points, provided the set $\{S(p_i)\}_{i=1}^k$ is linearly
independent.  Two questions
arise from this.  First, under what circumstances do the $k$ points generate
a linearly independent set $\{S(p_i)\}_{i=1}^k$; and secondly, under
those circumstances, what quadric surface do we get?  (Address this somehow,
preferably by finding the answers.)

From what we have thus far gathered, an $(n+1-k)$-dimensional quadric surface
would be fit to $(n+1)^2-k$ points if it were at all possible to find such a set
of points generating a linearly independent set.  Possible or not, it is easy to show that this
is certainly not the least upper bound on the number of points needed to determine
such a surface.  To see why, define $S_0:\V^n\to\V^m$ as
\begin{equation}
S_0(p) = e_{00}+\sum_{i=1}^n(p\cdot e_i)e_{0i} + \sum_{i=1}^n\sum_{j=i+1}^n(p\cdot e_i)(p\cdot e_j)e_{ij}
\end{equation}
where $m$ is given by
\begin{equation}
m = \binom{n}{0}+\binom{n}{1}+\binom{n}{2},
\end{equation}
and $\V^m$ is a proper vector sub-space of $\V^{(n+1)^2}$ and spanned
by the vectors in $\{e_{00}\}\cup\{e_{0i}\}_{i=1}^n\cup\{e_{ij}\}_{i<j}$.
Using now $I_0$, what we'll use to denote the unit psuedo-scalar of $\G(\V^m)$,
to transition between dual and direct quadrics\footnote{Notice that quadrics in $\G(\V^m)$
defined using $S_0$ are still valid quadrics in $\G(\V^{(n+1)^2})$ under the definition
using $S$.  The converse, however, is not true.  In any case, we may think of the
quadrics in $\G(\V^{(n+1)^2})$ satisfying both definitions as being in a reduced form.},
and using $S_0$ in place of $S$
to define dual and direct quadrics by equations \eqref{equ_dual_quadric} and
\eqref{equ_direct_quadric}, we see that
an $(n+1-k)$-dimensional quadric surface may be fit to $m-k$ points, from which it
is more likely that we'll generate a linearly independent set.

% Do it on the computer?  can we show a figure where i've
% fit points to a quadric?

\section{Switching Between $\G(\V^{n+1}\oplus\Vb^{n+1})$ And $\G(\V^{(n+1)^2})$}

% Compare canonical forms in this section.
% S(a).S(b) = (a.b)^2

If you found the choice of an anti-Euclidean vector
space in section \eqref{} odd, the reason for this will
now come to light.  To gain the advantages of working
in both $\G(\V^{n+1}\oplus\Vb^{n+1})$ and $\G(\V^{(n+1)^2})$,
it may not be unreasonable to switch between the two
algebras when needed.  To do this, we simply use the linear
function $f:\V^{(n+1)^2}\to\G(\V^{n+1}\oplus\Vb^{n+1})$,
defined in terms of how it maps the basis vectors of $\V^{(n+1)^2}$ onto the
basis bivectors of the linear sub-space of bivectors in $\G(\V^{n+1}\oplus\Vb^{n+1})$ as follows.
For all pairs of integers $(i,j)\in[0,n]\times[0,n]$, we define
\begin{equation}
f(e_{ij})=e_i\overline{e}_j.
\end{equation}
We now see that for any vector $E\in\V^{(n+1)^2}$ representative of
a quadric surface through the use of equation\eqref{equ_quadric_intersectable}, the bivector
$f(E)\in\G(\V^{n+1}\oplus\Vb^{n+1})$ is representative of the same
quadric through the use of equation \eqref{equ_quadric_transformable}.

This gives us the ability to transform any intersection of one or more quadrics
in $\G(\V^{(n+1)^2})$ as we would a single quadric in $\G(\V^{n+1}\oplus\Vb^{n+1})$.
For a given blade $E\in\G(\V^{(n+1)^2})$, we need only find a factorization
of the blade $E$ as $E_i\wedge\dots\wedge E_k$, then formulate the transformation $E'$
of $E$ by a versor $V\in\G(\V^{n+1}\oplus\Vb^{n+1})$ as
\begin{equation}
E' = \bigwedge_{i=1}^k f^{-1}\left(V\overline{V}f(E_i)(V\overline{V})^{-1}\right).
\end{equation}
The problem of blade factorization has been given a great deal of treatment in \cite{}.

\section{Closing Remarks}

% synthetic geometry <--> computational geometry

\bibliographystyle{amsplain}
\bibliography{Parkin_TheQuadraticFormInGA}

% cite http://en.wikipedia.org/wiki/Zero_set
% cite http://www.science.uva.nl/~geer/Ch2.pdf
% cite a work on blade factorization
% cite your own work on the quadric model
% cite your own work on the extension/variation of the quadric model

\end{document}