\documentclass{birkjour}

\usepackage{amsmath}
\usepackage{amssymb}
\usepackage{amsthm}
\usepackage{graphicx}
\usepackage{float}

\newtheorem{thm}{Theorem}[section]
 \newtheorem{cor}[thm]{Corollary}
 \newtheorem{lem}[thm]{Lemma}
 \newtheorem{prop}[thm]{Proposition}
 \theoremstyle{definition}
 \newtheorem{defn}[thm]{Definition}
 \theoremstyle{remark}
 \newtheorem{rem}[thm]{Remark}
 \newtheorem*{ex}{Example}
 \numberwithin{equation}{section}

\newcommand{\G}{\mathbb{G}}
\newcommand{\V}{\mathbb{V}}
\newcommand{\Vb}{\mathbb{\overline{V}}}
\newcommand{\W}{\mathbb{W}}
\newcommand{\R}{\mathbb{R}}

\newcommand{\Alpha}{A}
%\Omega is already defined

\newcommand{\nvao}{o}
\newcommand{\nvai}{\infty}
\newcommand{\nvaob}{\overline{o}}
\newcommand{\nvaib}{\overline{\infty}}

\newcommand{\eminus}{e_{-}}
\newcommand{\eplus}{e_{+}}
\newcommand{\eminusb}{\overline{e}_{-}}
\newcommand{\eplusb}{\overline{e}_{+}}

\begin{document}

\title{An Extension Of The Quadric Model}

\author{Spencer T. Parkin}
\address{%
2113 S. Claremont Dr.\\
Bountiful, Utah  84010\\
USA}
\email{spencer.parkin@gmail.com}

\numberwithin{equation}{section}

\subjclass{Primary 14J70; Secondary 14J29}

\keywords{Quadric Surface, Geometric Algebra, Quadric Model}

%\dedicatory{To Melinda and Naomi}

\begin{abstract}
An extension of the quadric model set forth in \cite{Parkin12} is
found in which the rigid body motions of any quadric surface can be performed
by versor transformations.  Extending the model yet further to include
cubic and quartic surfaces, we find that it is possible to transform
quadric surfaces by any transformation of the conformal model.
\end{abstract}

\maketitle

\section{Introduction}

In the original paper \cite{Parkin12}, a model for quadric surfaces was
presented based upon the ideas of projective geometry.  What was unfortunate
about this model, however, was its lack of support for the rigid body transformations.  It was
predicted in the conclusion of that paper that a better model for quadric
surfaces may exist that is more like the conformal model of geometric algebra.
The present paper details what may be such a model.  We'll find that the rigid
body transformations can be encorporated into the model with a small
extension of the quadric form.  An additional extension to the quadric form
will allow us to support the conformal transformations at the expense of
expanding our model to necessarily include the cubic and quartic surfaces.  Both extensions
of the original quadric model will use the same geometric algebra.

\section{The Geometric Algebra}

We begin here with a description of the structure of the geometric algebra upon
which our model will be imposed.  This
geometric algebra will contain the following nested vector spaces.
\begin{equation}\label{equ_vector_spaces}
\begin{array}{ll}
\mbox{Notation} & \mbox{Basis} \\
\hline
\V^e & \{e_i\}_{i=1}^n \\
\V^o & \{\nvao\}\cup\{e_i\}_{i=1}^n \\
\V & \{\nvao\}\cup\{e_i\}_{i=1}^n\cup\{\nvai\}
\end{array}
\end{equation}
The set of vectors $\{e_i\}_{i=1}^n$ forms an orthonormal set of basis
vectors for the $n$-dimensional Euclidean vector space $\V^e$, which we'll
use to represent $n$-dimensional Euclidean space.
The vectors $\nvao$ and $\nvai$ are the familiar null-vectors representing the
points at origin and infinity taken from the conformal model of geometric algebra.
An inner-product table for these basis vectors is given as follows, where
$1\leq i<j\leq n$.
\begin{equation}
\begin{array}{c|cccc}
\cdot & \nvao & e_i & e_j & \nvai \\
\hline
\nvao & 0 & 0 & 0 & -1 \\
e_i & 0 & 1 & 0 & 0 \\
e_j & 0 & 0 & 1 & 0 \\
\nvai & -1 & 0 & 0 & 0
\end{array}
\end{equation}
We will now let $\G(\V)$ denote the Minkowski geometric algebra generated by $\V$.
For each vector space in table \eqref{equ_vector_spaces}, we will let an over-bar
above this vector space denote an identical copy of that vector space.  The vector
space $\W$ will denote the smallest vector space containing each of $\V$ and $\Vb$
as vector subspaces.  In symbols, one may write
\begin{equation}
\G(\W) = \G(\V)\oplus\G(\Vb)
\end{equation}
to illustrate the structure of $\G(\W)$ in terms of its two isomorphic Minkowski
geometric sub-algebras $\G(\V)$ and $\G(\Vb)$.

We will use over-bar notation to distinguish between vectors taken from $\V$
with vectors taken from $\Vb$.  Though not necessary, we can work exclusively
in $\G(\V)$ by defining the over-bar notation as an outermorphic ismorphism between
$\G(\V)$ and $\G(\Vb)$.  Doing so, we see that for any element $E\in\G(\V)$,
we may define $\overline{E}\in\G(\Vb)$ as
\begin{equation}
\overline{E} = SE\tilde{S},
\end{equation}
where $S$ is the versor given by
\begin{equation}\label{equ_isomorphism_versor}
S = 2^{-n/2}(1+\eminus\eminusb)(1-\eplus\eplusb)\prod_{i=0}^n(1-e_i\overline{e}_i).
\end{equation}
This definition is non-circular if we let the over-bars in equation \eqref{equ_isomorphism_versor}
be purely notation.  The vectors $\eminus$ and $\eplus$, taken from \cite{LiRockwood},
are defined as
\begin{align}
\eminus &= \frac{1}{2}\nvai + \nvao \\
\eplus &= \frac{1}{2}\nvai - \nvao.
\end{align}
The vectors $\eminusb$ and $\eplusb$ are defined similarly in terms of $\nvaob$ and $\nvaib$.
Defined this way, it is important to realize that, unlike the over-bar function defined in \cite{Parkin12},
here we do not have the property that for any vector $w\in\W$, we have $\overline{\overline{w}}=w$.
This is because $\overline{\nvaob}=-\nvao$ and $\overline{\nvaib}=-\nvai$.

\section{The Form Of Quadric Surfaces In $\G(\W)$}

We now give a formal definition under which elements $E\in\G(\W)$
are representative of $n$-dimensional quadric surfaces in the first model.
\begin{defn}\label{def_quadric}
Referring to an element $E\in\G(\W)$ as a quadric surface, it is representative of such an $n$-dimensional
surface as the set of all points $p\in\V^o$ such that
\begin{equation}\label{equ_quadric_equation}
0 = p\wedge\overline{p}\cdot E.
\end{equation}
\end{defn}
From this definition it can be seen that the general form of a quadric $E\in\G(\W)$ is given by
\begin{equation}\label{equ_quadric_form}
E = \sum_{i=1}^n\sum_{j=1}^n\lambda_{ij}e_i\overline{e}_j+
\sum_{i=1}^n\lambda_i(e_i\nvaib + \nvai\overline{e}_j)+
\lambda\nvai\nvaib.
\end{equation}
This is because an element of the form \eqref{equ_quadric_form}, when
substituted into equation \eqref{equ_quadric_equation}, produces a polynomial
equation of degree 2 in the vector components of $p-\nvao$.
Doing so with $p=\nvao+x$, where $x\in\V^e$, we get the equation
\begin{equation}
0 = -\sum_{i=1}^n\lambda_{ii}(x\cdot e_i)^2-
\sum_{i=1}^n\sum_{\substack{j=1\\j\neq i}}^n 2\lambda_{ij}(x\cdot e_i)(x\cdot e_j)+
\sum_{i=1}^n 2\lambda_i(x\cdot e_i) - \lambda,
\end{equation}
which we may recognize as the equation for an $n$-dimensional quadric surface.
Of course, using geometric algebra, it is undesirable and unecessary to think of
quadrics in terms of polynomial equations.  A better way to think of quadrics is in terms
of an element of the algebra whose decomposition
produces the parameters characterizing the quadric surface.  For example, many common
quadrics are the solution set in $\V^e$ of the equation
\begin{equation}
0 = -r^2 + (x-c)^2 + \lambda((x-c)\cdot v)^2,
\end{equation}
in the variable $x$.  (An explanation of the parameters $r$, $c$, $v$ and $\lambda$
was given in \cite{Parkin12}.)  Then, factoring out $-p\wedge\overline{p}$, we see that
the element $E\in\G(\W)$, given by
\begin{equation}
\Omega + \lambda v\overline{v}+2(c+\lambda(c\cdot v)v)\nvaib+
(c^2+\lambda (c\cdot v)^2-r^2)\nvai\nvaib
\end{equation}
is representative of this very same quadric by Definition~\ref{def_quadric},
where $\Omega$ is defined as
\begin{equation}
\Omega = \sum_{i=1}^n e_i\overline{e}_i.
\end{equation}
Give table of canonical forms here...

\section{Transformations Supported By The Model}

The main result of this section will depend upon the following lemma.
\begin{lem}\label{lma_versor_transfer}
For any versor $V\in\G(\W)$, and any four vectors $a,b,c,d\in\V$, we have
\begin{equation}
V^{-1}aV\wedge\overline{V^{-1}bV}\cdot c\wedge\overline{d} =
a\wedge\overline{b}\cdot V\overline{V}(c\wedge\overline{d})(V\overline{V})^{-1}.
\end{equation}
\end{lem}
\begin{proof}
We begin by first establishing that
\begin{align}
 & V^{-1}aV\wedge\overline{V^{-1}bV}\cdot c\wedge\overline{d} \\
&\;= -(V^{-1}aV\cdot c)(V^{-1}bV\cdot d) \\
&\;= -(a\cdot VcV^{-1})(b\cdot VdV^{-1}) \\
&\;= a\wedge\overline{b}\cdot VcV^{-1}\wedge\overline{VdV^{-1}}.
\end{align}
We now notice that
\begin{align}
& VcV^{-1} \\
=\;& V\overline{VV^{-1}}cV^{-1} \\
=\;& (-1)^m V\overline{V}c\overline{V^{-1}}V^{-1} \\
=\;& (-1)^m V\overline{V}c(V\overline{V})^{-1},
\end{align}
where $m$ is the number of vectors taken together in a geometric
product to form $V$.  We then notice that
\begin{align}
& \overline{VdV^{-1}} \\
=\;& VV^{-1}\overline{VdV^{-1}} \\
=\;& (-1)^{m^2}V\overline{V}V^{-1}\overline{dV^{-1}} \\
=\;&(-1)^{m^2+m}V\overline{Vd}V^{-1}\overline{V^{-1}} \\
=\;&(-1)^{2m^2+m}V\overline{VdV^{-1}}V^{-1} \\
=\;&(-1)^mV\overline{V}d(V\overline{V})^{-1}.
\end{align}
It follows now that
\begin{equation}
a\wedge\overline{b}\cdot VcV^{-1}\wedge\overline{VdV^{-1}} =
a\wedge\overline{b}\cdot V\overline{V}(c\wedge\overline{d})(V\overline{V})^{-1}.
\end{equation}
\end{proof}
We're now ready to prove the main result as follows.
\begin{thm}\label{thm_quadric_transform}
Letting $E\in\G(\W)$ be a bivector of the form \eqref{equ_quadric_form}, if
$V\in\G(\V)$ is a versor for which the transformation $p'\in\V^o$ of any point $p\in\V^o$
by $V^{-1}$, given by
\begin{equation}\label{equ_get_rid_ni}
p' = \nvao\cdot V^{-1}pV\wedge\nvai,
\end{equation}
is understood, and if the element $E'$, given by
\begin{equation}\label{equ_transformed_surface}
E' = V\overline{V}E(V\overline{V})^{-1},
\end{equation}
is also of the form \eqref{equ_quadric_form}, then $E'$
is representative of the $n$-dimensional quadric surface by Definition~\ref{def_quadric}
that is the transformation of $E$ by $V$ as $p'$ is the transformation of $p$ by $V$,
considering $p$ to be any point on the surface of $E$, and $p'$ to be the corresponding
point on the surface of $E'$.
\end{thm}
\begin{proof}
The proof of this theorem follows from the observation that the desired transformation
of $E$ by $V$ is given by the set of all points $p\in\V^o$ such that
\begin{equation}\label{equ_transformed_variety}
0 = V^{-1}pV\wedge\overline{V^{-1}pV}\cdot E.
\end{equation}
To see this, realize that since $E$ is of the form \eqref{equ_quadric_form}, we have
\begin{equation}
V^{-1}pV\wedge\overline{V^{-1}pV}\cdot E = p'\wedge\overline{p}'\cdot E.
\end{equation}
Then, applying Lemma~\ref{lma_versor_transfer}, we then see that
\begin{equation}
V^{-1}pV\wedge\overline{V^{-1}pV}\cdot E = p\wedge\overline{p}\cdot (V\overline{V})E(V\overline{V})^{-1},
\end{equation}
showing that by Definition~\ref{def_quadric}, the element $E'$ is representative of the very same
algebraic variety given by equation \eqref{equ_transformed_variety}.
\end{proof}

We can now apply Theorem~\ref{thm_quadric_transform} to show
that the rigid body transformations are supported by our extended model.
Letting $\pi\in\V$ be a conformal dual plane, given by
\begin{equation}
\pi = v+(c\cdot v)\nvai,
\end{equation}
where $v\in\V^e$ is a unit-length vector indicating the norm of the plane,
and where $c\in\V^e$ is a vector representing a point on the plane,
we see that for any point $p\in\V^o$, we have
\begin{equation}
-\pi p\pi^{-1} = \nvao+x-2((x-c)\cdot n) + \lambda\nvai,
\end{equation}
where the scalar $\lambda\in\R$ is of no consequence.  Letting $V=\pi$,
the point $p'\in\V^o$ of consequence here is given by equation \eqref{equ_get_rid_ni},
from which we can recognize an orthogonal reflection about the plane $\pi$.
It now follows by Theorem~\ref{thm_quadric_transform} that $\pi$ is a versor
capable of reflecting any quadric surface about the plane it represents.
Being able to perform planar reflections of any quadric in any plane, it
now follows that we can find a versor $V\in\G(\W)$ capable of performing
any rigid body motion of any quadric surface.

%\section{Putting Theory Into Practice}

\section{Extending The Model Yet Further}

Interestingly, if we were not content with the rigid by motions of
quadrics only, then we really could find what is, for example, the spherical
reflection of, say, an finite cylinder in a sphere.  To do this, we change
Definition~\ref{def_quadric} into the following definition.
\begin{defn}\label{def_surface}
For any element $E\in\G(\W)$, we may refer to it as an $n$-dimensional
surface, quadric, cubic or quartic, as the set of all points $p\in\V^e$ such that
\begin{equation}\label{equ_surface_variety}
0 = P(p)\wedge\overline{P}(p)\cdot E,
\end{equation}
where $P:\V^e\to\V$ is the conformal mapping, defined as
\begin{equation}
P(p) = \nvao + p + \frac{1}{2}p^2\nvai.
\end{equation}
\end{defn}
A version of Theorem~\ref{thm_quadric_transform} is then easily found
where if $V\in\G(\W)$ is any versor of the conformal model, and if $E$
is a surface under Definition~\ref{def_surface}, then the element $E'\in\G(\W)$,
given by equation \eqref{equ_transformed_surface}, must, by Definition~\ref{def_surface},
 be representative of the transformation of $E$ by the versor $V$.  The form for
such elements $E$ in Defintion~\ref{def_surface} is much more involved than
the form \eqref{equ_quadric_form}.  Nevertheless, it is possible to extract
a general quartic equation from equation \eqref{equ_surface_variety}.

\bibliographystyle{amsplain}
\bibliography{Parkin_AnExtensionOfTheQuadricModel}

\end{document}