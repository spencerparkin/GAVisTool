\documentclass[12pt]{article}

\usepackage{amsmath}
\usepackage{amssymb}
\usepackage{amsthm}

\title{An Introduction To\\Projective Geometry\\Using\\Geometric Algebra}
\author{Spencer T. Parkin}

\newcommand{\G}{\mathbb{G}}
\newcommand{\V}{\mathbb{V}}
\newcommand{\R}{\mathbb{R}}
\newcommand{\B}{\mathbb{B}}
\newcommand{\nvao}{o}
\newcommand{\nvai}{\infty}

\newtheorem{theorem}{Theorem}[section]
\newtheorem{definition}{Definition}[section]
\newtheorem{corollary}{Corollary}[section]
\newtheorem{identity}{Identity}[section]
\newtheorem{lemma}{Lemma}[section]
\newtheorem{result}{Result}[section]

\begin{document}
\maketitle

This paper is my attempt to build up the subject of projective
geometry using geometric algebra in my own words.  I do not
claim originality to any result in this paper.  If nothing else, this
paper simply represents a formal compilation of my notes on the
subject.  I have mainly used \cite{hestenes91} and \cite{dorst07}
for research in the preparation of this paper.

I began my study of projective geometry using geometric algebra
after having already put much effort into understanding the
conformal model of geometric algebra.  I may therefore make reference
to concepts in the conformal model as we go along, and so I assume
a small familiarity with that model on the part of the reader, though I
would not consider a full undersatnding of that model a prerequisite.

\section{Representing Geometry}

Like the conformal model, we may think of geometries as subsets of the
set of all points in some $n$-dimensional Euclidean space, which we'll
denote by $\V^n$.  This also denotes an $n$-dimensional Euclidean vector space
as we adopt here the standard correlation between vectors in such a vector space
with points in an $n$-dimensional Euclidean space.

Points sets, of course, do not lend themselves easily to goemetric analysis.  So,
like the conformal model, we represent them using blades in a geometric algebra.
Why we use blades will become apparent after we define how a blade represents
a point set, because then it will become clear how the meet and join operations
of blades will allow us to do some interesting geometric operations, just as we
can in the conformal model.

For $n$-dimensional projective geometry, we use a geometric algebra generated
by an $(n+1)$-dimesnional Euclidean vector space.  If $\{e_k\}_{k=0}^{n-1}$ is any set of orthonormal basis
vectors spanning $\V^n$, let $\{e_k\}_{k=0}^n$ be a set of
orthonormal basis vectors spanning $\V^{n+1}$, which we'll use to generate our
geometric algebra $\G(\V^{n+1})$.

In projective geometry we can represent points, lines, planes, hyper-planes, and so on to higher dimensions.
Certainly results in geometry involving all of these types of geometric 
primitives can be found by simply using $\V^n$ alone, but what we'll see is that the extra dimension
in $\V^{n+1}$ will facilitate some amazingly useful constructions in $\G(\V^{n+1})$ that
make the finding of such results much easier than it would be otherwise.  Indeed, in
\cite{hestenes91}, it is shown how geometric algebra easily and naturally explains many
fundamental theorems in projective geometry.
It is my guess that interpretations of how these constructions work based on $(n+1)$-dimensional
projections into $n$-dimensional space are at least partially to blame for the title of the
subject being projective geometry.

Without further delay, we begin with a function $p:\V^n\to\V^{n+1}$ that defines
a mapping from points in our Euclidean space with vectors in our geometric algebra.
Blah blah blah...

\bibliographystyle{plain}
\bibliography{PGAIntro}

\end{document}