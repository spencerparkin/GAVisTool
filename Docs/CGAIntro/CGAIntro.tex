\documentclass[12pt]{article}

\usepackage{amsmath}
\usepackage{amssymb}
\usepackage{amsthm}

\title{An Introduction To\\Conformal Geometric Algebra}
\author{Spencer T. Parkin}

\newcommand{\G}{\mathbb{G}}
\newcommand{\V}{\mathbb{V}}
\newcommand{\R}{\mathbb{R}}
\newcommand{\B}{\mathbb{B}}
\newcommand{\nvao}{o}
\newcommand{\nvai}{\infty}

\newtheorem{theorem}{Theorem}[section]
\newtheorem{definition}{Definition}[section]
\newtheorem{corollary}{Corollary}[section]
\newtheorem{identity}{Identity}[section]
\newtheorem{lemma}{Lemma}[section]
\newtheorem{result}{Result}[section]

\begin{document}
\maketitle

Conformal geometric algebra is a model of
geometry implemented in the language of geometric algebra.
This document is my attempt to rigorously build the
conformal model from the ground up.
It is only assumed that the reader is familiar with geometric algebra.
I used the books $\cite{dorst07}$ and $\cite{hestenes87}$
to learn geometric algebra and the conformal model.
I recommend them for further study.

\section{Representing Geometry}

We begin by defining how geometries are represented in the model.
Letting $\R^n$ denote $n$-dimensional Euclidean space, we will
represent geometries as subsets of this space.  Having done so,
we may perform unions, intersections and other operations of geometries, but we
have no easy means of performing any geometric analysis.  Measurements, normals,
tangents, centers, shape and other things that may characterize a geometry are not
so easily gleaned or inferred from a set of points.  This is where
geometric algebra comes in.

Letting $\G$ denote the geometric algebra to be used by our model
of geometry, we begin by letting $p:\R^n\to\G$ be a vector-valued
function of a Euclidean point, the definition of which we leave
open for the moment.  We then use this function in the following definition.
\begin{definition}\label{def_direct_rep}
For any blade $B\in\G$, we say that $B$ directly
represents a geometry as the set of points
$G(B)=\{x\in\R^n|p(x)\in B\}$.
\end{definition}
Recall that for any vector $v\in\G$, we say that $v\in B$ if and only if $v\wedge B=0$.
Clearly this means that $v$ is in the vector space spanned by any vector factorization
of $B$.  Letting $B^*$ denote a dual of $B$, it is not hard to show that
$v\in B^*$ if and only if $v\cdot B=0$.
\begin{definition}\label{def_dual_rep}
For any blade $B\in\G$, we say that $B$ dually
represents a geometry as the set of points
$G^*(B)=\{x\in\R^n|p(x)\in B^*\}$.
\end{definition}
Notice that by any one of these two definitions, if $B$ represents
a given geometry, then so does $B^*$ by the other definition.
That is, $G^*(B)=G(B^*)$.
Furthermore, any non-zero scalar multiple of $B$ is also representative
of the same geometry.  That is, for all non-zero $\lambda\in\R$,
we have $G(B)=G(\lambda B)$.

\section{Operations of Geometry}

Noticing that the set of all blades in $\G$ is closed under the
inner and outer product operations of geometric algebra,
a natural question arrises as to what geometries are represented
by the results of these operations in terms of
the geometries represented by their operands.  To begin to answer this,
we start with a theorem.
\begin{theorem}\label{thm_intersect}
For any vector $v\in\G$ and any two blades $A,B\in\G$,
if $A\wedge B\neq 0$, then $v\cdot A=0$ and $v\cdot B=0$ if and only if $v\cdot A\wedge B=0$.
\end{theorem}
\begin{proof}
\end{proof}
We can apply theorem $\eqref{thm_intersect}$ to get the following result.
\begin{result}
For any two blades $A,B\in\G$ such that $A\wedge B\neq 0$, we have
\begin{equation*}
G^*(A)\cap G^*(B) = G^*(A\wedge B).
\end{equation*}
\end{result}
Interestingly, we see here that the outer product gives the
dual representation of the intersection between the two geometries
dually represented by the blades taken in that product.
\begin{theorem}\label{thm_union_and_more}
For any vector $v\in\G$ and any two blades $A,B\in\G$,
if $v\wedge A=0$ or $v\wedge B=0$, then $v\wedge A\wedge B=0$.
\end{theorem}
\begin{proof}
If $A\wedge B=0$, then we're done.  If $A\wedge B\neq 0$ and $v\in A$ or $v\in B$,
then $v\in A\cup B$, which is to say that $v$ is in the union of vector sub-spaces
represented by $A$ and $B$.
\end{proof}
Applying theorem $\eqref{thm_union_and_more}$, we get the following result.
\begin{result}
For any two blades $A,B\in\G$, we have
\begin{equation*}
G(A)\cup G(B)\subseteq G(A\wedge B).
\end{equation*}
\end{result}
Here we see that the outer product gives the direct representation
of a geometry that is at least the union of the geometries directly
represented by the blades taken in the product.  Unlike the intersection
result given earlier, however, here we cannot come to any certain conclusion
about what is being represented, even if we know exactly what
geometries are being represented by the operands of the operation.
To resolve this, we'll find a relationship between the geometries
generated through the use of the intersection operation and the geometries
generated through the use of the union-like operation.

\section{Generating Geometry}

We will now give an explicit formula for $p(x)$, but first we must
embed $\R^n$ in $\G$.  We do this by replacing $\R^n$
with an $n$-dimensional Euclidean vector space $\V^n$, and
make the geometric algebra generated by this vector space
a sub-algebra of $\G$.  Specifically, if $\{e_k\}_{k=1}^n$ is
any set of $n$ basis vectors for $\V^n$, then the set of
basis vectors for a vector space $\V$ generating $\G$
will be given by $\{\nvao,\nvai\}\cup\{e_k\}_{k=1}^n$.
Here, $\nvao$ and $\nvai$ are refered to as the null vectors
at the origin and infinity, respectively.
\begin{definition}
For any vector $v\in\V$, if $v\cdot v=0$, we call $v$ a null vector.
\end{definition}
The null vectors $\nvao$ and $\nvai$ obey the relationship
$\nvai\cdot\nvao=-1$.  Furthermore, for
all vectors $v\in\V^n$, we define $v\cdot\nvao=0$ and $v\cdot\nvai=0$.

Having precisely defined our geometric algebra $\G$, we define
$p:\V^n\to\G$ as follows.
\begin{equation*}
p(x) = \nvao + x + \frac{1}{2}x^2\nvai
\end{equation*}
It is now not hard to show that for any $x\in\V^n$, the vector $p(x)$
both directly and dually represents the Euclidean point $x$.  We leave
this as an exercise
for the reader, as well as showing that for any scalar $r>0$, that
$p(x)-\frac{1}{2}r^2\nvai$ dually represents an $n$-dimensional hyper-sphere
at $x$ with radius $r$.  The reader should also convince themselves that a
vector of the form $v+(x\cdot v)\nvai$ dually represents an $(n-1)$-dimensional
hyper-plane containing the
point $x$ and being orthogonal to the unit-normal $v\in\V^n$.

Now having blades that represent the spheres and planes in the highest possible
dimensions of interest in $n$-dimensional Euclidean space, let us now apply
the intersection result of the previous section to generate as many round
and flat geometries as we can.  Doing so, we see that we can generate
hyper-spheres and hyper-planes of dimensions $0$ through $n-1$ as
outer products of vectors.  For $n=3$, the following table summerizes
the geometries we find and the grades of the blades dually representing them.
\begin{equation*}
\begin{array}{ccccccc}
\mbox{Grade} & \vline & \mbox{Degenerate Dual Round} & \vline & \mbox{Dual Round} & \vline & \mbox{Dual Flat} \\
\hline
1 & \vline & \mbox{Point} & \vline & \mbox{Sphere} & \vline & \mbox{Plane} \\
2 & \vline & \mbox{Tangent-Point} & \vline & \mbox{Circle} & \vline & \mbox{Line} \\
3 & \vline & \mbox{Tangent-Point} & \vline & \mbox{Point-Pair} & \vline & \mbox{Flat-Point}
\end{array}
\end{equation*}
There is nothing more or less that characterizes a flat-point in comparison
to a regular point, which may be thought of as a round-point, also being
a degenerate sphere (a sphere of radius zero).  Flat-points are called flat,
because they're the first entry in the list of flat geometries in order of
increasing dimension.  (Flat-point, line, plane, hyper-plane, etc.)

At first glance, the point-pair may seem out-of-place, but it is simply
the 1-dimensional analog of a sphere or circle.  It has a center and
a radius, but only two points.

The dual tangent point of grade 2 is a degenerate
circle, and the dual tangent point of grade 3 is a degenerate point-pair.
These occur when we intersect a plane with one point of a round,
which is why they're called tangent points.

The question of what other geometric representations we may discover in the conformal model
will be left open for now as we continue on, content with what we have
found so far.  Let us now turn our attention to the method of generating
geometries using the union-like method of the previous section.  What we'll
find is that we can generate the above geometries using this method.
We start with a definition.
\begin{definition}
For all $m\geq 0$, we say that the $m+2$ points $\{x_k\}_{k=1}^{m+2}\subset\V^n$
are co-$m$-hyper-planar under the following circumstances.
\begin{equation*}
\begin{array}{l}
\mbox{For $m=0$, the points are identical.} \\
\mbox{For $m=1$, the points are co-linear.} \\
\mbox{For $m=2$, the points are co-planar.} \\
\mbox{For $m=3$, the points are co-hyper-planar.} \\
\mbox{etc.}
\end{array}
\end{equation*}
Here, $m$ corresponds to the dimension of the flat upon which all $m+2$
points lie.
\end{definition}
We then need the following theorem.
\begin{theorem}\label{thm_fit_round}
For any set of $m\geq 2$ points $\{x_k\}_{k=1}^m\subset\V^n$, if these
$m$ points are non-co-$(m-2)$-hyper-planar, then the set of $m$ vectors in $\{p(x_k)\}_{k=1}^m$
are linearly independent.
\end{theorem}
\begin{proof}
\end{proof}
Using this theorem, it is now not hard to show that blades directly representative
of non-degenerate rounds of the conformal model have factorizations in
terms of vectors representative of points.  To see this, let $B\in\G$ be
a blade directly representative of an $m$-dimensional round, where $m>0$.  Now convince yourself
that $m+1$ points can be found on the surface of this round that are also
non-co-$(m-1)$-hyper-planar.  The vectors representative of these
points are therefore linearly independent (by theorem $\eqref{thm_fit_round}$) and
in the vector space represented by $B$ (by definition $\eqref{def_direct_rep}$).
All that remains then, to show that $B$ is a scalar multiple of the outer product
of these vectors, is that the grade of $B$ is $m+1$.  Knowing that the round in question
here is $m$-dimensional, we see that $B^*$ is of grade $n-m+1$.  (We learned this from our
study of intersecting dual geometries.)
The grade of $B$ is therefore $n+2-(n-m+1)=m+1$.

For the case $m=0$, notice that the 0-dimensional round is the degenerate $n$-dimensional round
or point.  A factorization is trivially known as a vector representative of the point.

We now see that we can build up the rounds of the conformal model using
the outer product of vectors representative of points.  In fact, we now see
that it may be more accurate to think of this as a fitting operation instead of
a union-like operation.  Observe that the conformal model not only easily and
naturally solves the problem of fitting an $m$-dimensional hyper-sphere
to a set of $m+1$ points, but also allows us to think of the blade directly
representative of that sphere in terms of any appropriate factorization
of vectors representative of points on that sphere.  We
can choose any $m+1$ points on the sphere to be in the outer product, provided they uniquely
determine the sphere.  We'll see examples of how this idea is
useful when we later solve certain problems using the conformal model.

Of course, not all sets of $m+2$ points determine
an $m$-dimensional hyper-sphere.  In the cases where these points don't determine a sphere, what do
we get?  To answer this question, we need to start with another definition.
%Well, we already know that if the $m+1$ points are co-$(m-1)$-hyper-spherical,
%then the outer product of the vectors directly representative of those points must be zero.
\begin{definition}
For all $m\geq 0$, we say that the $m+2$ points $\{x_k\}_{k=1}^{m+2}\subset\V^n$
are co-$m$-hyper-spherical under the following circumstances.
\begin{equation*}
\begin{array}{l}
\mbox{For $m=0$, the points are identical.} \\
\mbox{For $m=1$, the points are co-point-pair.} \\
\mbox{For $m=2$, the points are co-circular.} \\
\mbox{For $m=3$, the points are co-spherical.} \\
\mbox{For $m=4$, the points are co-hyper-spherical.} \\
\mbox{etc.}
\end{array}
\end{equation*}
Here, $m$ corresponds to the dimension of the non-degenerate round upon
which all $m+2$ points lie.
\end{definition}

With this definition in place, consider $B\in\G$ as a blade directly representative
of an $m$-dimensional flat, where $m>0$.  Now convince yourself that $m+2$ points can
be found on the surface of this flat that are non-co-$(m-1)$-hyper-planar
and non-co-$m$-hyper-spherical.  By the first of these two conditions,
we know that there exists a subset of size $m+1$ of the $m+2$ points that
determines an $m$-dimensional hyper-sphere in the $m$-dimensional hyper-plane.
The second of these two conditions insures that the outer product
of the blade directly representative of this $m$-dimensional round
with the vector representative of the remaining point of the $m+2$
points is non-zero.  It follows that the vectors representative
of the $m+2$ points form a linearly independent set.  Then since
these points are on the hyper-plane, all that remains to be shown
to see that the outer product of the vectors representative of these
points is a scalar multiple of $B$ is to show that the grade of $B$ is $m+2$.
Knowing that the flat in question here is $m$-dimensional, we see that $B^*$
is of grade $n-m$.  (Again, we learned this from our study of intersecting
dual geometries.)  The grade of $B$ is therefore $n+2-(n-m)=m+2$.

For the case $m=0$, the case of flat-points, this argument doesn't work since clearly
one cannot find two unique points on a point.  Fortunately, a bit of work will show
that a flat point is directly represented by a 2-blade of the form $B=\lambda(i+xi\wedge\nvai)I$,
which simplifies to $B=\lambda(1-x\wedge\nvai)\nvao\wedge\nvai$.  (Here, $i$ is
the unit psuedo-scalar of the geometric algebra genearted by $V^n$ and $I$ is
the unit psuedo-scalar of $\G$.)  It follows that
$\nvai\in B$.  Then since we can clearly find a vector representative of a point
that is on the flat-point, we see that $B$ factors as a scalar multiple of the outer
product of this vector and $\nvai$.  (Note that no vector representative of a point
is a scalar multiple of $\nvai$.)

Interestingly, what we've learned so far is that all geometries,
with the exception of flat points, can be written as outer products
of vectors representative of points.  Our next result, however, will
show that we can represent direct flat geometries in what might be
considered a more convenient way.
\begin{theorem}\label{thm_round_to_flat}
If a blade $B\in\G$ directly represents an $m$-dimensional round,
then $B\wedge\nvai$ directly represents the $m$-dimensional flat
containing this $m$-dimensional round.
\end{theorem}
\begin{proof}
\end{proof}

\section{Solving for Geometry}

Knowing how geometries of the conformal model factor in terms of vectors
representative of points leads us to one of the reaons why the conformal model
is a powerful analytical tool in geometry.  Specifically, if we're given two blades $A,B\in\G$ that
we know are both directly representative of the same non-point geometry, then we can easily show
that $A$ is a scalar multiple of $B$.  Let us state this formally with a theorem.
\begin{theorem}\label{thm_same_geos}
For any two blades $A,B\in\G$, if $G(A)=G(B)$ and these are not singletons,
then there exists a scalar $\lambda\in\R$ such that $A=\lambda B$.
\end{theorem}
\begin{proof}
With the exception of points and flat-points,
if $A$ and $B$ both directly represent the same geometry,
then any factorization of $A$ in terms of vectors representative of points will
also be, up to scale, a factorization of $B$.
\end{proof}
This is a powerful result, because the formulation
of $A$ may have been made one way, while the formulation of $B$, another, and now
we have found a way to relate the two formulations.  For example,
we might formulate $A$ as the intersection between two spheres.  Our result then tells us
that we can interpret $A$ as we would write the geometry represented by $A$ in a
canonical form $B$.  After composing $A$, we can decompose it as we would $B$.

Right away we can use theorem $\eqref{thm_same_geos}$ to come up with an important result.
\begin{theorem}\label{thm_flat_xor_round}
A blade $B\in\G$ directly represents a flat geometry if and only if
$B\wedge\nvai=0$.
\end{theorem}
\begin{proof}
By the previous section and theorem $\eqref{thm_round_to_flat}$, if the blade $B$ directly represents a round,
then $B\wedge\nvai$ represents a flat.  But also by the previous section, we know that the
flat directly represented by $B\wedge\nvai$ can also be directly represented by an outer
product of vectors representative of points on the flat.  But by theorem $\eqref{thm_same_geos}$, these blades
are scalar multiples of one another.  Therefore, in any case, the outer product of $\nvai$
and a factorization of this flat must be zero.

Prove other direction here.
\end{proof}
Notice that this theorem can also be stated as follows.  A blade $B\in\G$
directly represents a round geometry if and only if $B\wedge\nvai\neq 0$.
This is because every geometry is either round or flat.

Another useful feature of the conformal model comes from the way it lets
us think about doing operations at a high level.  We needed only descend to
the lower levels of thinking to develop the model.  Once developed, what we can do now is illustrated
by the following example.  Suppose we're given a dual circle $A$ and
a point $B$, and we want to find the dual sphere $C$ fitting these two geometries.
Well, we can think of $A^*$ as any three points determining the circle.
Combining this in the outer product with $C$, we then see that we
get what may be four points that determine the desired sphere.
Finally, we can come to the conclusion that $C=(A^*\wedge B)^*=A\cdot B$,
which is a nice result!  Our answer is simply the inner product of the two blades
representing the geometries in question.  Furthermore, the blade $C$ gives
us useful information in all situations.  If $C=0$, then $B$ was on $A$.
If $C\wedge\nvai=0$, then the sphere is really a plane, because its
centered at infinity with radius infinity.  In the remaining case, $C$ is
a finite sphere.

\section{Transforming Geometry}

Address transformations by versors here.

Versors transform the blades representative of geometries in the conformal model,
and so versrors represent transformations.

\subsection{Uniform Scaling}

\subsection{Rotations}

\subsection{Translations}

\subsection{Toroidal Rotations}

\subsection{Transversions}

\subsection{Hyperbolic Scaling}

\section{Geometries as Transformers}

Interesting things happen when we use geometries as we would versors
to transform other geometries.  Address this here.

\section{Catalog of Dual Representations}

For reference, this section catalogs canonical dual representations of the geometries in the
conformal model of 3-dimensional space.  In each sub-section, the blade $B\in\G$ is
assumed to represent the geometry in question.  In addition to the composition
of each geometry's dual representation, a sequence of steps
are also provided that show how one can decompose this representation
into the variables that characterize the geometry.

\subsection{Points}

Points (round points) are characterized by a Euclidean point $x\in\V^n$ and
a non-zero scalar (weight) $\lambda\in\R$.
\begin{equation*}
B = \lambda\left(\nvao + x + \frac{1}{2}x^2\nvai\right)
\end{equation*}
We may decompose this as follows.
\begin{align*}
\lambda &= -\nvai\cdot B \\
v &= \nvao\wedge\nvai\cdot\frac{B}{\lambda}\wedge\nvao\wedge\nvai
\end{align*}

\subsection{Spheres}

Spheres are characterized by a Euclidean point (center) $x\in\V^n$, a
non-zero radius $r\in\R$ and a non-zero scalar (weight) $\lambda\in\R$.
\begin{equation*}
B = \lambda\left(\nvao + x + \frac{1}{2}(x^2\pm r^2)\nvai\right)
\end{equation*}
(Say something about imaginary spheres.)
We may decompose this as follows.
\begin{align*}
\lambda &= -\nvai\cdot B \\
x &= \nvao\wedge\nvai\cdot\frac{B}{\lambda}\wedge\nvao\wedge\nvai \\
r^2 &= x^2 + 2\nvao\cdot\frac{B}{\lambda}
\end{align*}

\subsection{Planes}

Planes are characterized by a Euclidean point (center, if you will) $x\in\V^n$,
a unit-normal $v\in\V^n$ and a non-zero scalar (weight) $\lambda\in\R$.
\begin{equation*}
B = \lambda(v + (x\cdot v)\nvai)
\end{equation*}
If $T=1-\frac{1}{2}x\nvai$, we may also formulate $B$ as $Tv\tilde{T}$.
We may decompose $B$ as follows.
\begin{align*}
v &= \nvao\cdot\frac{B}{\lambda}\wedge\nvai \\
x &= -v\left(\nvao\cdot\frac{B}{\lambda}\right)
\end{align*}
Notice here that any original weight, normal and position used in
the composition of $B$ are not recoverable in the decomposition of $B$.
Here, $x$ will be the point on the plane closest to the origin.

\subsection{Circles}

Circles are characterized by a Euclidean point (center) $x\in\V^n$,
a unit-normal $v\in\V^n$, a non-zero radius $r\in\R$ and a non-zero scalar (weight) $\lambda\in\R$.
\begin{equation*}
B = \lambda(v+(x\cdot v)\nvai)\wedge\left(\nvao+x+\frac{1}{2}(x^2\pm r^2)\nvai\right)
\end{equation*}
(Say something about imaginary circles.)  We may decompose this as follows.
\begin{align*}
v &= \nvao\wedge\nvai\cdot\frac{B}{\lambda}\wedge\nvai \\
x &= v\left(\nvao\wedge\nvai\cdot\frac{B}{\lambda}\wedge\nvao\nvai\right) \\
r^2 &= x^2 - 2v\left((x\cdot v)x-\nvao\wedge\nvai\cdot\nvao\wedge \frac{B}{\lambda}\right)
\end{align*}

\subsection{Lines}

Lines are characterized by a Euclidean point (center, if you will) $x\in\V^n$,
a unit-normal $v\in\V^n$ and a non-zero scalar (weight) $\lambda\in\R$.
\begin{equation*}
B = \lambda\left(vi - \left(x\cdot vi\right)\wedge\nvai\right)
\end{equation*}
We may decompose this as follows.
\begin{align*}
v &= \left(\nvao\cdot\frac{B}{\lambda}\wedge\nvai\right)i \\
x &= -v\left(\nvao\cdot\frac{B}{\lambda}\right)i
\end{align*}

\subsection{Point-Pairs}

Point-pairs are characterized by a Euclidean point (center) $x\in\V^n$,
a unit-normal $v\in\V^n$, a non-zero radius $r\in\R$ and a non-zero scalar (weight) $\lambda\in\R$.
\begin{equation*}
B = \lambda(vi-(x\cdot vi)\wedge\nvai)\wedge\left(\nvao+x+\frac{1}{2}(x^2\pm r^2)\nvai\right)
\end{equation*}
(Say something about imaginary point-pairs.)  We may decompose this as follows.
\begin{align*}
v &= -\left(\nvao\wedge\nvai\cdot\frac{B}{\lambda}\wedge\nvai\right)i \\
x &= -v\left(\nvao\wedge\nvai\cdot\frac{B}{\lambda}\wedge\nvao\nvai\right)i \\
r^2 &= -x^2+2v\left((x\cdot v)v+\left(\nvao\wedge\nvai\cdot\nvao\wedge\frac{B}{\lambda}\right)i\right)
\end{align*}

\subsection{Flat Points}

Flat-points are characterized by a Euclidean point $x\in\V^n$ and a non-zero
scalar (weight) $\lambda\in\R$.
\begin{equation*}
B = \lambda(i+xi\wedge\nvai)
\end{equation*}
We may decompose this as follows.
\begin{align*}
\lambda &= -(B\wedge\nvai)i \\
x &= \left(\nvao\cdot\frac{B}{\lambda}\right)i
\end{align*}

\subsection{Tangent Points}

A tangent-point is characterized by a Euclidean point $x\in\V^n$, a unit-normal $v\in\V^n$
and a non-zero scalar (weight) $\lambda\in\R$.  Dual canonical forms of tangent points for
grades 2 and 3 are given by the dual canonical forms of circles and point-pairs, respectively,
with a radius $r$ of zero.  For example, given any $r>0$, simplyfing the following equation recovers
the dual form of a tangent point for grade 2.
\begin{equation*}
B = \lambda(v+(x\cdot v)\nvai)\wedge\left(\nvao+x-rv+\frac{1}{2}((x-rv)^2-r^2)\nvai\right),
\end{equation*}
The reader will notice that $r$ cancels itself out.  The decomposition steps for tangent
points are the same as those given for circles and point-pairs.  The recovered radius
will be zero in the case of tangent points.

\subsection{Free Blades}

Address free-blades here.

\section{Catalog of Transformations}

\subsection{Scale-Rotate-Translation Transformations}

Such a transformation is characterized by a Euclidean translation vector $t\in\V^n$, a unit-axis $a\in\V^n$,
an angle $\theta\in\R$ and a scalar $\lambda\in\R$.
\begin{equation*}
V = \lambda\left(1-\frac{1}{2}t\nvai\right)\left(\cos\frac{\theta}{2}-ai\sin\frac{\theta}{2}\right),
\end{equation*}
Notice that $V$ here is not a blade.  It is an even versor.  If the blade $B\in\G$ represents
a geometry, (directyl or dually), the transformation of $B$ by $V$ is given by $VBV^{-1}$,
in the case that we wish to the apply the rotation first, then the translation.  The application
of the scale may be thought of as happening before or after the rotation, but not hafter the translation.
We may decompose this type of transformation as follows.
\begin{align*}
\lambda^2 &= VV^{\sim} \\
R &= -\nvao\cdot\frac{V}{\lambda}\wedge\nvai \\
T &= \frac{V}{\lambda}R^{\sim} \\
\theta &= 2\cos^{-1}\left\langle R\right\rangle_0 \\
a &= \frac{1}{\sin(\theta/2)}\left\langle R\right\rangle_2 i \\
t &= 2\nvao\cdot(1-T)
\end{align*}
(Say something about the polar decomposition of $V$.)

\bibliographystyle{plain}
\bibliography{CGAIntro}

\end{document}

% How does meet and join perform operations on geometry in the conformal model?