%-----------------------------------------------------------------------
% Beginning of ecgd-l-template.tex
%-----------------------------------------------------------------------
%
%     This is a topmatter template file for ECGD for use with AMS-LaTeX.
%
%     Templates for various common text, math and figure elements are
%     given following the \end{document} line.
%
%%%%%%%%%%%%%%%%%%%%%%%%%%%%%%%%%%%%%%%%%%%%%%%%%%%%%%%%%%%%%%%%%%%%%%%%

%     Remove any commented or uncommented macros you do not use.

\documentclass{ecgd-l}

%     If you need symbols beyond the basic set, uncomment this command.
%\usepackage{amssymb}

%     If your article includes graphics, uncomment this command.
%\usepackage{graphicx}

%     If the article includes commutative diagrams, ...
%\usepackage[cmtip,all]{xy}

\usepackage{url}

%     Update the information and uncomment if AMS is not the copyright
%     holder.
%\copyrightinfo{2009}{American Mathematical Society}

\newtheorem{theorem}{Theorem}[section]
\newtheorem{lemma}[theorem]{Lemma}

\theoremstyle{definition}
\newtheorem{definition}[theorem]{Definition}
\newtheorem{example}[theorem]{Example}
\newtheorem{xca}[theorem]{Exercise}

\theoremstyle{remark}
\newtheorem{remark}[theorem]{Remark}

\numberwithin{equation}{section}

\newcommand{\G}{\mathbb{G}}
\newcommand{\V}{\mathbb{V}}
\newcommand{\Vb}{\mathbb{\overline{V}}}
\newcommand{\W}{\mathbb{W}}
\newcommand{\R}{\mathbb{R}}

\begin{document}

% \title[short text for running head]{full title}
\title{The Quadric Model Of Geometric Algebra}

%    Only \author and \address are required; other information is
%    optional.  Remove any unused author tags.

%    author one information
% \author[short version for running head]{name for top of paper}
\author{Spencer T. Parkin}
\address{}
\curraddr{}
\email{spencer.parkin@gmail.com}
\thanks{}

%    author two information
%\author{}
%\address{}
%\curraddr{}
%\email{}
%\thanks{}

%    \subjclass is required.
\subjclass[2010]{Primary }

\date{}

\dedicatory{}

%    Abstract is required.
\begin{abstract}
A great achievement of the conformal model of geometric algebra is that the elements
of computation are representatives of geometry, and may therefore be thought of as geometry.
A limitation of the model, however, is its inability to represent all quadric surfaces.
Set forth in this paper, the quadric model of geometric algebra, while maintaining the idea
of geometries as elements of computation, overcomes this limitation at the expense of
added complexity and dimension.
\end{abstract}

\maketitle

%    Text of article.

\nocite{DoranHestenes93}
\nocite{WikipediaQuadricEntry}

\section{Finding The Quadric Equation}

Taking our que from \cite{WikipediaQuadricEntry}, the $n$-dimensional quadric surfaces
may be characterized as the set of all projective points in an $(n+1)$-dimensional homogeneous space
satisfying a matrix equation involving a symmetric matrix.  We will let $\V^{n+1}$ be an
$(n+1)$-dimensional vector space and identify vectors in this space with projective
points of $n$-dimensional space in the usual manner.  That is, letting $\{e_i\}_{i=0}^n$
be an orthonormal basis for $\V^{n+1}$, we identify the $n$-dimensional point
represented by any $p\in\V^{n+1}$ as the point $p/(p\cdot e_0)$ in the $e_0=1$ plane.

Letting $\{\alpha_{ij}\}\subset\R$ with $0\leq i\leq j\leq n$ be the scalar elements of a symmetric
matrix, an $n$-dimensional quadric surface is the projective solution set to the
matrix equation
\begin{equation}\label{equ_quadric_matrix_equation}
0 = p\left[\begin{array}{cccc}
\alpha_{00} & \alpha_{01} & \dots & \alpha_{0n} \\
\alpha_{01} & \alpha_{11} & \dots & \alpha_{1n} \\
\vdots & \vdots & \ddots & \vdots \\
\alpha_{0n} & \alpha_{1n} & \dots & \alpha_{nn}
\end{array}\right]p^T,
\end{equation}
where here we have abused notation by interpreting the vector $p$ taken from $\V^{n+1}$ as
a row-vector with $p^T$ as the corresponding column-vector.  Written another
way without abusing notation, we have
\begin{equation}\label{equ_lowlevel_equation}
0 = \sum_{i=0}^n\sum_{j=i}^n\sigma_{ij}\alpha_{ij}(p\cdot e_i)(p\cdot e_j),
\end{equation}
where $\sigma_{ij}$ is defined as
\begin{equation}
\sigma_{ij} = \left\{\begin{array}{ll}
1 & \mbox{if $i=j$,} \\
2 & \mbox{if $i\neq j$.}
\end{array}\right.
\end{equation}
The form \eqref{equ_quadric_matrix_equation} lends itself to the study
of quadrics using matrix algebra, while the form \eqref{equ_lowlevel_equation}
may be thought of as a low-level form of the equation in geometric algebra.
What we might think of as a high-level form in geometric algebra, coming from
a framework of compuation, may provide a better means of studying quadrics
using geometric algebra.  We procede now to develop such a form.

Let $\W^{2(n+1)}$ denote a $2(n+1)$-dimensional vector space
having $\{e_i\}_{i=0}^{2n+1}$ as a set orthonormal basis vectors
generating it.  The vector space $\V^{n+1}$ is therefore a vector
sub-space of $\W^{2(n+1)}$ and we will let $\Vb^{n+1}$ denote
the $(n+1)$-dimensional vector sub-space of $\W^{2(n+1)}$ that
is complement to $\V^{n+1}$.  It is then helpful to introduce the
notation $\overline{p}$ as the vector in $\Vb^{n+1}$ related to the
vector $p\in\V^{n+1}$ by the equation
\begin{equation}
\overline{p} = Rp\tilde{R},
\end{equation}
where $R$ is a rotor defined as
\begin{equation}
R = 2^{-n/2}\prod_{i=0}^n\left(1-e_ie_{i+n+1}\right).
\end{equation}
This idea comes from \cite{DoranHestenes93}, and it is easy to see
that for any integer $i\in[0,n]$, we have $\overline{e}_i=e_{i+n+1}$
and $\overline{e}_{i+n+1}=e_i$.  Notice that the over-bar operator
is an outermorphic function and that we may apply it to any element
of the geometric algebra $\G(\W^{2(n+1)})$.

We are now ready to give the high-level form of equation \eqref{equ_lowlevel_equation}
as\footnote{Here and throughout this paper, we assume that the outer product takes
precedence over the inner product.  We also assume that the geometric product takes
precedence over the inner and outer products.}
\begin{equation}\label{equ_highlevel_equation}
0 = p\wedge\overline{p}\cdot B,
\end{equation}
where $B\in\G(\W^{2(n+1)})$ is a bivector of the form
\begin{equation}
B = -\frac{1}{2}\sum_{i=0}^n\sum_{j=i}^n\alpha_{ij}(e_i\overline{e}_j+(-1)^{\sigma_{ij}}\overline{e}_ie_j).
\end{equation}
Here, as in the form \eqref{equ_quadric_matrix_equation} where we may think of
the symmetric matrix as representative of the quadric, the bivector $B$ may also be thought
of as representative of this quadric.

Realizing that we need to be careful, because the inner product is not associative,
it is interesting to write equation \eqref{equ_highlevel_equation} in a form
similar to that of equation \eqref{equ_quadric_matrix_equation}.  Doing so, we get
\begin{equation}\label{equ_conjugation_form}
0 = p\cdot B\cdot\overline{p}.
\end{equation}
We can get away with this, because the choice of associativity here only changes
the sign of the right-hand side, and the sign of the left-hand side clearly doesn't matter.
Considering $\overline{p}$ a type of conjugate to $p$, we may refer to equation
\eqref{equ_conjugation_form} as the inner product conjugation of $B$ by $p$.

\section{Using The Quadric Equation}

Having developed the quadric equation \eqref{equ_highlevel_equation} in geometric algebra, we can
now benefit from the language of geometric algebra in using it to answer questions about quadric geometry.

Notice that in our model we can make a distinction between members of $\V^{n+1}$ that are
representative of points and those representative of directions.  Specifically, a vector $v\in\V^{n+1}$
is a direction if and only if $v\cdot e_0=0$.  While we will use an arrow accent to distinguish
between direction vectors and position vectors, there should be no confusion on the form
of a vector and what we intend it to represent when we refer to it as a direction or a position.
Similarly, we will take the liberty of referring to bivectors taken
from $\G(\V^{2(n+1)})$ as quadrics.  This helps eliminate phrases that would otherwise sound a bit
too pedantic.

\subsection{Characterizing Flat Quadrics}

Letting $f:\V^{n+1}\to\R$ be the function defined as
\begin{equation}
f(x) = x\wedge\overline{x}\cdot B,
\end{equation}
we arrive at our first lemma.
\begin{lemma}\label{lma_flat_quadric}
Given any quadric $B$, if for all direction vectors $\vec{v}\in\V^{n+1}$,
we have $f(\vec{v})=0$, then $B$ is a linear (flat) quadric.
\end{lemma}
\begin{proof}
For any pair of homogenized points $p,x\in\V^{n+1}$, there is a direction $\vec{x}\in\V^{n+1}$
such that $x=p+\vec{x}$.  We then find that
\begin{equation}
f(x) = f(p+\vec{x}) = \nabla_{\vec{x}}f(p),
\end{equation}
in the case that $p$ is on $B$, where $\nabla_{\vec{x}}f(p)$ is the directional
derivative of $f$ at $p$ in the direction of $\vec{x}$.  It follows that the tangent
space of any point on the quadric is also in the quadric.  The quadric is therefore
flat at any point upon its surface.
\end{proof}
Recaling that for any $x\in\V^{n+1}$, the definition of $\nabla f(x)$ is given by
\begin{equation}
\nabla f(x) = \sum_{i=0}^n e_i\nabla_{e_i}f(x),
\end{equation}
it is not hard to show that for any vector $y\in\V^{n+1}$, we have
$y\cdot\nabla f(x)=\nabla_y f(x)$.  Seeing that $\vec{x}\cdot\nabla f(p)=\nabla_{\vec{x}}f(p)$
in the light of Lemma~\ref{lma_flat_quadric},
it follows that the direction $\vec{v}=e_0\cdot e_0\wedge\nabla f(p)$
is normal to the surface of the quadric $B$ at $p$.  We can then formulate
the quadric $A$ that is the plane tangent to $B$ at $p$ as follows.
\begin{equation}
A = (p\cdot\vec{v})e_0\overline{e}_0 + e_0\overline{\vec{v}}+\overline{e}_0\vec{v}
\end{equation}
A better way to formulate planes will be found in the next section.

\subsection{Quadric 2-Blades}

All quadrics are bivectors, but not all bivectors are 2-blades.  Here
we study the class of quadrics that are 2-blades.  For any four
points $a,b,c,d\in\V^{n+1}$, such a quadric $B$ has the form
\begin{align}
B &= (a+\overline{b})\wedge(c+\overline{d})\label{equ_quadric_twoblade} \\
 &= a\wedge c + a\wedge\overline{d} + \overline{b}\wedge c + \overline{b\wedge d}.\label{equ_quadric_twoblade_expanded}
\end{align}
It is curious to think what geometric significance the quadric $B$ has in relation
to these four points.  Whatever the case may be, it is clear from equation
\eqref{equ_quadric_twoblade_expanded} that the quadric $B$ contains the intersection, if any,
of the four quadrics appearing in the sum.  Considering the three forms of 2-blades
found in the expansion of equation \eqref{equ_quadric_twoblade} to be more
fundamental, (namely, $a\wedge c$, $\overline{b\wedge d}$ and
the identical forms $a\wedge\overline{d}$ and $-c\wedge\overline{b}$), we'll
start with a treatment of each of these forms.

We first notice that the quadrics of the form $a\wedge c$ and $\overline{b\wedge d}$
trivially represent the quadric of all space.  They therefore contribute nothing
to the shape of $B$.  The remaining form $a\wedge\overline{d}$, therefore,
deserves our full attention.  We break this form into two cases, the first being
the case when $a=d$, and the second when $a\neq d$.

In the first case, a quick application of equation
\eqref{equ_highlevel_equation} reveals the type of quadric
represented by $a\wedge\overline{a}$.  Doing so, we see that it represents
the set of all projective points $p\in\V^{n+1}$ such that
\begin{equation}
0=p\wedge\overline{p}\cdot a\wedge\overline{a}=-(p\cdot a)^2,
\end{equation}
which holds if and only if $p\cdot a=0$.  In turn, this holds if and only if
$p\cdot a/(a\cdot e_0)=0$.  Letting $p=e_0+\vec{p}$ and $a=e_0+\vec{a}$,
our equation becomes
\begin{equation}
\vec{p}\cdot\frac{\vec{a}}{|\vec{a}|} = -\frac{1}{|\vec{a}|},
\end{equation}
where it is now clear that $a\wedge\overline{a}$ is a plane having
a unit-normal of $\vec{a}/|\vec{a}|$ and containing the point $e_0-\vec{a}/|\vec{a}|^2$
as the point on $a\wedge\overline{a}$ closest to the origin.

In the second case, our use of equation \eqref{equ_highlevel_equation} is not nearly as
revealing at first sight.
\begin{equation}
0 = p\wedge\overline{p}\cdot a\wedge\overline{d} = -(p\cdot a)(p\cdot d)
\end{equation}
Letting $p=e_0+\vec{p}$, $a=e_0+\vec{a}$ and $d=e_0+\vec{d}$, this
equation becomes
\begin{equation}
-1 = \vec{p}\cdot(\vec{a}+\vec{d})+(\vec{p}\cdot\vec{a})(\vec{p}\cdot\vec{d}).
\end{equation}
Can we, without loss of generality, assume that $\vec{a}+\vec{d}=0$?

%    Bibliographies can be prepared with BibTeX using amsplain,
%    amsalpha, or (for "historical" overviews) natbib style.
\bibliographystyle{amsplain}
%    Insert the bibliography data here.

\bibliography{QuadricModelOfCGA}

\end{document}

%%%%%%%%%%%%%%%%%%%%%%%%%%%%%%%%%%%%%%%%%%%%%%%%%%%%%%%%%%%%%%%%%%%%%%%%

%    Templates for common elements of a journal article; for additional
%    information, see the AMS-LaTeX instructions manual, instr-l.pdf,
%    included in the ECGD author package, and the amsthm user's guide,
%    linked from http://www.ams.org/tex/amslatex.html .

%    Section headings
\section{}
\subsection{}

%    Ordinary theorem and proof
\begin{theorem}[Optional addition to theorem head]
% text of theorem
\end{theorem}

\begin{proof}[Optional replacement proof heading]
% text of proof
\end{proof}

%    Figure insertion; default placement is top; if the figure occupies
%    more than 75% of a page, the [p] option should be specified.
\begin{figure}
\includegraphics{filename}
\caption{text of caption}
\label{}
\end{figure}

%    Mathematical displays; for additional information, see the amsmath
%    user's guide, linked from http://www.ams.org/tex/amslatex.html .

% Numbered equation
\begin{equation}
\end{equation}

% Unnumbered equation
\begin{equation*}
\end{equation*}

% Aligned equations
\begin{align}
  &  \\
  &
\end{align}

%-----------------------------------------------------------------------
% End of ecgd-l-template.tex
%-----------------------------------------------------------------------
