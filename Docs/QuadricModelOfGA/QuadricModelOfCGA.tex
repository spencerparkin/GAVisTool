%-----------------------------------------------------------------------
% Beginning of ecgd-l-template.tex
%-----------------------------------------------------------------------
%
%     This is a topmatter template file for ECGD for use with AMS-LaTeX.
%
%     Templates for various common text, math and figure elements are
%     given following the \end{document} line.
%
%%%%%%%%%%%%%%%%%%%%%%%%%%%%%%%%%%%%%%%%%%%%%%%%%%%%%%%%%%%%%%%%%%%%%%%%

%     Remove any commented or uncommented macros you do not use.

\documentclass{ecgd-l}

%     If you need symbols beyond the basic set, uncomment this command.
%\usepackage{amssymb}

%     If your article includes graphics, uncomment this command.
%\usepackage{graphicx}

%     If the article includes commutative diagrams, ...
%\usepackage[cmtip,all]{xy}

\usepackage{url}

%     Update the information and uncomment if AMS is not the copyright
%     holder.
%\copyrightinfo{2009}{American Mathematical Society}

\newtheorem{theorem}{Theorem}[section]
\newtheorem{lemma}[theorem]{Lemma}

\theoremstyle{definition}
\newtheorem{definition}[theorem]{Definition}
\newtheorem{example}[theorem]{Example}
\newtheorem{xca}[theorem]{Exercise}

\theoremstyle{remark}
\newtheorem{remark}[theorem]{Remark}

\numberwithin{equation}{section}

\newcommand{\G}{\mathbb{G}}
\newcommand{\V}{\mathbb{V}}
\newcommand{\W}{\mathbb{W}}
\newcommand{\R}{\mathbb{R}}

\begin{document}

% \title[short text for running head]{full title}
\title{The Quadric Model Of Geometric Algebra}

%    Only \author and \address are required; other information is
%    optional.  Remove any unused author tags.

%    author one information
% \author[short version for running head]{name for top of paper}
\author{Spencer T. Parkin}
\address{}
\curraddr{}
\email{spencer.parkin@disney.com}
\thanks{}

%    author two information
%\author{}
%\address{}
%\curraddr{}
%\email{}
%\thanks{}

%    \subjclass is required.
\subjclass[2010]{Primary }

\date{}

\dedicatory{}

%    Abstract is required.
\begin{abstract}
A great achievement of the conformal model of geometric algebra is that the elements
of computations are representative of geometry, and may therefore be thought of as geometry.
A limitation of the model, however, is its inability to represent all quadric surfaces.
Set forth in this paper, the quadric model of geometric algebra, while maintaining the idea
of geometries as elements of computation, overcomes this limitation at the expense of
added complexity and dimension.
\end{abstract}

\maketitle

%    Text of article.

\nocite{DoranHestenes93}
\nocite{WikipediaQuadricEntry}

\section{Representing Quadric Surfaces}

We begin with a description of all $D$-dimensional quadric surfaces as
given in \cite{WikipediaQuadricEntry}.  Letting $\V^{D+1}$ be a
$(D+1)$-dimensional vector space, we may describe a $D$-dimensional
quadric surface as the set of all projective points $p\in\V^{D+1}$
such that
\begin{equation}\label{equ_quadric}
0 = \sum_{i=0}^{D}\sum_{j=0}^{D}\alpha_{ij}(p\cdot e_i)(p\cdot e_j),
\end{equation}
where $\{e_i\}_{k=0}^{D}$ is an orthonormal basis generating $\V^{D+1}$,
and the set of scalars $\{\alpha_{ij}\}$ characterize the surface.
We will let $p/(p\cdot e_0)$ be the homogenization of $p$ as a point in $D$-dimensional
Euclidean space.  In this way, all such Euclidean points satisfying equation \eqref{equ_quadric}
are those on the surface of the $D$-dimensional quadric it represents.

The equation \eqref{equ_quadric} has a matrix form where, in matrix algebra, we might
consider the matrix as a representative of the quadric geometry and therefore may
also consider matrix operations
as geometric operations.  Seeking a more basis independent formulation, however, we will
appeal to geometric algebra.

Let $\{e_i\}_{k=0}^{2D+1}$ be an orthonormal basis generating the $2(D+1)$-dimensional
Euclidean vector space $\V^{2(D+1)}$ with $\V^{D+1}$ as a $(D+1)$-dimensional
vector sub-space, and let $\G\left(\V^{2(D+1)}\right)$ denote the geometric
algebra generated by this vector space.  Then, barrowing from the ideas in \cite{DoranHestenes93},
if we define the versor $R$ as a rotor given by
\begin{equation}
R = 2^{-D/2}\prod_{i=0}^D(1-e_ie_{i+D+1}),
\end{equation}
then the equation \eqref{equ_quadric} may be rewritten as
\begin{equation}\label{equ_ga_quadric}
0 = pRp\tilde{R}\cdot G,
\end{equation}
where
\begin{equation}
G = \sum_{i=0}^D\sum_{j=0}^D\alpha_{ij}e_i e_{j+D+1}.
\end{equation}
In the form \eqref{equ_ga_quadric} we may consider the bivector $G$ as
a representative of the quadric surface.  Immediately we see here that,
unlike the conformal model, elements of the form
$pRp\tilde{R}$ do not represent points in the manner set
forth by equation \eqref{equ_ga_quadric}.  To see this, simply
realize that the inner product square of any non-zero 2-blade
is non-zero.  There are no null blades in our geometric algebra.
Nevertheless, we have succeeded here in finding a form of element
in a geometric algebra that, under a given definition, represents the
variety of algebraic varieties known as quadric surfaces.

\section{Constructing Quadric Surfaces}

% What quadric do we get when combining two projective pionts in
% the outer product?

% How do you make a sphere or a plane?

\section{Operations On Quadric Surfaces}

In the conformal model of geometric algebra, all geometries
are represented by blades, and therefore, the inner and outer products have closure
in the set of all geometric representatives.  This, however, is not true of the quadric
model, and so we cannot take advantage of the same results in geometric
algebra that allowed for intersecting and combining geometries.
In this model, only addition and subtraction have closure in the set
of geometric representatives, which is the set of all bivectors.

% (This is dumb, because it's the same as adding and subtracting matrices -- no different!)

% Can we somehow combine geometries like we can in CGA?
% Can we somehow intersect geometries like we can in CGA?
% We only have closure through addition.

% Not all bivectors are blades!

% It may still be useful to consider the class of quadrics
% that can be written as blades of the form a^b where
% a is in V and b is in W, the complement of V.

% A better model for quadric probably uses a non-Euclidean GA.

%    Bibliographies can be prepared with BibTeX using amsplain,
%    amsalpha, or (for "historical" overviews) natbib style.
\bibliographystyle{amsplain}
%    Insert the bibliography data here.

\bibliography{QuadricModelOfCGA}

\end{document}

%%%%%%%%%%%%%%%%%%%%%%%%%%%%%%%%%%%%%%%%%%%%%%%%%%%%%%%%%%%%%%%%%%%%%%%%

%    Templates for common elements of a journal article; for additional
%    information, see the AMS-LaTeX instructions manual, instr-l.pdf,
%    included in the ECGD author package, and the amsthm user's guide,
%    linked from http://www.ams.org/tex/amslatex.html .

%    Section headings
\section{}
\subsection{}

%    Ordinary theorem and proof
\begin{theorem}[Optional addition to theorem head]
% text of theorem
\end{theorem}

\begin{proof}[Optional replacement proof heading]
% text of proof
\end{proof}

%    Figure insertion; default placement is top; if the figure occupies
%    more than 75% of a page, the [p] option should be specified.
\begin{figure}
\includegraphics{filename}
\caption{text of caption}
\label{}
\end{figure}

%    Mathematical displays; for additional information, see the amsmath
%    user's guide, linked from http://www.ams.org/tex/amslatex.html .

% Numbered equation
\begin{equation}
\end{equation}

% Unnumbered equation
\begin{equation*}
\end{equation*}

% Aligned equations
\begin{align}
  &  \\
  &
\end{align}

%-----------------------------------------------------------------------
% End of ecgd-l-template.tex
%-----------------------------------------------------------------------
